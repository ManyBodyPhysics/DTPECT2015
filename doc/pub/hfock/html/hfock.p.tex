%%
%% Automatically generated file from DocOnce source
%% (https://github.com/hplgit/doconce/)
%%
% #ifdef PTEX2TEX_EXPLANATION
%%
%% The file follows the ptex2tex extended LaTeX format, see
%% ptex2tex: http://code.google.com/p/ptex2tex/
%%
%% Run
%%      ptex2tex myfile
%% or
%%      doconce ptex2tex myfile
%%
%% to turn myfile.p.tex into an ordinary LaTeX file myfile.tex.
%% (The ptex2tex program: http://code.google.com/p/ptex2tex)
%% Many preprocess options can be added to ptex2tex or doconce ptex2tex
%%
%%      ptex2tex -DMINTED myfile
%%      doconce ptex2tex myfile envir=minted
%%
%% ptex2tex will typeset code environments according to a global or local
%% .ptex2tex.cfg configure file. doconce ptex2tex will typeset code
%% according to options on the command line (just type doconce ptex2tex to
%% see examples). If doconce ptex2tex has envir=minted, it enables the
%% minted style without needing -DMINTED.
% #endif

% #define PREAMBLE

% #ifdef PREAMBLE
%-------------------- begin preamble ----------------------

\documentclass[%
twoside,                 % oneside: electronic viewing, twoside: printing
final,                   % or draft (marks overfull hboxes, figures with paths)
10pt]{article}

\listfiles               % print all files needed to compile this document

\usepackage{relsize,makeidx,color,setspace,amsmath,amsfonts}
\usepackage[table]{xcolor}
\usepackage{bm,microtype}

\usepackage{ptex2tex}

\usepackage[T1]{fontenc}
%\usepackage[latin1]{inputenc}
\usepackage[utf8]{inputenc}

\usepackage{lmodern}         % Latin Modern fonts derived from Computer Modern

% Hyperlinks in PDF:
\definecolor{linkcolor}{rgb}{0,0,0.4}
\usepackage[%
    colorlinks=true,
    linkcolor=linkcolor,
    urlcolor=linkcolor,
    citecolor=black,
    filecolor=black,
    %filecolor=blue,
    pdfmenubar=true,
    pdftoolbar=true,
    bookmarksdepth=3   % Uncomment (and tweak) for PDF bookmarks with more levels than the TOC
            ]{hyperref}
%\hyperbaseurl{}   % hyperlinks are relative to this root

\setcounter{tocdepth}{2}  % number chapter, section, subsection

\usepackage[framemethod=TikZ]{mdframed}

% --- begin definitions of admonition environments ---

% Admonition style "mdfbox" is an oval colored box based on mdframed
% "notice" admon
\colorlet{mdfbox_notice_background}{gray!5}
\newmdenv[
  skipabove=15pt,
  skipbelow=15pt,
  outerlinewidth=0,
  backgroundcolor=mdfbox_notice_background,
  linecolor=black,
  linewidth=2pt,       % frame thickness
  frametitlebackgroundcolor=mdfbox_notice_background,
  frametitlerule=true,
  frametitlefont=\normalfont\bfseries,
  shadow=false,        % frame shadow?
  shadowsize=11pt,
  leftmargin=0,
  rightmargin=0,
  roundcorner=5,
  needspace=0pt,
]{notice_mdfboxmdframed}

\newenvironment{notice_mdfboxadmon}[1][]{
\begin{notice_mdfboxmdframed}[frametitle=#1]
}
{
\end{notice_mdfboxmdframed}
}

% Admonition style "mdfbox" is an oval colored box based on mdframed
% "summary" admon
\colorlet{mdfbox_summary_background}{gray!5}
\newmdenv[
  skipabove=15pt,
  skipbelow=15pt,
  outerlinewidth=0,
  backgroundcolor=mdfbox_summary_background,
  linecolor=black,
  linewidth=2pt,       % frame thickness
  frametitlebackgroundcolor=mdfbox_summary_background,
  frametitlerule=true,
  frametitlefont=\normalfont\bfseries,
  shadow=false,        % frame shadow?
  shadowsize=11pt,
  leftmargin=0,
  rightmargin=0,
  roundcorner=5,
  needspace=0pt,
]{summary_mdfboxmdframed}

\newenvironment{summary_mdfboxadmon}[1][]{
\begin{summary_mdfboxmdframed}[frametitle=#1]
}
{
\end{summary_mdfboxmdframed}
}

% Admonition style "mdfbox" is an oval colored box based on mdframed
% "warning" admon
\colorlet{mdfbox_warning_background}{gray!5}
\newmdenv[
  skipabove=15pt,
  skipbelow=15pt,
  outerlinewidth=0,
  backgroundcolor=mdfbox_warning_background,
  linecolor=black,
  linewidth=2pt,       % frame thickness
  frametitlebackgroundcolor=mdfbox_warning_background,
  frametitlerule=true,
  frametitlefont=\normalfont\bfseries,
  shadow=false,        % frame shadow?
  shadowsize=11pt,
  leftmargin=0,
  rightmargin=0,
  roundcorner=5,
  needspace=0pt,
]{warning_mdfboxmdframed}

\newenvironment{warning_mdfboxadmon}[1][]{
\begin{warning_mdfboxmdframed}[frametitle=#1]
}
{
\end{warning_mdfboxmdframed}
}

% Admonition style "mdfbox" is an oval colored box based on mdframed
% "question" admon
\colorlet{mdfbox_question_background}{gray!5}
\newmdenv[
  skipabove=15pt,
  skipbelow=15pt,
  outerlinewidth=0,
  backgroundcolor=mdfbox_question_background,
  linecolor=black,
  linewidth=2pt,       % frame thickness
  frametitlebackgroundcolor=mdfbox_question_background,
  frametitlerule=true,
  frametitlefont=\normalfont\bfseries,
  shadow=false,        % frame shadow?
  shadowsize=11pt,
  leftmargin=0,
  rightmargin=0,
  roundcorner=5,
  needspace=0pt,
]{question_mdfboxmdframed}

\newenvironment{question_mdfboxadmon}[1][]{
\begin{question_mdfboxmdframed}[frametitle=#1]
}
{
\end{question_mdfboxmdframed}
}

% Admonition style "mdfbox" is an oval colored box based on mdframed
% "block" admon
\colorlet{mdfbox_block_background}{gray!5}
\newmdenv[
  skipabove=15pt,
  skipbelow=15pt,
  outerlinewidth=0,
  backgroundcolor=mdfbox_block_background,
  linecolor=black,
  linewidth=2pt,       % frame thickness
  frametitlebackgroundcolor=mdfbox_block_background,
  frametitlerule=true,
  frametitlefont=\normalfont\bfseries,
  shadow=false,        % frame shadow?
  shadowsize=11pt,
  leftmargin=0,
  rightmargin=0,
  roundcorner=5,
  needspace=0pt,
]{block_mdfboxmdframed}

\newenvironment{block_mdfboxadmon}[1][]{
\begin{block_mdfboxmdframed}[frametitle=#1]
}
{
\end{block_mdfboxmdframed}
}

% --- end of definitions of admonition environments ---

% prevent orhpans and widows
\clubpenalty = 10000
\widowpenalty = 10000

% --- end of standard preamble for documents ---


% insert custom LaTeX commands...

\raggedbottom
\makeindex

%-------------------- end preamble ----------------------

\begin{document}

% #endif


% ------------------- main content ----------------------

% Slides for FYS-KJM4480


% ----------------- title -------------------------

\title{FYS-KJM4480/9480 Lectures: \emph{Hartree-Fock theory}}

% ----------------- author(s) -------------------------

\author{Morten Hjorth-Jensen, Department of Physics, University of Oslo, Oslo, Norway and National Superconducting Cyclotron Laboratory, Michigan State University, East Lansing, MI 48824, USA\inst{}}
\institute{}
% ----------------- end author(s) -------------------------

\date{Fall 2014
% <optional titlepage figure>
}

% !split
\subsection{Hartree-Fock theory, weeks 40-42}
\begin{block}{Week 40 }
\begin{itemize}
\item Monday:
\begin{itemize}

 \item Derivation of Hartree-Fock theory in coordinate space and Fock space

 \item Koopman's theorem

\end{itemize}

\noindent
\item Tuesday:
\begin{itemize}

 \item Stability of Hartree-Fock equations and Thouless' theorem 

\end{itemize}

\noindent
\item Exercise set five has deadline October 6
\end{itemize}

\noindent
\end{block}

% !split
\subsection{Hartree-Fock theory, weeks 40-42}
\begin{block}{Week 41 }
\begin{itemize}
\item Monday:
\begin{itemize}

 \item Stability of Hartree-Fock equation

 \item Infinite matter, the electron gas

\end{itemize}

\noindent
\item Tuesday:
\begin{itemize}

 \item Infinite matter, the electron gas and neutron star matter

\end{itemize}

\noindent
\item Exercise set five has deadline October 6. Discussion of project 1, which counts 25 $\%$ of the final mark.
\end{itemize}

\noindent
\end{block}

% !split
\subsection{Hartree-Fock and Full Configuration Interaction theory, weeks 40-42}
\begin{block}{Week 42 }
\begin{itemize}
\item Monday:
\begin{itemize}

 \item Infinite matter, the electron gas and neutron star matter

\end{itemize}

\noindent
\item Tuesday:
\begin{itemize}

 \item Infinite matter, the electron gas and neutron star matter

\end{itemize}

\noindent
\item Work on  project 1, which counts 25 $\%$ of the final mark.
\end{itemize}

\noindent
\end{block}


% !split
\subsection{Derivation of Hartree-Fock equations in coordinate space}
\begin{block}{}
Hartree-Fock (HF) theory is an algorithm for a finding an approximative expression for the ground state of a given Hamiltonian. The basic ingredients are
\begin{itemize}
  \item Define a single-particle basis $\{\psi_{\alpha}\}$ so that
\end{itemize}

\noindent
\[ 
\hat{h}^{\mathrm{HF}}\psi_{\alpha} = \varepsilon_{\alpha}\psi_{\alpha}
\]
with the Hartree-Fock Hamiltonian defined as
\[
\hat{h}^{\mathrm{HF}}=\hat{t}+\hat{u}_{\mathrm{ext}}+\hat{u}^{\mathrm{HF}}
\]
\begin{itemize}
  \item The term  $\hat{u}^{\mathrm{HF}}$ is a single-particle potential to be determined by the HF algorithm.

  \item The HF algorithm means to choose $\hat{u}^{\mathrm{HF}}$ in order to have 
\end{itemize}

\noindent
\[ \langle \hat{H} \rangle = E^{\mathrm{HF}}= \langle \Phi_0 | \hat{H}|\Phi_0 \rangle
\]
that is to find a local minimum with a Slater determinant $\Phi_0$ being the ansatz for the ground state. 
\begin{itemize}
  \item The variational principle ensures that $E^{\mathrm{HF}} \ge E_0$, with $E_0$ the exact ground state energy.
\end{itemize}

\noindent
We will show that the Hartree-Fock Hamiltonian $\hat{h}^{\mathrm{HF}}$ equals our definition of the operator $\hat{f}$ discussed in connection with the new definition of the normal-ordered Hamiltonian, that is we have, for a specific matrix element
\[
\langle p |\hat{h}^{\mathrm{HF}}| q \rangle =\langle p |\hat{f}| q \rangle=\langle p|\hat{t}+\hat{u}_{\mathrm{ext}}+|q \rangle +\sum_{i\le F} \langle pi | \hat{V} | qi\rangle_{AS},
\]
meaning that
\[
\langle p|\hat{u}^{\mathrm{HF}}|q\rangle = \sum_{i\le F} \langle pi | \hat{V} | qi\rangle_{AS}.
\]
The so-called Hartree-Fock potential $\hat{u}^{\mathrm{HF}}$ brings an explicit medium dependence due to the summation over all single-particle states below the Fermi level $F$. It brings also in an explicit dependence on the two-body interaction (in nuclear physics we can also have complicated three- or higher-body forces). The two-body interaction, with its contribution from the other bystanding fermions, creates an effective mean field in which a given fermion moves, in addition to the external potential $\hat{u}_{\mathrm{ext}}$ which confines the motion of the fermion. For systems like nuclei, there is no external confining potential. Nuclei are examples of self-bound systems, where the binding arises due to the intrinsic nature of the strong force. For nuclear systems thus, there would be no external one-body potential in the Hartree-Fock Hamiltonian. 
\end{block}

% !split
\subsection{Variational Calculus and Lagrangian Multipliers}
\begin{block}{}
The calculus of variations involves 
problems where the quantity to be minimized or maximized is an integral. 

In the general case we have an integral of the type
\[ 
E[\Phi]= \int_a^b f(\Phi(x),\frac{\partial \Phi}{\partial x},x)dx,
\]
where $E$ is the quantity which is sought minimized or maximized.
The problem is that although $f$ is a function of the variables $\Phi$, $\partial \Phi/\partial x$ and $x$, the exact dependence of
$\Phi$ on $x$ is not known.  This means again that even though the integral has fixed limits $a$ and $b$, the path of integration is
not known. In our case the unknown quantities are the single-particle wave functions and we wish to choose an integration path which makes
the functional $E[\Phi]$ stationary. This means that we want to find minima, or maxima or saddle points. In physics we search normally for minima.
Our task is therefore to find the minimum of $E[\Phi]$ so that its variation $\delta E$ is zero  subject to specific
constraints. In our case the constraints appear as the integral which expresses the orthogonality of the  single-particle wave functions.
The constraints can be treated via the technique of Lagrangian multipliers

Let us specialize to the expectation value of the energy for one particle in three-dimensions.
This expectation value reads
\[
  E=\int dxdydz \psi^*(x,y,z) \hat{H} \psi(x,y,z),
\]
with the constraint
\[
 \int dxdydz \psi^*(x,y,z) \psi(x,y,z)=1,
\]
and a Hamiltonian
\[
\hat{H}=-\frac{1}{2}\nabla^2+V(x,y,z).
\]
We will, for the sake of notational convenience,  skip the variables $x,y,z$ below, and write for example $V(x,y,z)=V$.


The integral involving the kinetic energy can be written as, with the function $\psi$ vanishing
strongly for large values of $x,y,z$ (given here by the limits $a$ and $b$), 
 \[
  \int_a^b dxdydz \psi^* \left(-\frac{1}{2}\nabla^2\right) \psi dxdydz = \psi^*\nabla\psi|_a^b+\int_a^b dxdydz\frac{1}{2}\nabla\psi^*\nabla\psi.
\]
We will drop the limits $a$ and $b$ in the remaining discussion. 
Inserting this expression into the expectation value for the energy and taking the variational minimum  we obtain
\[
\delta E = \delta \left\{\int dxdydz\left( \frac{1}{2}\nabla\psi^*\nabla\psi+V\psi^*\psi\right)\right\} = 0.
\]
The constraint appears in integral form as 
\[
 \int dxdydz \psi^* \psi=\mathrm{constant},
\]
and multiplying with a Lagrangian multiplier $\lambda$ and taking the variational minimum we obtain the final variational equation
\[
\delta \left\{\int dxdydz\left( \frac{1}{2}\nabla\psi^*\nabla\psi+V\psi^*\psi-\lambda\psi^*\psi\right)\right\} = 0.
\]
We introduce the function  $f$
\[
  f =  \frac{1}{2}\nabla\psi^*\nabla\psi+V\psi^*\psi-\lambda\psi^*\psi=
\frac{1}{2}(\psi^*_x\psi_x+\psi^*_y\psi_y+\psi^*_z\psi_z)+V\psi^*\psi-\lambda\psi^*\psi,
\]
where we have skipped the dependence on $x,y,z$ and introduced the shorthand $\psi_x$, $\psi_y$ and $\psi_z$  for the various derivatives.

For $\psi^*$ the Euler-Lagrange  equations yield
\[
\frac{\partial f}{\partial \psi^*}- \frac{\partial }{\partial x}\frac{\partial f}{\partial \psi^*_x}-\frac{\partial }{\partial y}\frac{\partial f}{\partial \psi^*_y}-\frac{\partial }{\partial z}\frac{\partial f}{\partial \psi^*_z}=0,
\] 
which results in 
\[
    -\frac{1}{2}(\psi_{xx}+\psi_{yy}+\psi_{zz})+V\psi=\lambda \psi.
\]
We can then identify the  Lagrangian multiplier as the energy of the system. The last equation is 
nothing but the standard 
Schroedinger equation and the variational  approach discussed here provides 
a powerful method for obtaining approximate solutions of the wave function.
\end{block}

% !split
\subsection{Finding the Hartree-Fock functional $E[\Phi]$}
\begin{block}{}
In our discussions of Hartree-Fock theory, we will derive the equations in coordinate space first. 
Here, for the sake of simplicity,  we will  specialize to atomic physics, although the various potential energy terms can easily be changed with other ones. Using the Born-Oppenheimer approximation, the Hamiltonian reads
\[
  \hat{H} = -\sum_{i=1}^N \frac{1}{2} \nabla^2_i 
  - \sum_{i=1}^N \frac{Z}{r_i} + \sum_{i < j}^N \frac{1}{r_{ij}}.
\]
We rewrite it as
\[
    \hat{H} = \hat{H_0} + \hat{H_I} 
    = \sum_{i=1}^N\hat{h_o}(x_i) + \sum_{i < j=1}^N\frac{1}{r_{ij}},
\]
where we have defined the one-body operator as 
\[
  \hat{h}_0(x_i) = - \frac{1}{2} \nabla^2_i - \frac{Z}{r_i}.
\]
This term contains the kinetic energy of an electron with coordinates $x_i$ and the attraction on a given electron  from the atomic nucleus. 
\end{block}





% !split
\subsection{Derivation of Hartree-Fock equations in coordinate space}
\begin{block}{}
Let us denote the ground state energy by $E_0$. According to the
variational principle we have
\[
  E_0 \le E[\Phi] = \int \Phi^*\hat{H}\Phi d\mathbf{\tau}
\]
where $\Phi$ is a trial function which we assume to be normalized
\[
  \int \Phi^*\Phi d\mathbf{\tau} = 1,
\]
where we have used the shorthand $d\mathbf{\tau}=dx_1dx_2\dots dx_N$.

In the Hartree-Fock method the trial function is a Slater
determinant which can be rewritten as 
\[
  \Psi(x_1,x_2,\dots,x_N,\alpha,\beta,\dots,\nu) = \frac{1}{\sqrt{N!}}\sum_{P} (-)^PP\psi_{\alpha}(x_1)
    \psi_{\beta}(x_2)\dots\psi_{\nu}(x_N)=\sqrt{N!}\hat{A}\Phi_H,
\]
where we have introduced the anti-symmetrization operator $\hat{A}$ defined by the 
summation over all possible permutations \emph{p} of two fermions.
It is defined as
\[
  \hat{A} = \frac{1}{N!}\sum_{p} (-)^p\hat{P},
\]
with the the Hartree-function given by the simple product of all possible single-particle function
\[
  \Phi_H(x_1,x_2,\dots,x_N,\alpha,\beta,\dots,\nu) =
  \psi_{\alpha}(x_1)
    \psi_{\beta}(x_2)\dots\psi_{\nu}(x_N).
\]

Both $\hat{H_0}$ and $\hat{H_I}$ are invariant under 
permutations of fermions, and hence commute with $\hat{A}$
\[
  [H_0,\hat{A}] = [H_I,\hat{A}] = 0.
\]
Furthermore, as discussed in the introductory notes, $\hat{A}$ satisfies
\[
  \hat{A}^2 = \hat{A},
\]
since every permutation of the Slater determinant reproduces it.

Our functional is written (recall that we have specialized to the case of atoms) as 
\[
  E[\Phi] = \sum_{\mu=1}^N \int \psi_{\mu}^*(x_i)\hat{h}_0(x_i)\psi_{\mu}(x_i) dx_i 
  + \frac{1}{2}\sum_{\mu=1}^N\sum_{\nu=1}^N
   \left[ \int \psi_{\mu}^*(x_i)\psi_{\nu}^*(x_j)\frac{1} 
    {r_{ij}}\psi_{\mu}(x_i)\psi_{\nu}(x_j)dx_idx_j- \int \psi_{\mu}^*(x_i)\psi_{\nu}^*(x_j)
  \frac{1}{r_{ij}}\psi_{\nu}(x_i)\psi_{\mu}(x_j)dx_idx_j\right]
\]
The more compact version, as discussed in our introductory notes, reads
\[
  E[\Phi] 
  = \sum_{\mu}^N \langle \mu | \hat{h}_0 | \mu\rangle+ \frac{1}{2}\sum_{\mu\nu}^N\left[\langle \mu\nu |\frac{1}{r_{ij}}|\mu\nu\rangle-\langle \nu\mu |\frac{1}{r_{ij}}|\mu\nu\rangle\right].
\]
Recalling our discussions earlier, we introduced the following shorthands for the integrals, replacing here
the two-body Coulomb interaction with a more general two-body interaction $V$,
\[
\langle \mu\nu|V|\mu\nu\rangle =  \int \psi_{\mu}^*(\mathbf{r}_i)\psi_{\nu}^*(\mathbf{r}_j)V(r_{ij})\psi_{\mu}(\mathbf{r}_i)\psi_{\nu}(\mathbf{r}_j)
    d\mathbf{r}_i\mathbf{r}_j,
\]
and
\[
\langle \mu\nu|V|\nu\mu\rangle = \int \psi_{\mu}^*(\mathbf{r}_i)\psi_{\nu}^*(\mathbf{r}_j)
  V(r_{ij})\psi_{\nu}(\mathbf{r}_i)\psi_{\mu}(\mathbf{r}_i)
  d\mathbf{r}_i\mathbf{r}_j.  
\]
Since the interaction is invariant under the interchange of two particles it means for example that we have
\[
\langle \mu\nu|V|\mu\nu\rangle =  \langle \nu\mu|V|\nu\mu\rangle,  
\]
or in the more general case
\[
\langle \mu\nu|V|\sigma\tau\rangle =  \langle \nu\mu|V|\tau\sigma\rangle.  
\]
The direct and exchange matrix elements can be  brought together if we define the antisymmetrized matrix element
\[
\langle \mu\nu|V|\mu\nu\rangle_{AS}= \langle \mu\nu|V|\mu\nu\rangle-\langle \mu\nu|V|\nu\mu\rangle,
\]
or for a general matrix element  
\[
\langle \mu\nu|V|\sigma\tau\rangle_{AS}= \langle \mu\nu|V|\sigma\tau\rangle-\langle \mu\nu|V|\tau\sigma\rangle.
\]
It has the symmetry property
\[
\langle \mu\nu|V|\sigma\tau\rangle_{AS}= -\langle \mu\nu|V|\tau\sigma\rangle_{AS}=-\langle \nu\mu|V|\sigma\tau\rangle_{AS}.
\]
The antisymmetric matrix element is also hermitian, implying 
\[
\langle \mu\nu|V|\sigma\tau\rangle_{AS}= \langle \sigma\tau|V|\mu\nu\rangle_{AS}.
\]
With these notations we rewrite the Hartree-Fock functional as
\begin{equation}
  \int \Phi^*\hat{H_I}\Phi d\mathbf{\tau} 
  = \frac{1}{2}\sum_{\mu=1}^A\sum_{\nu=1}^A \langle \mu\nu|V|\mu\nu\rangle_{AS}. \label{H2Expectation2}
\end{equation}

Adding the contribution from the one-body operator $\hat{H}_0$ to
(\ref{H2Expectation2}) we obtain the energy functional 
\begin{equation}
  E[\Phi] 
  = \sum_{\mu=1}^N \langle \mu | h | \mu \rangle +
  \frac{1}{2}\sum_{{\mu}=1}^N\sum_{{\nu}=1}^N \langle \mu\nu|V|\mu\nu\rangle_{AS}. \label{FunctionalEPhi}
\end{equation}
In our coordinate space derivations below we will spell out the Hartree-Fock equations in terms of their integrals.
\end{block}


% !split
\subsection{Derivation of Hartree-Fock equations in coordinate space}
\begin{block}{}
If we generalize the Euler-Lagrange equations to more variables 
and introduce $N^2$ Lagrange multipliers which we denote by 
$\epsilon_{\mu\nu}$, we can write the variational equation for the functional of $E$
\[
  \delta E - \sum_{\mu\nu}^N \epsilon_{\mu\nu} \delta
  \int \psi_{\mu}^* \psi_{\nu} = 0.
\]
For the orthogonal wave functions $\psi_{i}$ this reduces to
\[
  \delta E - \sum_{\mu=1}^N \epsilon_{\mu} \delta
  \int \psi_{\mu}^* \psi_{\mu} = 0.
\]

Variation with respect to the single-particle wave functions $\psi_{\mu}$ yields then
\[
  \sum_{\mu=1}^N \int \delta\psi_{\mu}^*\hat{h_0}(x_i)\psi_{\mu}
  dx_i  
  + \frac{1}{2}\sum_{{\mu}=1}^N\sum_{{\nu}=1}^N \left[ \int
  \delta\psi_{\mu}^*\psi_{\nu}^*\frac{1} 
  {r_{ij}}\psi_{\mu}\psi_{\nu} dx_idx_j- \int
  \delta\psi_{\mu}^*\psi_{\nu}^*\frac{1}{r_{ij}}\psi_{\nu}\psi_{\mu}
  dx_idx_j \right]+ 
\]
\[
\sum_{\mu=1}^N \int \psi_{\mu}^*\hat{h_0}(x_i)\delta\psi_{\mu}
  dx_i 
  + \frac{1}{2}\sum_{{\mu}=1}^N\sum_{{\nu}=1}^N \left[ \int
  \psi_{\mu}^*\psi_{\nu}^*\frac{1} 
  {r_{ij}}\delta\psi_{\mu}\psi_{\nu} dx_idx_j- \int
  \psi_{\mu}^*\psi_{\nu}^*\frac{1}{r_{ij}}\psi_{\nu}\delta\psi_{\mu}
  dx_idx_j \right]-  \sum_{{\mu}=1}^N E_{\mu} \int \delta\psi_{\mu}^*
  \psi_{\mu}dx_i
  -  \sum_{{\mu}=1}^N E_{\mu} \int \psi_{\mu}^*
  \delta\psi_{\mu}dx_i = 0.
\]

Although the variations $\delta\psi$ and $\delta\psi^*$ are not
independent, they may in fact be treated as such, so that the 
terms dependent on either $\delta\psi$ and $\delta\psi^*$ individually 
may be set equal to zero. To see this, simply 
replace the arbitrary variation $\delta\psi$ by $i\delta\psi$, so that
$\delta\psi^*$ is replaced by $-i\delta\psi^*$, and combine the two
equations. We thus arrive at the Hartree-Fock equations
\begin{equation}
\left[ -\frac{1}{2}\nabla_i^2-\frac{Z}{r_i} + \sum_{\nu=1}^N\int \psi_{\nu}^*(x_j)\frac{1}{r_{ij}}\psi_{\nu}(x_j)dx_j \right]\psi_{\mu}(x_i) - \left[ \sum_{{\nu}=1}^N \int\psi_{\nu}^*(x_j) \frac{1}{r_{ij}}\psi_{\mu}(x_j) dx_j\right] \psi_{\nu}(x_i) = \epsilon_{\mu} \psi_{\mu}(x_i).  \label{eq:hartreefockcoordinatespace}
\end{equation}
Notice that the integration $\int dx_j$ implies an
integration over the spatial coordinates $\mathbf{r_j}$ and a summation
over the spin-coordinate of fermion $j$. We note that the factor of $1/2$ in front of the sum involving the two-body interaction, has been removed. This is due to the fact that we need to vary both $\delta\psi_{\mu}^*$ and
$\delta\psi_{\nu}^*$. Using the symmetry properties of the two-body interaction and interchanging $\mu$ and $\nu$
as summation indices, we obtain two identical terms. 

The two first terms in the last equation are the one-body kinetic energy and the
electron-nucleus potential. The third or \emph{direct} term is the averaged electronic repulsion of the other
electrons. As written, the
term includes the \emph{self-interaction} of 
electrons when $\mu=\nu$. The self-interaction is cancelled in the fourth
term, or the \emph{exchange} term. The exchange term results from our
inclusion of the Pauli principle and the assumed determinantal form of
the wave-function. Equation (\ref{eq:hartreefockcoordinatespace}), in addition to the kinetic energy and the attraction from the atomic nucleus that confines the motion of a single electron,   represents now the motion of a single-particle modified by the two-body interaction. The additional contribution to the Schroedinger equation due to the two-body interaction, represents a mean field set up by all the other bystanding electrons, the latter given by the sum over all single-particle states occupied by $N$ electrons. 

The Hartree-Fock equation is an example of an integro-differential equation. These equations involve repeated calculations of integrals, in addition to the solution of a set of coupled differential equations. 
The Hartree-Fock equations can also be rewritten in terms of an eigenvalue problem. The solution of an eigenvalue problem represents often a more practical algorithm and the  solution of  coupled  integro-differential equations.
This alternative derivation of the Hartree-Fock equations is given below.
\end{block}


% !split
\subsection{Analysis of Hartree-Fock equations in coordinate space}
\begin{block}{}
  A theoretically convenient form of the
Hartree-Fock equation is to regard the direct and exchange operator
defined through 
\begin{equation*}
  V_{\mu}^{d}(x_i) = \int \psi_{\mu}^*(x_j) 
  \frac{1}{r_{ij}}\psi_{\mu}(x_j) dx_j
\end{equation*}
and
\begin{equation*}
  V_{\mu}^{ex}(x_i) g(x_i) 
  = \left(\int \psi_{\mu}^*(x_j) 
  \frac{1}{r_{ij}}g(x_j) dx_j
  \right)\psi_{\mu}(x_i),
\end{equation*}
respectively. 

The function $g(x_i)$ is an arbitrary function,
and by the substitution $g(x_i) = \psi_{\nu}(x_i)$
we get
\begin{equation*}
  V_{\mu}^{ex}(x_i) \psi_{\nu}(x_i) 
  = \left(\int \psi_{\mu}^*(x_j) 
  \frac{1}{r_{ij}}\psi_{\nu}(x_j)
  dx_j\right)\psi_{\mu}(x_i).
\end{equation*}
We may then rewrite the Hartree-Fock equations as
\[
  \hat{h}^{HF}(x_i) \psi_{\nu}(x_i) = \epsilon_{\nu}\psi_{\nu}(x_i),
\]
with
\[
  \hat{h}^{HF}(x_i)= \hat{h}_0(x_i) + \sum_{\mu=1}^NV_{\mu}^{d}(x_i) -
  \sum_{\mu=1}^NV_{\mu}^{ex}(x_i),
\]
and where $\hat{h}_0(i)$ is the one-body part. The latter is normally chosen as a part which yields solutions in closed form. The harmonic oscilltor is a classical problem thereof.
We normally rewrite the last equation as
\[
  \hat{h}^{HF}(x_i)= \hat{h}_0(x_i) + \hat{u}^{HF}(x_i). 
\]
\end{block}


% !split
\subsection{Hartree-Fock by varying the coefficients of a wave function expansion}
\begin{block}{}
Another possibility is to expand the single-particle functions in a known basis  and vary the coefficients, 
that is, the new single-particle wave function is written as a linear expansion
in terms of a fixed chosen orthogonal basis (for example the well-known harmonic oscillator functions or the hydrogen-like functions etc).
We define our new Hartree-Fock single-particle basis by performing a unitary transformation 
on our previous basis (labelled with greek indices) as
\begin{equation}
\psi_p^{HF}  = \sum_{\lambda} C_{p\lambda}\phi_{\lambda}. \label{eq:newbasis}
\end{equation}
In this case we vary the coefficients $C_{p\lambda}$. If the basis has infinitely many solutions, we need
to truncate the above sum.  We assume that the basis $\phi_{\lambda}$ is orthogonal. A unitary transformation keeps the orthogonality, as discussed in exercise 1 below.  

It is normal to choose a single-particle basis defined as the eigenfunctions
of parts of the full Hamiltonian. The typical situation consists of the solutions of the one-body part of the Hamiltonian, that is we have
\[
\hat{h}_0\phi_{\lambda}=\epsilon_{\lambda}\phi_{\lambda}.
\]
The single-particle wave functions $\phi_{\lambda}({\bf r})$, defined by the quantum numbers $\lambda$ and ${\bf r}$
are defined as the overlap 
\[
   \phi_{\lambda}({\bf r})  = \langle {\bf r} | \lambda \rangle .
\]

In our discussions hereafter we will use our definitions of single-particle states above and below the Fermi ($F$) level given by the labels
$ijkl\dots \le F$ for so-called single-hole states and $abcd\dots > F$ for so-called particle states.
For general single-particle states we employ the labels $pqrs\dots$. 
\end{block}


% !split
\subsection{Hartree-Fock by varying the coefficients of a wave function expansion}
\begin{block}{}
In Eq.~(\ref{FunctionalEPhi}), restated here
\[
  E[\Phi] 
  = \sum_{\mu=1}^N \langle \mu | h | \mu \rangle +
  \frac{1}{2}\sum_{{\mu}=1}^N\sum_{{\nu}=1}^N \langle \mu\nu|V|\mu\nu\rangle_{AS},
\]
we found the expression for the energy functional in terms of the basis function $\phi_{\lambda}({\bf r})$. We then  varied the above energy functional with respect to the basis functions $|\mu \rangle$. 
Now we are interested in defining a new basis defined in terms of
a chosen basis as defined in Eq.~(\ref{eq:newbasis}). We can then rewrite the energy functional as
\begin{equation}
  E[\Phi^{HF}] 
  = \sum_{i=1}^N \langle i | h | i \rangle +
  \frac{1}{2}\sum_{ij=1}^N\langle ij|V|ij\rangle_{AS}, \label{FunctionalEPhi2}
\end{equation}
where $\Phi^{HF}$ is the new Slater determinant defined by the new basis of Eq.~(\ref{eq:newbasis}). 

Using Eq.~(\ref{eq:newbasis}) we can rewrite Eq.~(\ref{FunctionalEPhi2}) as 
\begin{equation}
  E[\Psi] 
  = \sum_{i=1}^N \sum_{\alpha\beta} C^*_{i\alpha}C_{i\beta}\langle \alpha | h | \beta \rangle +
  \frac{1}{2}\sum_{ij=1}^N\sum_{{\alpha\beta\gamma\delta}} C^*_{i\alpha}C^*_{j\beta}C_{i\gamma}C_{j\delta}\langle \alpha\beta|V|\gamma\delta\rangle_{AS}. \label{FunctionalEPhi3}
\end{equation}

We wish now to minimize the above functional. We introduce again a set of Lagrange multipliers, noting that
since $\langle i | j \rangle = \delta_{i,j}$ and $\langle \alpha | \beta \rangle = \delta_{\alpha,\beta}$, 
the coefficients $C_{i\gamma}$ obey the relation
\[
 \langle i | j \rangle=\delta_{i,j}=\sum_{\alpha\beta} C^*_{i\alpha}C_{i\beta}\langle \alpha | \beta \rangle=
\sum_{\alpha} C^*_{i\alpha}C_{i\alpha},
\]
which allows us to define a functional to be minimized that reads
\begin{equation}
  F[\Phi^{HF}]=E[\Phi^{HF}] - \sum_{i=1}^N\epsilon_i\sum_{\alpha} C^*_{i\alpha}C_{i\alpha}.
\end{equation}
\end{block}



% !split
\subsection{Hartree-Fock by varying the coefficients of a wave function expansion}
\begin{block}{}
Minimizing with respect to $C^*_{i\alpha}$, remembering that the equations for $C^*_{i\alpha}$ and $C_{i\alpha}$
can be written as two  independent equations, we obtain
\[
\frac{d}{dC^*_{i\alpha}}\left[  E[\Phi^{HF}] - \sum_{j}\epsilon_j\sum_{\alpha} C^*_{j\alpha}C_{j\alpha}\right]=0,
\]
which yields for every single-particle state $i$ and index $\alpha$ (recalling that the coefficients $C_{i\alpha}$ are matrix elements of a unitary (or orthogonal for a real symmetric matrix) matrix)
the following Hartree-Fock equations
\[
\sum_{\beta} C_{i\beta}\langle \alpha | h | \beta \rangle+
\sum_{j=1}^N\sum_{\beta\gamma\delta} C^*_{j\beta}C_{j\delta}C_{i\gamma}\langle \alpha\beta|V|\gamma\delta\rangle_{AS}=\epsilon_i^{HF}C_{i\alpha}.
\]
We can rewrite this equation as (changing dummy variables)
\[
\sum_{\beta} \left\{\langle \alpha | h | \beta \rangle+
\sum_{j}^N\sum_{\gamma\delta} C^*_{j\gamma}C_{j\delta}\langle \alpha\gamma|V|\beta\delta\rangle_{AS}\right\}C_{i\beta}=\epsilon_i^{HF}C_{i\alpha}.
\]
Note that the sums over greek indices run over the number of basis set functions (in principle an infinite number).
Defining 
\[
h_{\alpha\beta}^{HF}=\langle \alpha | h | \beta \rangle+
\sum_{j=1}^N\sum_{\gamma\delta} C^*_{j\gamma}C_{j\delta}\langle \alpha\gamma|V|\beta\delta\rangle_{AS},
\]
we can rewrite the new equations as 
\begin{equation}
\sum_{\gamma}h_{\alpha\beta}^{HF}C_{i\beta}=\epsilon_i^{HF}C_{i\alpha}. \label{eq:newhf}
\end{equation}
The latter is nothing but a standard eigenvalue problem. Compared with Eq.~(\ref{eq:hartreefockcoordinatespace}),
we see that we do not need to compute any integrals in an iterative procedure for solving the equations.
It suffices to tabulate the matrix elements $\langle \alpha | h | \beta \rangle$ and $\langle \alpha\gamma|V|\beta\delta\rangle_{AS}$ once and for all. Successive iterations require thus only a look-up in tables over one-body and two-body matrix elements. These details will be discussed below when we solve the Hartree-Fock equations numerical. 
\end{block}

% !split
\subsection{Hartree-Fock algorithm}
\begin{block}{}
Our Hartree-Fock matrix  is thus
\[
\hat{h}_{\alpha\beta}^{HF}=\langle \alpha | \hat{h}_0 | \beta \rangle+
\sum_{j=1}^N\sum_{\gamma\delta} C^*_{j\gamma}C_{j\delta}\langle \alpha\gamma|V|\beta\delta\rangle_{AS}.
\]
The Hartree-Fock equations are solved in an iterative waym starting with a guess for the coefficients $C_{j\gamma}=\delta_{j,\gamma}$ and solving the equations by diagonalization till the new single-particle energies
$\epsilon_i^{\mathrm{HF}}$ do not change anymore by a prefixed quantity. 

Normally we assume that the single-particle basis $|\beta\rangle$ forms an eigenbasis for the operator
$\hat{h}_0$, meaning that the Hartree-Fock matrix becomes  
\[
\hat{h}_{\alpha\beta}^{HF}=\epsilon_{\alpha}\delta_{\alpha,\beta}+
\sum_{j=1}^N\sum_{\gamma\delta} C^*_{j\gamma}C_{j\delta}\langle \alpha\gamma|V|\beta\delta\rangle_{AS}.
\]
The Hartree-Fock eigenvalue problem
\[
\sum_{\beta}\hat{h}_{\alpha\beta}^{HF}C_{i\beta}=\epsilon_i^{\mathrm{HF}}C_{i\alpha},
\]
can be written out in a more compact form as
\[
\hat{h}^{HF}\hat{C}=\epsilon^{\mathrm{HF}}\hat{C}. 
\]
The Hartree-Fock equations are, in their simplest form, solved in an iterative way, starting with a guess for the
coefficients $C_{i\alpha}$. We label the coefficients as $C_{i\alpha}^{(n)}$, where the subscript $n$ stands for iteration $n$.
To set up the algorithm we can proceed as follows:

\begin{itemize}
 \item We start with a guess $C_{i\alpha}^{(0)}=\delta_{i,\alpha}$. Alternatively, we could have used random starting values as long as the vectors are normalized. Another possibility is to give states below the Fermi level a larger weight.

 \item The Hartree-Fock matrix simplifies then to (assuming that the coefficients $C_{i\alpha} $  are real)
\end{itemize}

\noindent
\[
\hat{h}_{\alpha\beta}^{HF}=\epsilon_{\alpha}\delta_{\alpha,\beta}+
\sum_{j = 1}^N\sum_{\gamma\delta} C_{j\gamma}^{(0)}C_{j\delta}^{(0)}\langle \alpha\gamma|V|\beta\delta\rangle_{AS}.
\]
Solving the Hartree-Fock eigenvalue problem yields then new eigenvectors $C_{i\alpha}^{(1)}$ and eigenvalues
$\epsilon_i^{HF(1)}$. 
\begin{itemize}
 \item With the new eigenvalues we can set up a new Hartree-Fock potential 
\end{itemize}

\noindent
\[
\sum_{j = 1}^N\sum_{\gamma\delta} C_{j\gamma}^{(1)}C_{j\delta}^{(1)}\langle \alpha\gamma|V|\beta\delta\rangle_{AS}.
\]
The diagonalization with the new Hartree-Fock potential yields new eigenvectors and eigenvalues.
This process is continued till for example
\[
\frac{\sum_{p} |\epsilon_i^{(n)}-\epsilon_i^{(n-1)}|}{m} \le \lambda,  
\]
where $\lambda$ is a user prefixed quantity ($\lambda \sim 10^{-8}$ or smaller) and $p$ runs over all calculated single-particle
energies and $m$ is the number of single-particle states.
\end{block}

% !split
\subsection{Analysis of Hartree-Fock equations and Koopman's theorem}
\begin{block}{}
We can rewrite the ground state energy by adding and subtracting $\hat{u}^{HF}(x_i)$ 
\[
  E_0^{HF} =\langle \Phi_0 | \hat{H} | \Phi_0\rangle = 
\sum_{i\le F}^N \langle i | \hat{h}_0 +\hat{u}^{HF}| j\rangle+ \frac{1}{2}\sum_{i\le F}^N\sum_{j \le F}^N\left[\langle ij |\hat{v}|ij \rangle-\langle ij|\hat{v}|ji\rangle\right]-\sum_{i\le F}^N \langle i |\hat{u}^{HF}| i\rangle,
\]
which results in
\[
  E_0^{HF}
  = \sum_{i\le F}^N \varepsilon_i^{HF} + \frac{1}{2}\sum_{i\le F}^N\sum_{j \le F}^N\left[\langle ij |\hat{v}|ij \rangle-\langle ij|\hat{v}|ji\rangle\right]-\sum_{i\le F}^N \langle i |\hat{u}^{HF}| i\rangle.
\]
Our single-particle states $ijk\dots$ are now single-particle states obtained from the solution of the Hartree-Fock equations.
Using our definition of the Hartree-Fock single-particle energies we obtain then the following expression for the total ground-state energy
\[
  E_0^{HF}
  = \sum_{i\le F}^N \varepsilon_i - \frac{1}{2}\sum_{i\le F}^N\sum_{j \le F}^N\left[\langle ij |\hat{v}|ij \rangle-\langle ij|\hat{v}|ji\rangle\right].
\]
This form will be used in our discussion of Koopman's theorem.
\end{block}

% !split
\subsection{Analysis of Hartree-Fock equations and Koopman's theorem}
\begin{block}{Atomic physics case }
We have defined 
\[
  E[\Phi^{\mathrm{HF}}(N)] 
  = \sum_{i=1}^N \langle i | \hat{h}_0 | i \rangle +
  \frac{1}{2}\sum_{ij=1}^N\langle ij|V|ij\rangle_{AS},
\]
where $\Phi^{\mathrm{HF}}(N)$ is the new Slater determinant defined by the new basis of Eq.~(\ref{eq:newbasis})
for $N$ electrons (same $Z$).  If we assume that the single-particle wave functions in the new basis do not change 
when we remove one electron or add one electron, we can then define the corresponding energy for the $N-1$ systems as 
\[
  E[\Phi^{\mathrm{HF}}(N-1)] 
  = \sum_{i=1; i\ne k}^N \langle i | \hat{h}_0 | i \rangle +
  \frac{1}{2}\sum_{ij=1;i,j\ne k}^N\langle ij|V|ij\rangle_{AS},
\]
where we have removed a single-particle state $k\le F$, that is a state below the Fermi level.  

Calculating the difference 
\[
  E[\Phi^{\mathrm{HF}}(N)]-   E[\Phi^{\mathrm{HF}}(N-1)] = \langle k | \hat{h}_0 | k \rangle +
  \frac{1}{2}\sum_{i=1;i\ne k}^N\langle ik|V|ik\rangle_{AS}  \frac{1}{2}\sum_{j=1;j\ne k}^N\langle kj|V|kj\rangle_{AS},
\]
we obtain
\[
  E[\Phi^{\mathrm{HF}}(N)]-   E[\Phi^{\mathrm{HF}}(N-1)] = \langle k | \hat{h}_0 | k \rangle +
  \frac{1}{2}\sum_{j=1}^N\langle kj|V|kj\rangle_{AS}
\]
which is just our definition of the Hartree-Fock single-particle energy
\[
  E[\Phi^{\mathrm{HF}}(N)]-   E[\Phi^{\mathrm{HF}}(N-1)] = \epsilon_k^{\mathrm{HF}} 
\]
Similarly, we can now compute the difference (we label the single-particle states above the Fermi level as $abcd > F$)
\[
  E[\Phi^{\mathrm{HF}}(N+1)]-   E[\Phi^{\mathrm{HF}}(N)]= \epsilon_a^{\mathrm{HF}}. 
\]
These two equations can thus be used to the electron affinity or ionization energies, respectively. 
Koopman's theorem states that for example the ionization energy of a closed-shell system is given by the energy of the highest occupied single-particle state.  If we assume that changing the number of electrons from $N$ to $N+1$ does not change the Hartree-Fock single-particle energies and eigenfunctions, then Koopman's theorem simply states that the ionization energy of an atom is given by the single-particle energy of the last bound state. In a similar way, we can also define the electron affinities. 

As an example, consider a simple model for atomic sodium, Na. Neutral sodium has eleven electrons, 
with the weakest bound one being confined the $3s$ single-particle quantum numbers. The energy needed to remove an electron from neutral sodium is rather small, 5.1391 eV, a feature which pertains to all alkali metals.
Having performed a  Hartree-Fock calculation for neutral sodium would then allows us to compute the
ionization energy by using the single-particle energy for the $3s$ states, namely $\epsilon_{3s}^{\mathrm{HF}}$. 

From these considerations, we see that Hartree-Fock theory allows us to make a connection between experimental 
observables (here ionization and affinity energies) and the underlying interactions between particles.  
In this sense, we are now linking the dynamics and structure of a many-body system with the laws of motion which govern the system. Our approach is a reductionistic one, meaning that we want to understand the laws of motion 
in terms of the particles or degrees of freedom which we believe are the fundamental ones. Our Slater determinant, being constructed as the product of various single-particle functions, follows this philosophy. 
\end{block}

% !split
\subsection{Analysis of Hartree-Fock equations in coordinate space, Koopman's theorem}
\begin{block}{Nuclear physics case }
With similar arguments as in atomic physics, we can now use Hartree-Fock theory to make a link
between nuclear forces and separation energies. Changing to nuclear system, we define
\[
  E[\Phi^{\mathrm{HF}}(A)] 
  = \sum_{i=1}^A \langle i | \hat{h}_0 | i \rangle +
  \frac{1}{2}\sum_{ij=1}^A\langle ij|V|ij\rangle_{AS},
\]
where $\Phi^{\mathrm{HF}}(A)$ is the new Slater determinant defined by the new basis of Eq.~(\ref{eq:newbasis})
for $A$ nucleons, where $A=N+Z$, with $N$ now being the number of neutrons and $Z$ th enumber of protons.  If we assume again that the single-particle wave functions in the new basis do not change from a nucleus with $A$ nucleons to a nucleus with $A-1$  nucleons, we can then define the corresponding energy for the $A-1$ systems as 
\[
  E[\Phi^{\mathrm{HF}}(A-1)] 
  = \sum_{i=1; i\ne k}^A \langle i | \hat{h}_0 | i \rangle +
  \frac{1}{2}\sum_{ij=1;i,j\ne k}^A\langle ij|V|ij\rangle_{AS},
\]
where we have removed a single-particle state $k\le F$, that is a state below the Fermi level.  

Calculating the difference 
\[
  E[\Phi^{\mathrm{HF}}(A)]-   E[\Phi^{\mathrm{HF}}(A-1)] 
  = \langle k | \hat{h}_0 | k \rangle +
  \frac{1}{2}\sum_{i=1;i\ne k}^A\langle ik|V|ik\rangle_{AS}  \frac{1}{2}\sum_{j=1;j\ne k}^A\langle kj|V|kj\rangle_{AS},
\]
which becomes 
\[
  E[\Phi^{\mathrm{HF}}(A)]-   E[\Phi^{\mathrm{HF}}(A-1)] 
  = \langle k | \hat{h}_0 | k \rangle +
  \frac{1}{2}\sum_{j=1}^A\langle kj|V|kj\rangle_{AS}
\]
which is just our definition of the Hartree-Fock single-particle energy
\[
  E[\Phi^{\mathrm{HF}}(A)]-   E[\Phi^{\mathrm{HF}}(A-1)] 
  = \epsilon_k^{\mathrm{HF}} 
\]
Similarly, we can now compute the difference (recall that the single-particle states $abcd > F$)
\[
  E[\Phi^{\mathrm{HF}}(A+1)]-   E[\Phi^{\mathrm{HF}}(A)]= \epsilon_a^{\mathrm{HF}}. 
\]
If we then recall that the binding energy differences 
\[
BE(A)-BE(A-1) \hspace{0.5cm} \mathrm{and} \hspace{0.5cm} BE(A+1)-BE(A), 
\]
define the separation energies, we see that the Hartree-Fock single-particle energies can be used to
define separation energies. We have thus our first link between nuclear forces (included in the potential energy term) and an observable quantity defined by differences in binding energies. 

We have thus the following interpretations (if the single-particle field do not change)
\[
BE(A)-BE(A-1)\approx  E[\Phi^{\mathrm{HF}}(A)]-   E[\Phi^{\mathrm{HF}}(A-1)] 
  = \epsilon_k^{\mathrm{HF}}, 
\]
and
\[
BE(A+1)-BE(A)\approx  E[\Phi^{\mathrm{HF}}(A+1)]-   E[\Phi^{\mathrm{HF}}(A)] =  \epsilon_a^{\mathrm{HF}}. 
\]
If  we use oxygen-16 as our closed-shell nucleus, we could then interpret the separation energy
\[
BE(^{16}\mathrm{O})-BE(^{15}\mathrm{O})\approx \epsilon_{0p^{\nu}_{1/2}}^{\mathrm{HF}}, 
\]
and
\[
BE(^{16}\mathrm{O})-BE(^{15}\mathrm{N})\approx \epsilon_{0p^{\pi}_{1/2}}^{\mathrm{HF}}.
\]
Similalry, we could interpret
\[
BE(^{17}\mathrm{O})-BE(^{16}\mathrm{O})\approx \epsilon_{0d^{\nu}_{5/2}}^{\mathrm{HF}}, 
\]
and 
\[
BE(^{17}\mathrm{F})-BE(^{16}\mathrm{O})\approx\epsilon_{0d^{\pi}_{5/2}}^{\mathrm{HF}}.
\]
We can continue like this for all $A\pm 1$ nuclei where $A$ is a good closed-shell (or subshell closure)
nucleus. Examples are $^{4}$He, $^{22}$O, $^{24}$O, $^{40}$Ca, $^{48}$Ca, $^{52}$Ca, $^{54}$Ca, $^{56}$Ni, 
$^{68}$Ni, $^{78}$Ni, $^{90}$Zr, $^{88}$Sr, $^{100}$Sn, $^{132}$Sn and $^{208}$Pb, to mention some possile cases.

We can thus make our first interpretation of the separation energies in terms of the simplest
possible many-body theory. 
If we also recall that the so-called energy gap for neutrons (or protons) is defined as
\[
\Delta S_n= 2BE(N,Z)-BE(N-1,Z)-BE(N+1,Z),
\]
for neutrons and the corresponding gap for protons
\[
\Delta S_p= 2BE(N,Z)-BE(N,Z-1)-BE(N,Z+1),
\]
we can define the neutron and proton energy gaps for $^{16}$O as
\[
\Delta S_{\nu}=\epsilon_{0d^{\nu}_{5/2}}^{\mathrm{HF}}-\epsilon_{0p^{\nu}_{1/2}}^{\mathrm{HF}}, 
\]
and 
\[
\Delta S_{\pi}=\epsilon_{0d^{\pi}_{5/2}}^{\mathrm{HF}}-\epsilon_{0p^{\pi}_{1/2}}^{\mathrm{HF}}. 
\]
\end{block}




% !split
\subsection{Hartree-Fock in second quantization and stability of HF solution}
\begin{block}{}
We wish now to derive the Hartree-Fock equations using our second-quantized formalism and study the stability of the equations. 
Our ansatz for the ground state of the system is approximated as (this is our representation of a Slater determinant in second quantization)
\[   
|\Phi_0\rangle = |c\rangle = a^{\dagger}_i a^{\dagger}_j \dots a^{\dagger}_l|0\rangle.
\]
We wish to determine $\hat{u}^{HF}$ so that 
$E_0^{HF}= \langle c|\hat{H}| c\rangle$ becomes a local minimum. 

In our analysis here we will need Thouless' theorem, which states that
an arbitrary Slater determinant $|c'\rangle$ which is not orthogonal to a determinant
$| c\rangle ={\displaystyle\prod_{i=1}^{n}}
a_{\alpha_{i}}^{\dagger}|0\rangle$, can be written as
\[
|c'\rangle=exp\left\{\sum_{a>F}\sum_{i\le F}C_{ai}a_{a}^{\dagger}a_{i}\right\}| c\rangle 
\]


Let us give a simple proof of Thouless' theorem. The theorem states that we can make a linear combination av particle-hole excitations  with respect to a given reference state $| c\rangle$. With this linear combination, we can make a new Slater determinant $| c'\rangle $ which is not orthogonal to $| c\rangle , that is
\[
\langle c|c'\rangle \ne 0.
\] 
To show this we need some intermediate steps. The exponential product of two operators  $\exp{\hat{A}}\times\exp{\hat{B}}$ is equal to $\exp{(\hat{A}+\hat{B})}$ only if the two operators commute, that is
\[
[\hat{A},\hat{B}] = 0.
\]
If the operators do not commute, we need to resort to the Baker-Campbell-Hauersdorf, see for example \href{{http://www.encyclopediaofmath.org/index.php/Campbell%E2%80%93Hausdorff_formula}}{\nolinkurl{http://www.encyclopediaofmath.org/index.php/Campbell\%E2\%80\%93Hausdorff_formula}}. This relation states that
\[
\exp{\hat{C}}=\exp{\hat{A}}\exp{\hat{B}},
\]
with 
\[
\hat{C}=\hat{A}+\hat{B}+\frac{1}{2}[\hat{A},\hat{B}]+\frac{1}{12}[[\hat{A},\hat{B}],\hat{B}]-\frac{1}{12}[[\hat{A},\hat{B}],\hat{A}]+\dots
\]
From these relations, we note that 
in our expression  for $|c'\rangle$ we have commutators of the type
\[
[a_{a}^{\dagger}a_{i},a_{b}^{\dagger}a_{j}],
\]
and it is easy to convince oneself that these commutators, or higher powers thereof, are all zero. This means that we can write out our new representation of a Slater determinant as
\[
|c'\rangle=exp\left\{\sum_{a>F}\sum_{i\le F}C_{ai}a_{a}^{\dagger}a_{i}\right\}| c\rangle=\prod_{i}\left\{1+\sum_{a>F}C_{ai}a_{a}^{\dagger}a_{i}+\left(\sum_{a>F}C_{ai}a_{a}^{\dagger}a_{i}\right)^2+\dots\right\}| c\rangle
\]
We note that
\[
\prod_{i}\sum_{a>F}C_{ai}a_{a}^{\dagger}a_{i}\sum_{b>F}C_{bi}a_{b}^{\dagger}a_{i}| c\rangle =0,
\]
and all higher-order powers of these combinations of creation and annihilation operators disappear 
due to the fact that $(a_i)^n| c\rangle =0$ when $n > 1$. This allows us to rewrite the expression for $|c'\rangle $ as
\[
|c'\rangle=\prod_{i}\left\{1+\sum_{a>F}C_{ai}a_{a}^{\dagger}a_{i}\right\}| c\rangle,
\]
which we can rewrite as 
\[
|c'\rangle=\prod_{i}\left\{1+\sum_{a>F}C_{ai}a_{a}^{\dagger}a_{i}\right\}| a^{\dagger}_{i_1} a^{\dagger}_{i_2} \dots a^{\dagger}_{i_n}|0\rangle.
\]
The last equation can be written as
\[
|c'\rangle=\prod_{i}\left\{1+\sum_{a>F}C_{ai}a_{a}^{\dagger}a_{i}\right\}| a^{\dagger}_{i_1} a^{\dagger}_{i_2} \dots a^{\dagger}_{i_n}|0\rangle=\left(1+\sum_{a>F}C_{ai_1}a_{a}^{\dagger}a_{i_1}\right)a^{\dagger}_{i_1}\left(1+\sum_{a>F}C_{ai_2}a_{a}^{\dagger}a_{i_2}\right)a^{\dagger}_{i_2} \dots |0\rangle=\prod_{i}\left(a^{\dagger}_{i}+\sum_{a>F}C_{ai}a_{a}^{\dagger}\right)|0\rangle.
\]
If we define a new creation operator 
\begin{equation}
b^{\dagger}_{i}=a^{\dagger}_{i}+\sum_{a>F}C_{ai}a_{a}^{\dagger}, \label{eq:newb}
\end{equation}
we have 
\[
|c'\rangle=\prod_{i}b^{\dagger}_{i}|0\rangle=\prod_{i}\left(a^{\dagger}_{i}+\sum_{a>F}C_{ai}a_{a}^{\dagger}\right)|0\rangle,
\]
meaning that the new representation of the Slater determinant in second quantization, $|c'\rangle$, looks like our previous ones. However, this representation is not general enough since we have a restriction on the sum over single-particle states in Eq.~(\ref{eq:newb}). The single-particle states have all to be above the Fermi level.
The question then is whether we can construct a general representation of a Slater determinant with a creation operator 
\[
\tilde{b}^{\dagger}_{i}=\sum_{p}f_{ip}a_{p}^{\dagger},
\]
where $f_{ip}$ is a matrix element of a unitary matrix which transforms our creation and annihilation operators
$a^{\dagger}$ and $a$ to $\tilde{b}^{\dagger}$ and $\tilde{b}$. These new operators define a new representation of a Slater determinant as
\[
|\tilde{c}\rangle=\prod_{i}\tilde{b}^{\dagger}_{i}|0\rangle.
\]
We need to show that $|\tilde{c}\rangle= |c'\rangle$. We need also to assume that the new state
is not orthogonal to $|c\rangle$, that is $\langle c| \tilde{c}\rangle \ne 0$. From this it follows that 
\[
\langle c| \tilde{c}\rangle=\langle 0| a_{i_n}\dots a_{i_1}\left(\sum_{p=i_1}^{i_n}f_{i_1p}a_{p}^{\dagger} \right)\left(\sum_{q=i_1}^{i_n}f_{i_2q}a_{q}^{\dagger} \right)\dots \left(\sum_{t=i_1}^{i_n}f_{i_nt}a_{t}^{\dagger} \right)|0\rangle,
\]
which is nothing but the determinant $det(f_{ip})$ which we can, using the intermediate normalization condition, 
normalize to one, that is
\[
det(f_{ip})=1,
\]
meaning that $f$ has an inverse defined as (since we are dealing with orthogonal, and in our case unitary as well, transformations)
\[
\sum_{k} f_{ik}f^{-1}_{kj} = \delta_{ij},
\]
and 
\[
\sum_{j} f^{-1}_{ij}f_{jk} = \delta_{ik}.
\]
Using these relations we can then define the linear combination of creation (and annihilation as well) 
operators as
\[
\sum_{i}f^{-1}_{ki}\tilde{b}^{\dagger}_{i}=\sum_{i}f^{-1}_{ki}\sum_{p=i_1}^{\infty}f_{ip}a_{p}^{\dagger}=a_{k}^{\dagger}+\sum_{i}\sum_{p=i_{n+1}}^{\infty}f^{-1}_{ki}f_{ip}a_{p}^{\dagger}.
\]
Defining 
\[
c_{kp}=\sum_{i \le F}f^{-1}_{ki}f_{ip},
\]
we can redefine 
\[
a_{k}^{\dagger}+\sum_{i}\sum_{p=i_{n+1}}^{\infty}f^{-1}_{ki}f_{ip}a_{p}^{\dagger}=a_{k}^{\dagger}+\sum_{p=i_{n+1}}^{\infty}c_{kp}a_{p}^{\dagger}=b_k^{\dagger},
\]
our starting point. We have shown that our general representation of a Slater determinant 
\[
|\tilde{c}\rangle=\prod_{i}\tilde{b}^{\dagger}_{i}|0\rangle=|c'\rangle=\prod_{i}b^{\dagger}_{i}|0\rangle,
\]
with 
\[
b_k^{\dagger}=a_{k}^{\dagger}+\sum_{p=i_{n+1}}^{\infty}c_{kp}a_{p}^{\dagger}.
\]
This means that we can actually write an ansatz for the ground state of the system as a linear combination of
terms which contain the ansatz itself $|c\rangle$ with  an admixture from an infinity of one-particle-one-hole states. The latter has important consequences when we wish to interpret the Hartree-Fock equations and their stability. We can rewrite the new representation as 
\[
|c'\rangle = |c\rangle+|\delta c\rangle,
\]
where $|\delta c\rangle$ can now be interpreted as a small variation. If we approximate this term with 
contributions from one-particle-one-hole (\emph{1p-1h}) states only, we arrive at 
\[
|c'\rangle = \left(1+\sum_{ai}\delta C_{ai}a_{a}^{\dagger}a_i\right)|c\rangle.
\]
In our derivation of the Hartree-Fock equations we have shown that 
\[
\langle \delta c| \hat{H} | c\rangle =0,
\]
which means that we have to satisfy
\[
\langle c|\sum_{ai}\delta C_{ai}\left\{a_{a}^{\dagger}a_i\right\} \hat{H} | c\rangle =0.
\]
With this as a background, we are now ready to study the stability of the Hartree-Fock equations.


\end{block}

% !split
\subsection{Hartree-Fock in second quantization and stability of HF solution}
\begin{block}{}
The variational condition for deriving the Hartree-Fock equations guarantees only that the expectation value $\langle c | \hat{H} | c \rangle$ has an extreme value, not necessarily a minimum. To figure out whether the extreme value we have found  is a minimum, we can use second quantization to analyze our results and find a criterion 
for the above expectation value to a local minimum. We will use Thouless' theorem and show that
\[
\frac{\langle c' |\hat{H} | c'\rangle}{\langle c' |c'\rangle} \ge \langle c |\hat{H} | c\rangle= E_0,
\]
with
\[
 {|c'\rangle} = {|c\rangle + |\delta c\rangle}.
\]
Using Thouless' theorem we can write out ${|c'\rangle}$ as
\[
 {|c'\rangle}=\exp\left\{\sum_{a > F}\sum_{i \le F}\delta C_{ai}a_{a}^{\dagger}a_{i}\right\}| c\rangle =\left\{1+\sum_{a > F}\sum_{i \le F}\delta C_{ai}a_{a}^{\dagger}
a_{i}+\frac{1}{2!}\sum_{ab > F}\sum_{ij \le F}\delta C_{ai}\delta C_{bj}a_{a}^{\dagger}a_{i}a_{b}^{\dagger}a_{j}+\dots\right\}
\]
where the amplitudes $\delta C$ are small.

The norm of $|c'\rangle$ is given by (using the intermediate normalization condition $\langle c' |c\rangle=1$) 
\[
\langle c' | c'\rangle = 1+\sum_{a>F}
\sum_{i\le F}|\delta C_{ai}|^2+O(\delta C_{ai}^3).
\]
The expectation value for the energy is now given by (using the Hartree-Fock condition)
\[
\langle c' |\hat{H} | c'\rangle=\langle c |\hat{H} | c\rangle +
\sum_{ab>F}
\sum_{ij\le F}\delta C_{ai}^*\delta C_{bj}\langle c |a_{i}^{\dagger}a_{a}\hat{H}a_{b}^{\dagger}a_{j}|c\rangle+
\]
\[
\frac{1}{2!}\sum_{ab>F}
\sum_{ij\le F}\delta C_{ai}\delta C_{bj}\langle c |\hat{H}a_{a}^{\dagger}a_{i}a_{b}^{\dagger}a_{j}|c\rangle+\frac{1}{2!}\sum_{ab>F}
\sum_{ij\le F}\delta C_{ai}^*\delta C_{bj}^*\langle c|a_{j}^{\dagger}a_{b}a_{i}^{\dagger}a_{a}\hat{H}|c\rangle
+\dots
\] 

We have already calculated the second term on the right-hand side of the previous equation
\[
\langle c | \left(\{a^\dagger_i a_a\} \hat{H} \{a^\dagger_b a_j\} \right) | c\rangle=\sum_{pq} \sum_{ijab}\delta C_{ai}^*\delta C_{bj} \langle p|\hat{h}_0 |q\rangle 
            \langle c | \left(\{a^{\dagger}_i a_a\}\{a^{\dagger}_pa_q\} 
             \{a^{\dagger}_b a_j\} \right)| c\rangle +\frac{1}{4} \sum_{pqrs} \sum_{ijab}\delta C_{ai}^*\delta C_{bj} \langle pq| \hat{v}|rs\rangle 
            \langle c | \left(\{a^\dagger_i a_a\}\{a^{\dagger}_p a^{\dagger}_q a_s  a_r\} \{a^{\dagger}_b a_j\} \right)| c\rangle ,
\]
resulting in
\[
E_0\sum_{ai}|\delta C_{ai}|^2+\sum_{ai}|\delta C_{ai}|^2(\varepsilon_a-\varepsilon_i)-\sum_{ijab} \langle aj|\hat{v}| bi\rangle \delta C_{ai}^*\delta C_{bj}.
\]
\end{block}

% !split
\subsection{Hartree-Fock in second quantization and stability of HF solution}
\begin{block}{}
\[
\frac{1}{2!}\langle c |\left(\{a^\dagger_j a_b\} \{a^\dagger_i a_a\} \hat{V}_N  \right) | c\rangle  = 
\frac{1}{2!}\langle c |\left( \hat{V}_N \{a^\dagger_a a_i\} \{a^\dagger_b a_j\} \right)^{\dagger} | c\rangle 
\]
which is nothing but
\[
\frac{1}{2!}\langle c |  \left( \hat{V}_N \{a^\dagger_a a_i\} \{a^\dagger_b a_j\} \right) | c\rangle^*
=\frac{1}{2} \sum_{ijab} (\langle ij|\hat{v}|ab\rangle)^*\delta C_{ai}^*\delta C_{bj}^*
\]
or 
\[
\frac{1}{2} \sum_{ijab} (\langle ab|\hat{v}|ij\rangle)\delta C_{ai}^*\delta C_{bj}^*
\]
where we have used the relation
\[ 
\langle a |\hat{A} | b\rangle =  (\langle b |\hat{A}^{\dagger} | a\rangle)^*
\]
due to the hermiticity of $\hat{H}$ and $\hat{V}$.
\end{block}

% !split
\subsection{Hartree-Fock in second quantization and stability of HF solution}
\begin{block}{}
We define two matrix elements
\[
A_{ai,bj}=-\langle aj|\hat{v} bi\rangle
\]
and
\[
B_{ai,bj}=\langle ab|\hat{v}|ij\rangle
\]
both being anti-symmetrized.

With these definitions we write out the energy as
\[
\langle c'|H|c'\rangle = \left(1+\sum_{ai}|\delta C_{ai}|^2\right)\langle c |H|c\rangle+\sum_{ai}|\delta C_{ai}|^2(\varepsilon_a^{HF}-\varepsilon_i^{HF})+\sum_{ijab}A_{ai,bj}\delta C_{ai}^*\delta C_{bj}+\frac{1}{2} \sum_{ijab} B_{ai,bj}^*\delta C_{ai}\delta C_{bj}+\frac{1}{2} \sum_{ijab} B_{ai,bj}\delta C_{ai}^*\delta C_{bj}^*
+O(\delta C_{ai}^3),\]
which can be rewritten as
\[
\langle c'|H|c'\rangle = \left(1+\sum_{ai}|\delta C_{ai}|^2\right)\langle c |H|c\rangle+\Delta E+O(\delta C_{ai}^3),
\]
and skipping higher-order terms we arrived
\[
\frac{\langle c' |\hat{H} | c'\rangle}{\langle c' |c'\rangle} =E_0+\frac{\Delta E}{\left(1+\sum_{ai}|\delta C_{ai}|^2\right)}.
\]
\end{block}

% !split
\subsection{Hartree-Fock in second quantization and stability of HF solution}
\begin{block}{}
We have defined 
\[
\Delta E = \frac{1}{2} \langle \chi | \hat{M}| \chi \rangle
\]
with the vectors 
\[ 
\chi = \left[ \delta C\hspace{0.2cm} \delta C^*\right]^T
\]
and the matrix 
\[
\hat{M}=\left(\begin{array}{cc} \Delta + A & B \\ B^* & \Delta + A^*\end{array}\right),
\]
with $\Delta_{ai,bj} = (\varepsilon_a-\varepsilon_i)\delta_{ab}\delta_{ij}$.

The condition
\[
\Delta E = \frac{1}{2} \langle \chi | \hat{M}| \chi \rangle \ge 0
\]
for an arbitrary  vector 
\[ 
\chi = \left[ \delta C\hspace{0.2cm} \delta C^*\right]^T
\]
means that all eigenvalues of the matrix have to be larger than or equal zero. 
A necessary (but no sufficient) condition is that the matrix elements (for all $ai$ )
\[
(\varepsilon_a-\varepsilon_i)\delta_{ab}\delta_{ij}+A_{ai,bj} \ge 0.
\]
This equation can be used as a first test of the stability of the Hartree-Fock equation.
\end{block}



% !split
\subsection{Electron gas and HF solution}
\begin{block}{}
The electron gas is perhaps the only realistic model of a 
system of many interacting particles that allows for a solution
of the Hartree-Fock equations on a closed form. Furthermore, to first order in the interaction, one can also
compute on a closed form the total energy and several other properties of a many-particle systems. 
The model gives a very good approximation to the properties of valence electrons in metals.
The assumptions are

\begin{itemize}
 \item System of electrons that is not influenced by external forces except by an attraction provided by a uniform background of ions. These ions give rise to a uniform background charge. The ions are stationary.

 \item The system as a whole is neutral.

 \item We assume we have $N_e$ electrons in a cubic box of length $L$ and volume $\Omega=L^3$. This volume contains also a
\end{itemize}

\noindent
uniform distribution of positive charge with density $N_ee/\Omega$. 


This is a homogeneous system and the one-particle wave functions are given by plane wave functions normalized to a volume $\Omega$ 
for a box with length $L$ (the limit $L\rightarrow \infty$ is to be taken after we have computed various expectation values)
\[
\psi_{{\bf k}\sigma}({\bf r})= \frac{1}{\sqrt{\Omega}}\exp{(i{\bf kr})}\xi_{\sigma}
\]
where ${\bf k}$ is the wave number and  $\xi_{\sigma}$ is a spin function for either spin up or down
\[ 
\xi_{\sigma=+1/2}=\left(\begin{array}{c} 1 \\ 0 \end{array}\right) \hspace{0.5cm}
\xi_{\sigma=-1/2}=\left(\begin{array}{c} 0 \\ 1 \end{array}\right).
\]
We assume that we have periodic boundary conditions which limit the allowed wave numbers to
\[
k_i=\frac{2\pi n_i}{L}\hspace{0.5cm} i=x,y,z \hspace{0.5cm} n_i=0,\pm 1,\pm 2, \dots
\]
We assume first that the electrons interact via a central, symmetric and translationally invariant
interaction  $V(r_{12})$ with
$r_{12}=|{\bf r}_1-{\bf r}_2|$.  The interaction is spin independent.

The total Hamiltonian consists then of kinetic and potential energy
\[
\hat{H} = \hat{T}+\hat{V}.
\]
The operator for the kinetic energy can be written as
\[
\hat{T}=\sum_{{\bf k}\sigma}\frac{\hbar^2k^2}{2m}a_{{\bf k}\sigma}^{\dagger}a_{{\bf k}\sigma}.
\]

The Hamilton operator is given by
\[
\hat{H}=\hat{H}_{el}+\hat{H}_{b}+\hat{H}_{el-b},
\]
with the electronic part
\[
\hat{H}_{el}=\sum_{i=1}^N\frac{p_i^2}{2m}+\frac{e^2}{2}\sum_{i\ne j}\frac{e^{-\mu |{\bf r}_i-{\bf r}_j|}}{|{\bf r}_i-{\bf r}_j|},
\]
where we have introduced an explicit convergence factor
(the limit $\mu\rightarrow 0$ is performed after having calculated the various integrals).
Correspondingly, we have
\[
\hat{H}_{b}=\frac{e^2}{2}\int\int d{\bf r}d{\bf r}'\frac{n({\bf r})n({\bf r}')e^{-\mu |{\bf r}-{\bf r}'|}}{|{\bf r}-{\bf r}'|},
\]
which is the energy contribution from the positive background charge with density
$n({\bf r})=N/\Omega$. Finally,
\[
\hat{H}_{el-b}=-\frac{e^2}{2}\sum_{i=1}^N\int d{\bf r}\frac{n({\bf r})e^{-\mu |{\bf r}-{\bf x}_i|}}{|{\bf r}-{\bf x}_i|},
\]
is the interaction between the electrons and the positive background.
\end{block}

% !split
\subsection{Electron gas and HF solution}
\begin{block}{}
In exercise 5 we show that the Hartree-Fock energy can be written as 
\[
\varepsilon_{k}^{HF}=\frac{\hbar^{2}k^{2}}{2m_e}-\frac{e^{2}}
{\Omega^{2}}\sum_{k'\leq
k_{F}}\int d{\bf r}e^{i({\bf k}'-{\bf k}){\bf r}}\int
d{\bf r'}\frac{e^{i({\bf k}-{\bf k}'){\bf r}'}}
{\vert{\bf r}-{\bf r}'\vert}
\]
resulting in
\[
\varepsilon_{k}^{HF}=\frac{\hbar^{2}k^{2}}{2m_e}-\frac{e^{2}
k_{F}}{2\pi}
\left[
2+\frac{k_{F}^{2}-k^{2}}{kk_{F}}ln\left\vert\frac{k+k_{F}}
{k-k_{F}}\right\vert
\right]
\]

The results can be rewritten in terms of the density
\[
n= \frac{k_F^3}{3\pi^2}=\frac{3}{4\pi r_s^3},
\]
where $n=N_e/\Omega$, $N_e$ being the number of electrons, and $r_s$ is the radius of a sphere which represents the volum per conducting electron.  
It can be convenient to use the Bohr radius $a_0=\hbar^2/e^2m_e$.
For most metals we have a relation $r_s/a_0\sim 2-6$.  The quantity $r_s$ is dimensionless.

In exercise 6 we find that
the total energy
$E_0/N_e=\langle\Phi_{0}|\hat{H}|\Phi_{0}\rangle/N_e$ for
for this system to first order in the interaction is given as 
\[
E_0/N_e=\frac{e^2}{2a_0}\left[\frac{2.21}{r_s^2}-\frac{0.916}{r_s}\right].
\]

\end{block}





% !split
\subsection{Exercises: Derivation of Hartree-Fock equations}
\begin{block}{Exercise 1 }
Consider a Slater determinant built up of single-particle orbitals $\psi_{\lambda}$, 
with $\lambda = 1,2,\dots,N$.

The unitary transformation
\[
\psi_a  = \sum_{\lambda} C_{a\lambda}\phi_{\lambda},
\]
brings us into the new basis.  
The new basis has quantum numbers $a=1,2,\dots,N$.
Show that the new basis is orthonormal.
Show that the new Slater determinant constructed from the new single-particle wave functions can be
written as the determinant based on the previous basis and the determinant of the matrix $C$.
Show that the old and the new Slater determinants are equal up to a complex constant with absolute value unity.
(Hint, $C$ is a unitary matrix). 

\end{block}

% !split
\subsection{Exercises: Derivation of Hartree-Fock equations}
\begin{block}{Exercise 2  }
Consider the  Slater  determinant
\[
\Phi_{0}=\frac{1}{\sqrt{n!}}\sum_{p}(-)^{p}P
\prod_{i=1}^{n}\psi_{\alpha_{i}}(x_{i}).
\]
A small variation in this function is given by
\[
\delta\Phi_{0}=\frac{1}{\sqrt{n!}}\sum_{p}(-)^{p}P
\psi_{\alpha_{1}}(x_{1})\psi_{\alpha_{2}}(x_{2})\dots
\psi_{\alpha_{i-1}}(x_{i-1})(\delta\psi_{\alpha_{i}}(x_{i}))
\psi_{\alpha_{i+1}}(x_{i+1})\dots\psi_{\alpha_{n}}(x_{n}).
\]
Show that
\[
\langle \delta\Phi_{0}|\sum_{i=1}^{n}\left\{t(x_{i})+u(x_{i})
\right\}+\frac{1}{2}
\sum_{i\neq j=1}^{n}v(x_{i},x_{j})|\Phi_{0}\rangle=\sum_{i=1}^{n}\langle \delta\psi_{\alpha_{i}}|\hat{t}+\hat{u}
|\phi_{\alpha_{i}}\rangle
+\sum_{i\neq j=1}^{n}\left\{\langle\delta\psi_{\alpha_{i}}
\psi_{\alpha_{j}}|\hat{v}|\psi_{\alpha_{i}}\psi_{\alpha_{j}}\rangle-
\langle\delta\psi_{\alpha_{i}}\psi_{\alpha_{j}}|\hat{v}
|\psi_{\alpha_{j}}\psi_{\alpha_{i}}\rangle\right\}
\]
\end{block}


% !split
\subsection{Exercises: Derivation of Hartree-Fock equations}
\begin{block}{Exercise 3  }
What is the diagrammatic representation of the HF equation?
\[
-\langle\alpha_{k}|u^{HF}|\alpha_{i}\rangle+\sum_{j=1}^{n}
\left[\langle\alpha_{k}\alpha_{j}|\hat{v}|\alpha_{i}\alpha_{j}\rangle-
\langle\alpha_{k}\alpha_{j}|v|\alpha_{j}\alpha_{i}\rangle\right]=0
\]
(Represent $(-u^{HF})$ by the symbol $---$X .)
\end{block}

% !split
\subsection{Exercises: Derivation of Hartree-Fock equations}
\begin{block}{Exercise 4  }
Consider the ground state $|\Phi\rangle$ 
of a bound many-particle system of fermions. Assume that we remove one particle
from the single-particle state $\lambda$ and that our system ends in a new state
$|\Phi_{n}\rangle$. 
Define the energy needed to remove this particle as
\[
E_{\lambda}=\sum_{n}\vert\langle\Phi_{n}|a_{\lambda}|\Phi\rangle\vert^{2}(E_{0}-E_{n}),
\]
where $E_{0}$ and $E_{n}$  are the ground state energies of the states
$|\Phi\rangle$  and  $|\Phi_{n}\rangle$, respectively.
\begin{itemize}
 \item Show that
\end{itemize}

\noindent
\[
E_{\lambda}=\langle\Phi|a_{\lambda}^{\dagger}\left[
a_{\lambda},H \right]|\Phi\rangle,
\]
where $H$ is the Hamiltonian of this system.
\begin{itemize}
 \item If we assume that $\Phi$ is the  Hartree-Fock result, find the 
\end{itemize}

\noindent
relation between $E_{\lambda}$ and the single-particle energy
$\varepsilon_{\lambda}$
for states $\lambda \leq F$ and $\lambda >F$, with
\[
\varepsilon_{\lambda}=\langle\lambda|\hat{t}+\hat{u}|\lambda\rangle,
\]
and
\[
\langle\lambda|\hat{u}|\lambda\rangle=\sum_{\beta \leq F}
\langle\lambda\beta|\hat{v}|\lambda\beta\rangle.
\]
We have assumed an antisymmetrized matrix element here.
Discuss the result.

The Hamiltonian operator is defined as
\[
H=\sum_{\alpha\beta}\langle\alpha|\hat{t}|\beta\rangle a_{\alpha}^{\dagger}a_{\beta}+
\frac{1}{2}\sum_{\alpha\beta\gamma\delta}\langle\alpha\beta|\hat{v}|\gamma\delta\rangle a_{\alpha}^{\dagger}a_{\beta}^{\dagger}a_{\delta}a_{\gamma}.
\]
\end{block}


% !split
\subsection{Exercises: Electron gas}
\begin{block}{Exercise 5 }
The electron gas model allows closed form solutions for quantities like the 
single-particle Hartree-Fock energy.  The latter quantity is given by the following expression
\[
\varepsilon_{k}^{HF}=\frac{\hbar^{2}k^{2}}{2m}-\frac{e^{2}}
{V^{2}}\sum_{k'\leq
k_{F}}\int d{\bf r}e^{i({\bf k'}-{\bf k}){\bf r}}\int
d{\bf r}'\frac{e^{i({\bf k}-{\bf k'}){\bf r}'}}
{\vert{\bf r}-{\bf r'}\vert}
\]
\begin{itemize}
 \item Show that
\end{itemize}

\noindent
\[
\varepsilon_{k}^{HF}=\frac{\hbar^{2}k^{2}}{2m}-\frac{e^{2}
k_{F}}{2\pi}
\left[
2+\frac{k_{F}^{2}-k^{2}}{kk_{F}}ln\left\vert\frac{k+k_{F}}
{k-k_{F}}\right\vert
\right]
\]
(Hint: Introduce the convergence factor 
$e^{-\mu\vert{\bf r}-{\bf r}'\vert}$
in the potential and use  $\sum_{{\bf k}}\rightarrow
\frac{V}{(2\pi)^{3}}\int d{\bf k}$ )
\begin{itemize}
 \item Rewrite the above result as a function of the density
\end{itemize}

\noindent
\[
n= \frac{k_F^3}{3\pi^2}=\frac{3}{4\pi r_s^3},
\]
where $n=N/V$, $N$ being the number of particles, and $r_s$ is the radius of a sphere which represents the volum per conducting electron.  
It can be convenient to use the Bohr radius $a_0=\hbar^2/e^2m$.

For most metals we have a relation $r_s/a_0\sim 2-6$.

Make a plot of the free electron energy and the Hartree-Fock energy and discuss the behavior around the Fermi surface. Extract also   the Hartree-Fock band width $\Delta\varepsilon^{HF}$ defined as
\[ 
\Delta\varepsilon^{HF}=\varepsilon_{k_{F}}^{HF}-
\varepsilon_{0}^{HF}.
\]
Compare this results with the corresponding one for a free electron and comment your results. How large is the contribution due to the exchange term in the Hartree-Fock equation?\newline
\begin{itemize}
 \item We will now define a quantity called the effective mass.
\end{itemize}

\noindent
For $\vert{\bf k}\vert$ near $k_{F}$, we can Taylor expand the Hartree-Fock energy as  
\[
\varepsilon_{k}^{HF}=\varepsilon_{k_{F}}^{HF}+
\left(\frac{\partial\varepsilon_{k}^{HF}}{\partial k}\right)_{k_{F}}(k-k_{F})+\dots
\]
If we compare the latter with the corresponding expressiyon for the non-interacting system
\[
\varepsilon_{k}^{(0)}=\frac{\hbar^{2}k^{2}_{F}}{2m}+
\frac{\hbar^{2}k_{F}}{m}\left(k-k_{F}\right)+\dots ,
\]
we can define the so-called effective Hartree-Fock mass as
\[
m_{HF}^{*}\equiv\hbar^{2}k_{F}\left(
\frac{\partial\varepsilon_{k}^{HF}}
{\partial k}\right)_{k_{F}}^{-1}
\]
Compute $m_{HF}^{*}$ and comment your results after you have done 
point d). \newline
\begin{itemize}
 \item Show that the level density (the number of single-electron states per unit energy) can be written as
\end{itemize}

\noindent
\[
n(\varepsilon)=\frac{Vk^{2}}{2\pi^{2}}\left(
\frac{\partial\varepsilon}{\partial k}\right)^{-1}
\]
Calculate $n(\varepsilon_{F}^{HF})$ and comment the results.


\end{block}



% !split
\subsection{Exercises: Electron gas}
\begin{block}{Solution to exercise 5 }
We want to show that, given the Hartree-Fock equation for the electron gas
\[
\varepsilon_{k}^{HF}=\frac{\hbar^{2}k^{2}}{2m}-\frac{e^{2}}
{V^{2}}\sum_{p\leq
k_{F}}\int d{\bf r}\exp{(i({\bf p}-{\bf k}){\bf r})}\int
d{\bf r}'\frac{\exp{(i({\bf k}-{\bf p}){\bf r}'})}
{\vert{\bf r}-{\bf r'}\vert}
\]
the single-particle energy can be written as
\[
\varepsilon_{k}^{HF}=\frac{\hbar^{2}k^{2}}{2m}-\frac{e^{2}
k_{F}}{2\pi}
\left[
2+\frac{k_{F}^{2}-k^{2}}{kk_{F}}ln\left\vert\frac{k+k_{F}}
{k-k_{F}}\right\vert
\right].
\]
We introduce the convergence factor 
$e^{-\mu\vert{\bf r}-{\bf r}'\vert}$
in the potential and use  $\sum_{{\bf k}}\rightarrow
\frac{V}{(2\pi)^{3}}\int d{\bf k}$. We can then rewrite the integral as 
\[
\frac{e^{2}}
{V^{2}}\sum_{k'\leq
k_{F}}\int d{\bf r}\exp{(i({\bf k'}-{\bf k}){\bf r})}\int
d{\bf r}'\frac{\exp{(i({\bf k}-{\bf p}){\bf r}'})}
{\vert{\bf r}-{\bf r'}\vert}=\frac{e^{2}}{V (2\pi)^3}  \int d{\bf r}\int
\frac{d{\bf r}'}{\vert{\bf r}-{\bf r'}\vert}\exp{(-i{\bf k}({\bf r}-{\bf r}'))}\int d{\bf p}\exp{(i{\bf p}({\bf r}-{\bf r}'))},
\]
and introducing the abovementioned convergence factor we have
\[
\lim_{\mu \to 0}\frac{e^{2}}{V (2\pi)^3}  \int d{\bf r}\int d{\bf r}'\frac{\exp{(-\mu\vert{\bf r}-{\bf r}'\vert})}{\vert{\bf r}-{\bf r'}\vert}\int d{\bf p}\exp{(i({\bf p}-{\bf k})({\bf r}-{\bf r}'))}.
\]
With a change variables to ${\bf x} = {\bf r}-{\bf r}'$ and ${\bf y}={\bf r}'$ we rewrite the last integral as
\[
\lim_{\mu \to 0}\frac{e^{2}}{V (2\pi)^3}  \int d{\bf p}\int d{\bf y}\int d{\bf x}\exp{(i({\bf p}-{\bf k}){\bf x})}\frac{\exp{(-\mu\vert{\bf x}\vert})}{\vert{\bf x}\vert}.
\]
The integration over ${\bf x}$ can be performed using spherical coordinates, resulting in (with $x=\vert {\bf x}\vert$)
\[
\int d{\bf x}\exp{(i({\bf p}-{\bf k}){\bf x})}\frac{\exp{(-\mu\vert{\bf x}\vert})}{\vert{\bf x}\vert}=\int x^2 dx d\phi d\cos{(\theta)}\exp{(i({\bf p}-{\bf k})x\cos{(\theta))}}\frac{\exp{(-\mu x)}}{x},
\]
which yields
\[
4\pi \int dx \frac{ \sin{(\vert {\bf p}-{\bf k}\vert)x} }{\vert {\bf p}-{\bf k}\vert}{\exp{(-\mu x)}}= \frac{4\pi}{\mu^2+\vert {\bf p}-{\bf k}\vert^2}.
\]
This results gives us 
\[
\lim_{\mu \to 0}\frac{e^{2}}{V (2\pi)^3}  \int d{\bf p}\int d{\bf y}\frac{4\pi}{\mu^2+\vert {\bf p}-{\bf k}\vert^2}=\lim_{\mu \to 0}\frac{e^{2}}{ 2\pi^2}  \int d{\bf p}\frac{1}{\mu^2+\vert {\bf p}-{\bf k}\vert^2},
\]
where we have used that the integrand on teh left-hand side does not depend on ${\bf y}$ and that $\int d{\bf y}=V$.Introducing spherical coordinates we can rewrite the integral as 
\[
\lim_{\mu \to 0}\frac{e^{2}}{ 2\pi^2}  \int d{\bf p}\frac{1}{\mu^2+\vert {\bf p}-{\bf k}\vert^2}=\frac{e^{2}}{ 2\pi^2}  \int d{\bf p}\frac{1}{\vert {\bf p}-{\bf k}\vert^2}=\frac{e^{2}}{\pi}  \int_0^{k_F} p^2dp\int_0^{\pi} d\theta\cos{(\theta)}\frac{1}{p^2+k^2-2pk\cos{(\theta)}},
\]
and with the change of variables $\cos{(\theta)}=u$ we have 
\[
\frac{e^{2}}{\pi}  \int_0^{k_F} p^2dp\int_{0}^{\pi} d\theta\cos{(\theta)}\frac{1}{p^2+k^2-2pk\cos{(\theta)}}=\frac{e^{2}}{\pi}  \int_0^{k_F} p^2dp\int_{-1}^{1} du\frac{1}{p^2+k^2-2pku},
\]
which gives
\[
\frac{e^{2}}{k\pi}  \int_0^{k_F} pdp\left\{ln(\vert p+k\vert)-ln(\vert p-k\vert)\right\},
\]
and introducing new variables $x=p+k$ and $y=p-k$, we obtain after some straightforward reordering of the integral
\[
\frac{e^{2}}{k\pi}\left[
kk_F+\frac{k_{F}^{2}-k^{2}}{kk_{F}}ln\left\vert\frac{k+k_{F}}
{k-k_{F}}\right\vert
\right],
\]
which gives the abovementioned expression for the single-particle energy.

Introducing the dimensionless quantity $x=k/k_F$ and the function
\[
F(x) = \frac{1}{2}+\frac{1-x^2}{4x}\ln{\left\vert \frac{1+x}{1-x}\right\vert},
\]
we can rewrite the single-particle Hartree-Fock energy as 
\[
\varepsilon_{k}^{HF}=\frac{\hbar^{2}k^{2}}{2m}-\frac{2e^{2}
k_{F}}{\pi}F(k/k_F),
\]
and dividing by the non-interacting contribution at the Fermi level, 
\[
\varepsilon_{0}^{HF}=\frac{\hbar^{2}k_F^{2}}{2m},
\]
we have
\[
\frac{\varepsilon_{k}^{HF} }{\varepsilon_{0}^{HF}}=x^2-\frac{e^2m}{\hbar^2 k_F\pi}F(x)=x^2-\frac{4}{\pi k_Fa_0}F(x),
\]
where $a_0=0.0529$ nm is the Bohr radius, setting thereby a natural length scale. 
By introducing the radius $r_s$ of a sphere whose volume is the volume occupied by each electron, we can rewrite the previous equation in terms of $r_s$ using that the electron density $n=N/V$
\[
n=\frac{k_F^3}{3\pi^2} = \frac{3}{4\pi r_s^3},
\]
we have (with $k_F=1.92/r_s$,
\[
\frac{\varepsilon_{k}^{HF} }{\varepsilon_{0}^{HF}}=x^2-\frac{e^2m}{\hbar^2 k_F\pi}F(x)=x^2-\frac{r_s}{a_0}0.663F(x),
\]
with $r_s \sim 2-6$ for most metals. 

We can now define the so-called bandwidth, that is the scatter between the maximal and the minimal value of the electrons in the conductance band of a metal (up to the Fermi level). For $x=1$, which corresponds to 



\end{block}


% !split
\subsection{Exercises: Electron gas}
\begin{block}{Exercise 6 }
We consider a system of electrons in infinite matter, the so-called electron gas. This is a homogeneous system and the one-particle states are given by plane wave function normalized to a volume $\Omega$ 
for a box with length $L$ (the limit $L\rightarrow \infty$ is to be taken after we have computed various expectation values)
\[
\psi_{{\bf k}\sigma}({\bf r})= \frac{1}{\sqrt{\Omega}}\exp{(i{\bf kr})}\xi_{\sigma}
\]
where ${\bf k}$ is the wave number and  $\xi_{\sigma}$ is a spin function for either spin up or down
\[ 
\xi_{\sigma=+1/2}=\left(\begin{array}{c} 1 \\ 0 \end{array}\right) \hspace{0.5cm}
\xi_{\sigma=-1/2}=\left(\begin{array}{c} 0 \\ 1 \end{array}\right).
\]
We assume that we have periodic boundary conditions which limit the allowed wave numbers to
\[
k_i=\frac{2\pi n_i}{L}\hspace{0.5cm} i=x,y,z \hspace{0.5cm} n_i=0,\pm 1,\pm 2, \dots
\]
We assume first that the particles interact via a central, symmetric and translationally invariant
interaction  $V(r_{12})$ with
$r_{12}=|{\bf r}_1-{\bf r}_2|$.  The interaction is spin independent.

The total Hamiltonian consists then of kinetic and potential energy
\[
\hat{H} = \hat{T}+\hat{V}.
\]
\begin{itemize}
 \item Show that the operator for the kinetic energy can be written as
\end{itemize}

\noindent
\[
\hat{T}=\sum_{{\bf k}\sigma}\frac{\hbar^2k^2}{2m}a_{{\bf k}\sigma}^{\dagger}a_{{\bf k}\sigma}.
\]
Find also the number operator $\hat{N}$ and find a corresponding expression for the interaction
$\hat{V}$ expressed with creation and annihilation operators.   The expression for the interaction
has to be written in  $k$ space, even though $V$ depends only on the relative distance. It means that you ned to set up the Fourier transform $\langle {\bf k}_i{\bf k}_j| V | {\bf k}_m{\bf k}_n\rangle$.
\begin{itemize}
 \item We will now study the electron gas. The Hamilton operator is given by
\end{itemize}

\noindent
\[
\hat{H}=\hat{H}_{el}+\hat{H}_{b}+\hat{H}_{el-b},
\]
with the electronic part
\[
\hat{H}_{el}=\sum_{i=1}^N\frac{p_i^2}{2m}+\frac{e^2}{2}\sum_{i\ne j}\frac{e^{-\mu |{\bf r}_i-{\bf r}_j|}}{|{\bf r}_i-{\bf r}_j|},
\]
where we have introduced an explicit convergence factor
(the limit $\mu\rightarrow 0$ is performed after having calculated the various integrals).
Correspondingly, we have
\[
\hat{H}_{b}=\frac{e^2}{2}\int\int d{\bf r}d{\bf r}'\frac{n({\bf r})n({\bf r}')e^{-\mu |{\bf r}-{\bf r}'|}}{|{\bf r}-{\bf r}'|},
\]
which is the energy contribution from the positive background charge with density
$n({\bf r})=N/\Omega$. Finally,
\[
\hat{H}_{el-b}=-\frac{e^2}{2}\sum_{i=1}^N\int d{\bf r}\frac{n({\bf r})e^{-\mu |{\bf r}-{\bf x}_i|}}{|{\bf r}-{\bf x}_i|},
\]
is the interaction between the electrons and the positive background.

Show that
\[
\hat{H}_{b}=\frac{e^2}{2}\frac{N^2}{\Omega}\frac{4\pi}{\mu^2},
\]
and
\[
\hat{H}_{el-b}=-e^2\frac{N^2}{\Omega}\frac{4\pi}{\mu^2}.
\]
Show thereafter that the final Hamiltonian can be written as 
\[
H=H_{0}+H_{I},
\]
with
\[
H_{0}={\displaystyle\sum_{{\bf k}\sigma}}
\frac{\hbar^{2}k^{2}}{2m}a_{{\bf k}\sigma}^{\dagger}
a_{{\bf k}\sigma},
\]
and
\[
H_{I}=\frac{e^{2}}{2\Omega}{\displaystyle\sum_{\sigma_{1}
\sigma_{2}}}{\displaystyle
\sum_{{\bf q}\neq 0,{\bf k},{\bf p}}}\frac{4\pi}{q^{2}}
a_{{\bf k}+{\bf q},\sigma_{1}}^{\dagger}
a_{{\bf p}-{\bf q},\sigma_{2}}^{\dagger}
a_{{\bf p}\sigma_{2}}a_{{\bf k}\sigma_{1}}.
\] 
\begin{itemize}
 \item Calculate $E_0/N=\bra{\Phi_{0}}H\ket{\Phi_{0}}/N$ for for this system to first order in the interaction. Show that, by using
\end{itemize}

\noindent
\[
\rho= \frac{k_F^3}{3\pi^2}=\frac{3}{4\pi r_0^3},
\]
with $\rho=N/\Omega$, $r_0$
being the radius of a sphere representing the volume an electron occupies 
and the Bohr radius $a_0=\hbar^2/e^2m$, 
that the energy per electron can be written as 
\[
E_0/N=\frac{e^2}{2a_0}\left[\frac{2.21}{r_s^2}-\frac{0.916}{r_s}\right].
\]
Here we have defined
$r_s=r_0/a_0$ to be a dimensionless quantity.

Plot your results. Why is this system stable?
\begin{itemize}
 \item Calculate thermodynamical quantities like the pressure, given by
\end{itemize}

\noindent
\[
P=-\left(\frac{\partial E}{\partial \Omega}\right)_N,
\]
and the bulk modulus
\[
B=-\Omega\left(\frac{\partial P}{\partial \Omega}\right)_N,
\]
and comment your results.

\end{block}


% !split
\subsection{Exercises: Electron gas}
\begin{block}{Solution to exercise 6 }
We have to show first  that
\[
\hat{H}_{b}=\frac{e^2}{2}\frac{N_e^2}{\Omega}\frac{4\pi}{\mu^2},
\]
and
\[
\hat{H}_{el-b}=-e^2\frac{N_e^2}{\Omega}\frac{4\pi}{\mu^2}.
\]
And then that the final Hamiltonian can be written as 
\[
H=H_{0}+H_{I},
\]
with
\[
H_{0}={\displaystyle\sum_{{\bf k}\sigma}}
\frac{\hbar^{2}k^{2}}{2m_e}a_{{\bf k}\sigma}^{\dagger}
a_{{\bf k}\sigma},
\]
and
\[
H_{I}=\frac{e^{2}}{2\Omega}{\displaystyle\sum_{\sigma_{1}
\sigma_{2}}}{\displaystyle
\sum_{{\bf q}\neq 0,{\bf k},{\bf p}}}\frac{4\pi}{q^{2}}
a_{{\bf k}+{\bf q},\sigma_{1}}^{\dagger}
a_{{\bf p}-{\bf q},\sigma_{2}}^{\dagger}
a_{{\bf p}\sigma_{2}}a_{{\bf k}\sigma_{1}}.
\] 

\end{block}

% !split
\subsection{Electron gas and HF solution}
\begin{block}{}
Let us now calculate the following part of the Hamiltonian
\[ \hat H_b = \frac{e^2}{2} \iint \frac{n({\bf r}) n({\bf r}')e^{-\mu|{\bf r} - {\bf r}'|}}{|{\bf r} - {\bf r}'|} d{\bf r} d{\bf r}' , 
\]
where $n({\bf r}) = N_e/\Omega$, the density of the positive background charge. We define ${\bf r}_{12} = {\bf r} - {\bf r}'$, resulting in $d{\bf r}_{12} = d{\bf r}$, and allowing us to rewrite the integral as
\[ 
\hat H_b = \frac{e^2 N_e^2}{2\Omega^2} \iint \frac{e^{-\mu |{\bf r}_{12}|}}{|{\bf r}_{12}|} d{\bf r}_{12} d{\bf r}' = \frac{e^2 N_e^2}{2\Omega} \int \frac{e^{-\mu |{\bf r}_{12}|}}{|{\bf r}_{12}|} d{\bf r}_{12} . 
\]
Here we have used that $\int \! {\bf r} = \Omega$. We change to spherical coordinates and the lack of angle 
dependencies yields a factor $4\pi$, resulting in
\[ 
\hat H_b = \frac{4\pi e^2 N_e^2}{2\Omega} \int_0^\infty re^{-\mu r} \, \mathrm{d} r . 
\]


Solving by partial integration
\[ \int_0^\infty re^{-\mu r} \, \mathrm{d} r = \left[ -\frac{r}{\mu} e^{-\mu r} \right]_0^\infty + \frac{1}{\mu} \int_0^\infty e^{-\mu r} \, \mathrm{d} r
= \frac{1}{\mu} \left[ - \frac{1}{\mu} e^{-\mu r} \right]_0^\infty = \frac{1}{\mu^2}, 
\]
gives
\[
\hat{H}_b = \frac{e^2}{2} \frac{N_e^2}{\Omega} \frac{4\pi}{\mu^2} .
\]
The next term is 
\[ 
\hat H_{el-b} = -e^2 \sum_{i = 1}^N \int \frac{n({\bf r}) e^{-\mu |{\bf r} - {\bf x}_i|}}{|{\bf r} - {\bf x}_i|} {\bf r} . 
\]
Inserting  $n({\bf r})$ and changing variables in the same way as in the previous integral ${\bf y} = {\bf r} - {\bf x}_i$, we get $\mathrm{d}^3 {\bf y} = \mathrm{d}^3 {\bf r}$. This gives
\[ 
\hat H_{el-b} = -\frac{e^2 N_e}{\Omega} \sum_{i = 1^N} \int \frac{e^{-\mu |{\bf y}|}}{|{\bf y}|} \, \mathrm{d}^3 {\bf y}
=  -\frac{4\pi e^2 N_e}{\Omega} \sum_{i = 1}^N \int_0^\infty y e^{-\mu y} \mathrm{d} y. 
\]
We have already seen this  type of integral. The answer is 
\[ 
\hat H_{el-b} = -\frac{4\pi e^2 N_e}{\Omega} \sum_{i = 1}^N \frac{1}{\mu^2}, 
\]
which gives
\[
\hat H_{el-b} = -e^2 \frac{N_e^2}{\Omega} \frac{4\pi}{\mu^2} .
\]

Finally, we need to evaluate $\hat H_{el}$. This term reads
\[ 
\hat H_{el} = \sum_{i=1}^{N_e} \frac{\hat{\vec p}_i^2}{2m_e} + \frac{e^2}{2} \sum_{i \neq j} \frac{e^{-\mu |{\bf r}_i - {\bf r}_j|}}{{\bf r}_i - {\bf r}_j} . 
\]
The last term represents the repulsion between two electrons. It is a central symmetric interaction
and is translationally invariant. The potential is given by the expression
\[ 
v(|{\bf r}|) = e^2 \frac{e^{\mu|{\bf r}|}}{|{\bf r}|}, 
\]
which we derived in connection with the Hartree-Fock derivation.


\end{block}




% ------------------- end of main content ---------------


% #ifdef PREAMBLE
\printindex

\end{document}
% #endif

