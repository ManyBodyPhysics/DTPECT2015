
% LaTeX Beamer file automatically generated from DocOnce
% https://github.com/hplgit/doconce

%-------------------- begin beamer-specific preamble ----------------------

\documentclass{beamer}

\usetheme{red_plain}
\usecolortheme{default}

% turn off the almost invisible, yet disturbing, navigation symbols:
\setbeamertemplate{navigation symbols}{}

% Examples on customization:
%\usecolortheme[named=RawSienna]{structure}
%\usetheme[height=7mm]{Rochester}
%\setbeamerfont{frametitle}{family=\rmfamily,shape=\itshape}
%\setbeamertemplate{items}[ball]
%\setbeamertemplate{blocks}[rounded][shadow=true]
%\useoutertheme{infolines}
%
%\usefonttheme{}
%\useinntertheme{}
%
%\setbeameroption{show notes}
%\setbeameroption{show notes on second screen=right}

% fine for B/W printing:
%\usecolortheme{seahorse}

\usepackage{pgf,pgfarrows,pgfnodes,pgfautomata,pgfheaps,pgfshade}
\usepackage{graphicx}
\usepackage{epsfig}
\usepackage{relsize}

\usepackage{fancybox}  % make sure fancybox is loaded before fancyvrb

\usepackage{fancyvrb}
%\usepackage{minted} % requires pygments and latex -shell-escape filename
%\usepackage{anslistings}
%\usepackage{listingsutf8}

\usepackage{amsmath,amssymb,bm}
%\usepackage[latin1]{inputenc}
\usepackage[T1]{fontenc}
\usepackage[utf8]{inputenc}
\usepackage{colortbl}
\usepackage[english]{babel}
\usepackage{tikz}
\usepackage{framed}
% Use some nice templates
\beamertemplatetransparentcovereddynamic

% --- begin table of contents based on sections ---
% Delete this, if you do not want the table of contents to pop up at
% the beginning of each section:
% (Only section headings can enter the table of contents in Beamer
% slides generated from DocOnce source, while subsections are used
% for the title in ordinary slides.)
\AtBeginSection[]
{
  \begin{frame}<beamer>[plain]
  \frametitle{}
  %\frametitle{Outline}
  \tableofcontents[currentsection]
  \end{frame}
}
% --- end table of contents based on sections ---

% If you wish to uncover everything in a step-wise fashion, uncomment
% the following command:

%\beamerdefaultoverlayspecification{<+->}

\newcommand{\shortinlinecomment}[3]{\note{\textbf{#1}: #2}}
\newcommand{\longinlinecomment}[3]{\shortinlinecomment{#1}{#2}{#3}}

\definecolor{linkcolor}{rgb}{0,0,0.4}
\hypersetup{
    colorlinks=true,
    linkcolor=linkcolor,
    urlcolor=linkcolor,
    pdfmenubar=true,
    pdftoolbar=true,
    bookmarksdepth=3
    }
\setlength{\parskip}{0pt}  % {1em}

\newenvironment{doconceexercise}{}{}
\newcounter{doconceexercisecounter}
\newenvironment{doconce:movie}{}{}
\newcounter{doconce:movie:counter}

\newcommand{\subex}[1]{\noindent\textbf{#1}}  % for subexercises: a), b), etc

%-------------------- end beamer-specific preamble ----------------------

% Add user's preamble




% insert custom LaTeX commands...

\raggedbottom
\makeindex

%-------------------- end preamble ----------------------

\begin{document}

% endif for #ifdef PREAMBLE



% ------------------- main content ----------------------

% Slides for PHY981


% ----------------- title -------------------------

\title{Nuclear Shell Model}

% ----------------- author(s) -------------------------

\author{Morten Hjorth-Jensen, National Superconducting Cyclotron Laboratory and Department of Physics and Astronomy, Michigan State University, East Lansing, MI 48824, USA {\&} Department of Physics, University of Oslo, Oslo, Norway\inst{}}
\institute{}
% ----------------- end author(s) -------------------------

\date{May 18-22
% <optional titlepage figure>
}

\begin{frame}[plain,fragile]
\titlepage
\end{frame}

\begin{frame}[plain,fragile]
\frametitle{Slater determinants as basis states, Repetition}

\begin{block}{}
The simplest possible choice for many-body wavefunctions are \textbf{product} wavefunctions.
That is
\[ 
\Psi(x_1, x_2, x_3, \ldots, x_A) \approx \phi_1(x_1) \phi_2(x_2) \phi_3(x_3) \ldots
\]
because we are really only good  at thinking about one particle at a time. Such 
product wavefunctions, without correlations, are easy to 
work with; for example, if the single-particle states $\phi_i(x)$ are orthonormal, then 
the product wavefunctions are easy to orthonormalize.   

Similarly, computing matrix elements of operators are relatively easy, because the 
integrals factorize.


The price we pay is the lack of correlations, which we must build up by using many, many product 
wavefunctions. (Thus we have a trade-off: compact representation of correlations but 
difficult integrals versus easy integrals but many states required.) 
\end{block}
\end{frame}

\begin{frame}[plain,fragile]
\frametitle{Slater determinants as basis states, repetition}

\begin{block}{}
Because we have fermions, we are required to have antisymmetric wavefunctions, e.g.
\[
\Psi(x_1, x_2, x_3, \ldots, x_A) = - \Psi(x_2, x_1, x_3, \ldots, x_A)
\]
etc. This is accomplished formally by using the determinantal formalism
\[
\Psi(x_1, x_2, \ldots, x_A) 
= \frac{1}{\sqrt{A!}} 
\det \left | 
\begin{array}{cccc}
\phi_1(x_1) & \phi_1(x_2) & \ldots & \phi_1(x_A) \\
\phi_2(x_1) & \phi_2(x_2) & \ldots & \phi_2(x_A) \\
 \vdots & & &  \\
\phi_A(x_1) & \phi_A(x_2) & \ldots & \phi_A(x_A) 
\end{array}
\right |
\]
Product wavefunction + antisymmetry = Slater determinant. 
\end{block}
\end{frame}

\begin{frame}[plain,fragile]
\frametitle{Slater determinants as basis states}

\begin{block}{}
\[
\Psi(x_1, x_2, \ldots, x_A) 
= \frac{1}{\sqrt{N!}} 
\det \left | 
\begin{array}{cccc}
\phi_1(x_1) & \phi_1(x_2) & \ldots & \phi_1(x_A) \\
\phi_2(x_1) & \phi_2(x_2) & \ldots & \phi_2(x_A) \\
 \vdots & & &  \\
\phi_A(x_1) & \phi_A(x_2) & \ldots & \phi_A(x_A) 
\end{array}
\right |
\]
Properties of the determinant (interchange of any two rows or 
any two columns yields a change in sign; thus no two rows and no 
two columns can be the same) lead to the Pauli principle:

\begin{itemize}
\item No two particles can be at the same place (two columns the same); and

\item No two particles can be in the same state (two rows the same).
\end{itemize}

\noindent
\end{block}
\end{frame}

\begin{frame}[plain,fragile]
\frametitle{Slater determinants as basis states}

\begin{block}{}
As a practical matter, however, Slater determinants beyond $N=4$ quickly become 
unwieldy. Thus we turn to the \textbf{occupation representation} or \textbf{second quantization} to simplify calculations. 

The occupation representation, using fermion \textbf{creation} and \textbf{annihilation} 
operators, is compact and efficient. It is also abstract and, at first encounter, not easy to 
internalize. It is inspired by other operator formalism, such as the ladder operators for 
the harmonic oscillator or for angular momentum, but unlike those cases, the operators \textbf{do not have coordinate space representations}.

Instead, one can think of fermion creation/annihilation operators as a game of symbols that 
compactly reproduces what one would do, albeit clumsily, with full coordinate-space Slater 
determinants. 
\end{block}
\end{frame}

\begin{frame}[plain,fragile]
\frametitle{Quick repetition of the occupation representation}

\begin{block}{}
We start with a set of orthonormal single-particle states $\{ \phi_i(x) \}$. 
(Note: this requirement, and others, can be relaxed, but leads to a 
more involved formalism.) \textbf{Any} orthonormal set will do. 

To each single-particle state $\phi_i(x)$ we associate a creation operator 
$\hat{a}^\dagger_i$ and an annihilation operator $\hat{a}_i$. 

When acting on the vacuum state $| 0 \rangle$, the creation operator $\hat{a}^\dagger_i$ causes 
a particle to occupy the single-particle state $\phi_i(x)$:
\[
\phi_i(x) \rightarrow \hat{a}^\dagger_i |0 \rangle
\]
\end{block}
\end{frame}

\begin{frame}[plain,fragile]
\frametitle{Quick repetition  of the occupation representation}

\begin{block}{}
But with multiple creation operators we can occupy multiple states:
\[
\phi_i(x) \phi_j(x^\prime) \phi_k(x^{\prime \prime}) 
\rightarrow \hat{a}^\dagger_i \hat{a}^\dagger_j \hat{a}^\dagger_k |0 \rangle.
\]

Now we impose antisymmetry, by having the fermion operators satisfy  \textbf{anticommutation relations}:
\[
\hat{a}^\dagger_i \hat{a}^\dagger_j + \hat{a}^\dagger_j \hat{a}^\dagger_i
= [ \hat{a}^\dagger_i ,\hat{a}^\dagger_j ]_+ 
= \{ \hat{a}^\dagger_i ,\hat{a}^\dagger_j \} = 0
\]
so that 
\[
\hat{a}^\dagger_i \hat{a}^\dagger_j = - \hat{a}^\dagger_j \hat{a}^\dagger_i
\]
\end{block}
\end{frame}

\begin{frame}[plain,fragile]
\frametitle{Quick repetition  of the occupation representation}

\begin{block}{}
Because of this property, automatically $\hat{a}^\dagger_i \hat{a}^\dagger_i = 0$, 
enforcing the Pauli exclusion principle.  Thus when writing a Slater determinant 
using creation operators, 
\[
\hat{a}^\dagger_i \hat{a}^\dagger_j \hat{a}^\dagger_k \ldots |0 \rangle
\]
each index $i,j,k, \ldots$ must be unique.
\end{block}
\end{frame}

\begin{frame}[plain,fragile]
\frametitle{Full Configuration Interaction Theory}

\begin{block}{}
We have defined the ansatz for the ground state as 
\[
|\Phi_0\rangle = \left(\prod_{i\le F}\hat{a}_{i}^{\dagger}\right)|0\rangle,
\]
where the index $i$ defines different single-particle states up to the Fermi level. We have assumed that we have $N$ fermions. 
A given one-particle-one-hole ($1p1h$) state can be written as
\[
|\Phi_i^a\rangle = \hat{a}_{a}^{\dagger}\hat{a}_i|\Phi_0\rangle,
\]
while a $2p2h$ state can be written as
\[
|\Phi_{ij}^{ab}\rangle = \hat{a}_{a}^{\dagger}\hat{a}_{b}^{\dagger}\hat{a}_j\hat{a}_i|\Phi_0\rangle,
\]
and a general $NpNh$ state as 
\[
|\Phi_{ijk\dots}^{abc\dots}\rangle = \hat{a}_{a}^{\dagger}\hat{a}_{b}^{\dagger}\hat{a}_{c}^{\dagger}\dots\hat{a}_k\hat{a}_j\hat{a}_i|\Phi_0\rangle.
\]
\end{block}
\end{frame}

\begin{frame}[plain,fragile]
\frametitle{Full Configuration Interaction Theory}

\begin{block}{}
We can then expand our exact state function for the ground state 
as
\[
|\Psi_0\rangle=C_0|\Phi_0\rangle+\sum_{ai}C_i^a|\Phi_i^a\rangle+\sum_{abij}C_{ij}^{ab}|\Phi_{ij}^{ab}\rangle+\dots
=(C_0+\hat{C})|\Phi_0\rangle,
\]
where we have introduced the so-called correlation operator 
\[
\hat{C}=\sum_{ai}C_i^a\hat{a}_{a}^{\dagger}\hat{a}_i  +\sum_{abij}C_{ij}^{ab}\hat{a}_{a}^{\dagger}\hat{a}_{b}^{\dagger}\hat{a}_j\hat{a}_i+\dots
\]
Since the normalization of $\Psi_0$ is at our disposal and since $C_0$ is by hypothesis non-zero, we may arbitrarily set $C_0=1$ with 
corresponding proportional changes in all other coefficients. Using this so-called intermediate normalization we have
\[
\langle \Psi_0 | \Phi_0 \rangle = \langle \Phi_0 | \Phi_0 \rangle = 1, 
\]
resulting in 
\[
|\Psi_0\rangle=(1+\hat{C})|\Phi_0\rangle.
\]
\end{block}
\end{frame}

\begin{frame}[plain,fragile]
\frametitle{Full Configuration Interaction Theory}

\begin{block}{}
We rewrite 
\[
|\Psi_0\rangle=C_0|\Phi_0\rangle+\sum_{ai}C_i^a|\Phi_i^a\rangle+\sum_{abij}C_{ij}^{ab}|\Phi_{ij}^{ab}\rangle+\dots,
\]
in a more compact form as 
\[
|\Psi_0\rangle=\sum_{PH}C_H^P\Phi_H^P=\left(\sum_{PH}C_H^P\hat{A}_H^P\right)|\Phi_0\rangle,
\]
where $H$ stands for $0,1,\dots,n$ hole states and $P$ for $0,1,\dots,n$ particle states. 
Our requirement of unit normalization gives
\[
\langle \Psi_0 | \Phi_0 \rangle = \sum_{PH}|C_H^P|^2= 1,
\]
and the energy can be written as 
\[
E= \langle \Psi_0 | \hat{H} |\Phi_0 \rangle= \sum_{PP'HH'}C_H^{*P}\langle \Phi_H^P | \hat{H} |\Phi_{H'}^{P'} \rangle C_{H'}^{P'}.
\]
\end{block}
\end{frame}

\begin{frame}[plain,fragile]
\frametitle{Full Configuration Interaction Theory}

\begin{block}{}
Normally 
\[
E= \langle \Psi_0 | \hat{H} |\Phi_0 \rangle= \sum_{PP'HH'}C_H^{*P}\langle \Phi_H^P | \hat{H} |\Phi_{H'}^{P'} \rangle C_{H'}^{P'},
\]
is solved by diagonalization setting up the Hamiltonian matrix defined by the basis of all possible Slater determinants. A diagonalization
% to do: add text about Rayleigh-Ritz
is equivalent to finding the variational minimum   of 
\[
 \langle \Psi_0 | \hat{H} |\Phi_0 \rangle-\lambda \langle \Psi_0 |\Phi_0 \rangle,
\]
where $\lambda$ is a variational multiplier to be identified with the energy of the system.
The minimization process results in 
\[
\delta\left[ \langle \Psi_0 | \hat{H} |\Phi_0 \rangle-\lambda \langle \Psi_0 |\Phi_0 \rangle\right]=
\]
\[
\sum_{P'H'}\left\{\delta[C_H^{*P}]\langle \Phi_H^P | \hat{H} |\Phi_{H'}^{P'} \rangle C_{H'}^{P'}+
C_H^{*P}\langle \Phi_H^P | \hat{H} |\Phi_{H'}^{P'} \rangle \delta[C_{H'}^{P'}]-
\lambda( \delta[C_H^{*P}]C_{H'}^{P'}+C_H^{*P}\delta[C_{H'}^{P'}]\right\} = 0.
\]
Since the coefficients $\delta[C_H^{*P}]$ and $\delta[C_{H'}^{P'}]$ are complex conjugates it is necessary and sufficient to require the quantities that multiply with $\delta[C_H^{*P}]$ to vanish.  

This leads to 
\[
\sum_{P'H'}\langle \Phi_H^P | \hat{H} |\Phi_{H'}^{P'} \rangle C_{H'}^{P'}-\lambda C_H^{P}=0,
\]
for all sets of $P$ and $H$.

If we then multiply by the corresponding $C_H^{*P}$ and sum over $PH$ we obtain
\[ 
\sum_{PP'HH'}C_H^{*P}\langle \Phi_H^P | \hat{H} |\Phi_{H'}^{P'} \rangle C_{H'}^{P'}-\lambda\sum_{PH}|C_H^P|^2=0,
\]
leading to the identification $\lambda = E$. This means that we have for all $PH$ sets
\begin{equation}
\sum_{P'H'}\langle \Phi_H^P | \hat{H} -E|\Phi_{H'}^{P'} \rangle = 0. \label{eq:fullci}
\end{equation}
\end{block}
\end{frame}

\begin{frame}[plain,fragile]
\frametitle{Full Configuration Interaction Theory}

\begin{block}{}
An alternative way to derive the last equation is to start from 
\[
(\hat{H} -E)|\Psi_0\rangle = (\hat{H} -E)\sum_{P'H'}C_{H'}^{P'}|\Phi_{H'}^{P'} \rangle=0, 
\]
and if this equation is successively projected against all $\Phi_H^P$ in the expansion of $\Psi$, then the last equation on the previous slide
results.   As stated previously, one solves this equation normally by diagonalization. If we are able to solve this equation exactly (that is
numerically exactly) in a large Hilbert space (it will be truncated in terms of the number of single-particle states included in the definition
of Slater determinants), it can then serve as a benchmark for other many-body methods which approximate the correlation operator
$\hat{C}$.  
\end{block}
\end{frame}

\begin{frame}[plain,fragile]
\frametitle{Full Configuration Interaction Theory}

\begin{block}{}
For reasons to come (links with Coupled-Cluster theory and Many-Body perturbation theory), 
we will rewrite Eq.~(~\ref{eq:fullci}) as a set of coupled non-linear equations in terms of the unknown coefficients $C_H^P$. 

To see this, we look at the contributions arising from 
\[
\langle \Phi_H^P | = \langle \Phi_0|
\]
in  Eq.~(\ref{eq:fullci}), that is we multiply with $\langle \Phi_0 |$
from the left in 
\[
(\hat{H} -E)\sum_{P'H'}C_{H'}^{P'}|\Phi_{H'}^{P'} \rangle=0. 
\]
If we assume that we have a two-body operator at most, Slater's rule gives then an equation for the 
correlation energy in terms of $C_i^a$ and $C_{ij}^{ab}$ only.  We get then
\[
\langle \Phi_0 | \hat{H} -E| \Phi_0\rangle + \sum_{ai}\langle \Phi_0 | \hat{H} -E|\Phi_{i}^{a} \rangle C_{i}^{a}+
\sum_{abij}\langle \Phi_0 | \hat{H} -E|\Phi_{ij}^{ab} \rangle C_{ij}^{ab}=0,
\]
or 
\[
E-E_0 =\Delta E=\sum_{ai}\langle \Phi_0 | \hat{H}|\Phi_{i}^{a} \rangle C_{i}^{a}+
\sum_{abij}\langle \Phi_0 | \hat{H}|\Phi_{ij}^{ab} \rangle C_{ij}^{ab},
\]
where the energy $E_0$ is the reference energy and $\Delta E$ defines the so-called correlation energy.
The single-particle basis functions  could be the results of a Hartree-Fock calculation or just the eigenstates of the non-interacting part of the Hamiltonian. 

In our chapter on Hartree-Fock calculations, 
we have already computed the matrix $\langle \Phi_0 | \hat{H}|\Phi_{i}^{a}\rangle $ and $\langle \Phi_0 | \hat{H}|\Phi_{ij}^{ab}\rangle$.  If we are using a Hartree-Fock basis, then the matrix elements
$\langle \Phi_0 | \hat{H}|\Phi_{i}^{a}\rangle $ and we are left with a correlation energy given by
\[
E-E_0 =\Delta E^{HF}=\sum_{abij}\langle \Phi_0 | \hat{H}|\Phi_{ij}^{ab} \rangle C_{ij}^{ab}. 
\]

\end{block}
\end{frame}

\begin{frame}[plain,fragile]
\frametitle{Full Configuration Interaction Theory}

\begin{block}{}
Inserting the various matrix elements we can rewrite the previous equation as
\[
\Delta E=\sum_{ai}\langle i| \hat{f}|a \rangle C_{i}^{a}+
\sum_{abij}\langle ij | \hat{v}| ab \rangle C_{ij}^{ab}.
\]
This equation determines the correlation energy but not the coefficients $C$. 
We need more equations. Our next step is to set up
\[
\langle \Phi_i^a | \hat{H} -E| \Phi_0\rangle + \sum_{bj}\langle \Phi_i^a | \hat{H} -E|\Phi_{j}^{b} \rangle C_{j}^{b}+
\sum_{bcjk}\langle \Phi_i^a | \hat{H} -E|\Phi_{jk}^{bc} \rangle C_{jk}^{bc}+
\sum_{bcdjkl}\langle \Phi_i^a | \hat{H} -E|\Phi_{jkl}^{bcd} \rangle C_{jkl}^{bcd}=0,
\]
as this equation will allow us to find an expression for the coefficents $C_i^a$ since we can rewrite this equation as 
\[
\langle i | \hat{f}| a\rangle +\langle \Phi_i^a | \hat{H}|\Phi_{i}^{a} \rangle C_{i}^{a}+ \sum_{bj\ne ai}\langle \Phi_i^a | \hat{H}|\Phi_{j}^{b} \rangle C_{j}^{b}+
\sum_{bcjk}\langle \Phi_i^a | \hat{H}|\Phi_{jk}^{bc} \rangle C_{jk}^{bc}+
\sum_{bcdjkl}\langle \Phi_i^a | \hat{H}|\Phi_{jkl}^{bcd} \rangle C_{jkl}^{bcd}=0.
\]
\end{block}
\end{frame}

\begin{frame}[plain,fragile]
\frametitle{Full Configuration Interaction Theory}

\begin{block}{}
We rewrite this equation as
\[
C_{i}^{a}=-(\langle \Phi_i^a | \hat{H}|\Phi_{i}^{a})^{-1}
\]
\[
\times\left(\langle i | \hat{f}| a\rangle+ \sum_{bj\ne ai}\langle \Phi_i^a | \hat{H}|\Phi_{j}^{b} \rangle C_{j}^{b}+\sum_{bcjk}\langle \Phi_i^a | \hat{H}|\Phi_{jk}^{bc} \rangle C_{jk}^{bc}+
\sum_{bcdjkl}\langle \Phi_i^a | \hat{H}|\Phi_{jkl}^{bcd} \rangle C_{jkl}^{bcd}\right).
\]
Since these equations are solved iteratively ( that is we can start with a guess for the coefficients $C_i^a$), it is common to start the  iteration 
by setting 
\[
 C_{i}^{a}=-\frac{\langle i | \hat{f}| a\rangle}{\langle \Phi_i^a | \hat{H}|\Phi_{i}^{a}\rangle},
\]
and the denominator can be written as
\[
  C_{i}^{a}=\frac{\langle i | \hat{f}| a\rangle}{\langle i | \hat{f}| i\rangle-\langle a | \hat{f}| a\rangle+\langle ai | \hat{v}| ai\rangle}.
\]
The observant reader will however see that we need an equation for $C_{jk}^{bc}$ and $C_{jkl}^{bcd}$ as well.
To find equations for these coefficients we need then to continue our multiplications from the left with the various
$\Phi_{H}^P$ terms. 


\end{block}
\end{frame}

\begin{frame}[plain,fragile]
\frametitle{Full Configuration Interaction Theory}

\begin{block}{}
For $C_{jk}^{bc}$ we need then
\[
\langle \Phi_{ij}^{ab} | \hat{H} -E| \Phi_0\rangle + \sum_{kc}\langle \Phi_{ij}^{ab} | \hat{H} -E|\Phi_{k}^{c} \rangle C_{k}^{c}+
\]
\[
\sum_{cdkl}\langle \Phi_{ij}^{ab} | \hat{H} -E|\Phi_{kl}^{cd} \rangle C_{kl}^{cd}+\sum_{cdeklm}\langle \Phi_{ij}^{ab} | \hat{H} -E|\Phi_{klm}^{cde} \rangle C_{klm}^{cde}+\sum_{cdefklmn}\langle \Phi_{ij}^{ab} | \hat{H} -E|\Phi_{klmn}^{cdef} \rangle C_{klmn}^{cdef}=0,
\]
and we can isolate the coefficients $C_{kl}^{cd}$ in a similar way as we did for the coefficients $C_{i}^{a}$. 
At the end we can rewrite our solution of the Schr\"odinger equation in terms of $n$ coupled equations for the coefficients $C_H^P$.
This is a very cumbersome way of solving the equation. However, by using this iterative scheme we can illustrate how we can compute the
various terms in the wave operator or correlation operator $\hat{C}$. We will later identify the calculation of the various terms $C_H^P$
as parts of different many-body approximations to full CI. In particular, ww can  relate this non-linear scheme with Coupled Cluster theory and
many-body perturbation theory. .
\end{block}
\end{frame}

\begin{frame}[plain,fragile]
\frametitle{Full Configuration Interaction Theory}

\begin{block}{}
If we use a Hartree-Fock basis, we simplify this equation
\[
\Delta E=\sum_{ai}\langle i| \hat{f}|a \rangle C_{i}^{a}+
\sum_{abij}\langle ij | \hat{v}| ab \rangle C_{ij}^{ab}.
\]
What about
\[
\langle \Phi_i^a | \hat{H} -E| \Phi_0\rangle + \sum_{bj}\langle \Phi_i^a | \hat{H} -E|\Phi_{j}^{b} \rangle C_{j}^{b}+
\sum_{bcjk}\langle \Phi_i^a | \hat{H} -E|\Phi_{jk}^{bc} \rangle C_{jk}^{bc}+
\sum_{bcdjkl}\langle \Phi_i^a | \hat{H} -E|\Phi_{jkl}^{bcd} \rangle C_{jkl}^{bcd}=0,
\]
and
\[
\langle \Phi_{ij}^{ab} | \hat{H} -E| \Phi_0\rangle + \sum_{kc}\langle \Phi_{ij}^{ab} | \hat{H} -E|\Phi_{k}^{c} \rangle C_{k}^{c}+
\sum_{cdkl}\langle \Phi_{ij}^{ab} | \hat{H} -E|\Phi_{kl}^{cd} \rangle C_{kl}^{cd}+
\]
\[
\sum_{cdeklm}\langle \Phi_{ij}^{ab} | \hat{H} -E|\Phi_{klm}^{cde} \rangle C_{klm}^{cde}+\sum_{cdefklmn}\langle \Phi_{ij}^{ab} | \hat{H} -E|\Phi_{klmn}^{cdef} \rangle C_{klmn}^{cdef}=0?
\]
\end{block}
\end{frame}

\begin{frame}[plain,fragile]
\frametitle{Building a many-body basis}

\begin{block}{}

Let us now sketch how  construct a working code that constructs the 
many-body Hamiltonian matrix in a basis of Slater determinants and allows us to find the low-lying eigenenergies. 
This is referred to as the configuration-interaction method or shell-model diagonalization 
(or the interacting shell model). 

The first step in such codes is to construct the many-body basis.  

While the formalism is independent of the choice of basis, the \textbf{effectiveness} of a calculation 
will certainly be basis dependent. 

Furthermore there are common conventions useful to know.
\end{block}
\end{frame}

\begin{frame}[plain,fragile]
\frametitle{Building a many-body basis}

\begin{block}{}
First, the single-particle basis has angular momentum as a good quantum number.  You can 
imagine the single-particle wavefunctions being generated by a one-body Hamiltonian, 
for example a harmonic oscillator.  Modifications include harmonic oscillator plus 
spin-orbit splitting, or self-consistent mean-field potentials, or the Woods-Saxon potential which mocks 
up the self-consistent mean-field. 


For nuclei, the harmonic oscillator, modified by spin-orbit splitting, provides a useful language 
for describing single-particle states.


Each single-particle state is labeled by the following quantum numbers: 

\begin{itemize}
\item Orbital angular momentum $l$

\item Intrinsic spin $s$ = 1/2 for protons and neutrons

\item Angular momentum $j = l \pm 1/2$

\item $z$-component $j_z$ (or $m$)

\item Some labeling of the radial wavefunction, typically $n$ the number of nodes in  the radial wavefunction, but in the case of harmonic oscillator one can also use the principal quantum number $N$, where the harmonic oscillator energy is $(N+3/2)\hbar \omega$. 
\end{itemize}

\noindent
\end{block}
\end{frame}

\begin{frame}[plain,fragile]
\frametitle{Building a many-body basis}

\begin{block}{}
In this format one labels states by $n(l)_j$, with $(l)$ replaced by a letter:
$s$ for $l=0$, $p$ for $l=1$, $d$ for $l=2$, $f$ for $l=3$, and thenceforth alphabetical.


 In practice the single-particle space has to be severely truncated.  This truncation is 
typically based upon the single-particle energies, which is the effective energy 
from a mean-field potential. 

Sometimes we freeze the core and only consider a valence space. For example, one 
may assume a frozen ${}^{4}\mbox{He}$ core, with two protons and two neutrons in the $0s_{1/2}$ 
shell, and then only allow active particles in the $0p_{1/2}$ and $0p_{3/2}$ orbits. 
\end{block}
\end{frame}

\begin{frame}[plain,fragile]
\frametitle{Building a many-body basis}

\begin{block}{}

Another example is a frozen ${}^{16}\mbox{O}$ core, with eight protons and eight neutrons filling the 
$0s_{1/2}$,  $0p_{1/2}$ and $0p_{3/2}$ orbits, with valence particles in the 
$0d_{5/2}, 1s_{1/2}$ and $0d_{3/2}$ orbits.


Sometimes we refer to nuclei by the valence space where their last nucleons go.  
So, for example, we call ${}^{12}\mbox{C}$ a $p$-shell nucleus, while ${}^{26}\mbox{Al}$ is an 
$sd$-shell nucleus and ${}^{56}\mbox{Fe}$ is a $pf$-shell nucleus.
\end{block}
\end{frame}

\begin{frame}[plain,fragile]
\frametitle{Building a many-body basis}

\begin{block}{}
There are different kinds of truncations.

\begin{itemize}
\item For example, one can start with `filled' orbits (almost always the lowest), and then  allow one, two, three... particles excited out of those filled orbits. These are called  1p-1h, 2p-2h, 3p-3h excitations. 

\item Alternately, one can state a maximal orbit and allow all possible configurations with  particles occupying states up to that maximum. This is called \emph{full configuration}.

\item Finally, for particular use in nuclear physics, there is the \emph{energy} truncation, also  called the $N\hbar\Omega$ or $N_{max}$ truncation. 
\end{itemize}

\noindent
\end{block}
\end{frame}

\begin{frame}[plain,fragile]
\frametitle{Building a many-body basis}

\begin{block}{}
 Here one works in a harmonic oscillator basis, with each major oscillator shell assigned 
a principal quantum number $N=0,1,2,3,...$. 

The $N\hbar\Omega$ or $N_{max}$ truncation: Any configuration is given an noninteracting energy, which is the sum 
of the single-particle harmonic oscillator energies. (Thus this ignores 
spin-orbit splitting.)
\end{block}
\end{frame}

\begin{frame}[plain,fragile]
\frametitle{Building a many-body basis}

\begin{block}{}
Excited state are labeled relative to the lowest configuration by the 
number of harmonic oscillator quanta.

This truncation is useful because: if one includes \emph{all} configuration up to 
some $N_{max}$, and has a translationally invariant interaction, then the intrinsic 
motion and the center-of-mass motion factor. In other words, we can know exactly 
the center-of-mass wavefunction. 
\end{block}
\end{frame}

\begin{frame}[plain,fragile]
\frametitle{Building a many-body basis}

\begin{block}{}
In almost all cases, the many-body Hamiltonian is rotationally invariant. This means 
it commutes with the operators $\hat{J}^2, \hat{J}_z$ and so eigenstates will have 
good $J,M$. Furthermore, the eigenenergies do not depend upon the orientation $M$. 


Therefore we can choose to construct a many-body basis which has fixed $M$; this is 
called an $M$-scheme basis. 


Alternately, one can construct a many-body basis which has fixed $J$, or a $J$-scheme 
basis. 

The Hamiltonian matrix will have smaller dimensions (a factor of 10 or more)
 in the $J$-scheme than in the $M$-scheme. 
On the other hand, as we'll show in the next slide, the $M$-scheme is very easy to 
construct with Slater determinants, while the $J$-scheme basis states, and thus the 
matrix elements, are more complicated, almost always being linear combinations of 
$M$-scheme states. $J$-scheme bases are important and useful, but we'll focus on the 
simpler $M$-scheme.
\end{block}
\end{frame}

\begin{frame}[plain,fragile]
\frametitle{Building a many-body basis}

\begin{block}{}
The quantum number $m$ is additive (because the underlying group is Abelian): 
if a Slater determinant $\hat{a}_i^\dagger \hat{a}^\dagger_j \hat{a}^\dagger_k \ldots | 0 \rangle$ 
is built from single-particle states all with good $m$, then the total 
\[
M = m_i + m_j + m_k + \ldots
\]
This is \emph{not} true of $J$, because the angular momentum group SU(2) is not Abelian.

The upshot is that 
\begin{itemize}
\item It is easy to construct a Slater determinant with good total $M$;

\item It is trivial to calculate $M$ for each Slater determinant;

\item So it is easy to construct an $M$-scheme basis with fixed total $M$.
\end{itemize}

\noindent
Note that the individual $M$-scheme basis states will \emph{not}, in general, 
have good total $J$. 
Because the Hamiltonian is rotationally invariant, however, the eigenstates will 
have good $J$. (The situation is muddied when one has states of different $J$ that are 
nonetheless degenerate.) 
\end{block}
\end{frame}

\begin{frame}[plain,fragile]
\frametitle{Building a many-body basis}

\begin{block}{}
Example: two $j=1/2$ orbits


{\footnotesize
\begin{tabular}{ccccc}
\hline
\multicolumn{1}{c}{ Index } & \multicolumn{1}{c}{ $n$ } & \multicolumn{1}{c}{ $l$ } & \multicolumn{1}{c}{ $j$ } & \multicolumn{1}{c}{ $m_j$ } \\
\hline
1     & 0   & 0   & 1/2 & -1/2  \\
2     & 0   & 0   & 1/2 & 1/2   \\
3     & 1   & 0   & 1/2 & -1/2  \\
4     & 1   & 0   & 1/2 & 1/2   \\
\hline
\end{tabular}
}

\noindent
Note: the order is arbitrary.
There are $\left ( \begin{array}{c} 4 \\ 2 \end{array} \right) = 6$ two-particle states, 
which we list with the total $M$:


{\footnotesize
\begin{tabular}{cc}
\hline
\multicolumn{1}{c}{ Occupied } & \multicolumn{1}{c}{ $M$ } \\
\hline
1,2      & 0   \\
1,3      & -1  \\
1,4      & 0   \\
2,3      & 0   \\
2,4      & 1   \\
3,4      & 0   \\
\hline
\end{tabular}
}

\noindent
and 1 each with $M = \pm 1$.
\end{block}
\end{frame}

\begin{frame}[plain,fragile]
\frametitle{Building a many-body basis}

\begin{block}{}

Example: consider using only single particle states from the $0d_{5/2}$ space. 
They have the following quantum numbers


{\footnotesize
\begin{tabular}{ccccc}
\hline
\multicolumn{1}{c}{ Index } & \multicolumn{1}{c}{ $n$ } & \multicolumn{1}{c}{ $l$ } & \multicolumn{1}{c}{ $j$ } & \multicolumn{1}{c}{ $m_j$ } \\
\hline
1     & 0   & 2   & 5/2 & -5/2  \\
2     & 0   & 2   & 5/2 & -3/2  \\
3     & 0   & 2   & 5/2 & -1/2  \\
4     & 0   & 2   & 5/2 & 1/2   \\
5     & 0   & 2   & 5/2 & 3/2   \\
6     & 0   & 2   & 5/2 & 5/2   \\
\hline
\end{tabular}
}

\noindent
There are $\left ( \begin{array}{c} 6 \\ 2 \end{array} \right) = 15$ two-particle states, 
which we list with the total $M$:


{\footnotesize
\begin{tabular}{cccccc}
\hline
\multicolumn{1}{c}{ Occupied } & \multicolumn{1}{c}{ $M$ } & \multicolumn{1}{c}{ Occupied } & \multicolumn{1}{c}{ $M$ } & \multicolumn{1}{c}{ Occupied } & \multicolumn{1}{c}{ $M$ } \\
\hline
1,2      & -4  & 2,3      & -2  & 3,5      & 1   \\
1,3      & -3  & 2,4      & -1  & 3,6      & 2   \\
1,4      & -2  & 2,5      & 0   & 4,5      & 2   \\
1,5      & -1  & 2,6      & 1   & 4,6      & 3   \\
1,6      & 0   & 3,4      & 0   & 5,6      & 4   \\
\hline
\end{tabular}
}

\noindent
\end{block}
\end{frame}

\begin{frame}[plain,fragile]
\frametitle{Example case: pairing Hamiltonian}

\begin{block}{}

We consider a space with $2\Omega$ single-particle states, with each 
state labeled by 
$k = 1, 2, 3, \Omega$ and $m = \pm 1/2$. The convention is that 
the state with $k>0$ has $m = + 1/2$ while $-k$ has $m = -1/2$.


The Hamiltonian we consider is 
\[
\hat{H} = -G \hat{P}_+ \hat{P}_-,
\]
where
\[
\hat{P}_+ = \sum_{k > 0} \hat{a}^\dagger_k \hat{a}^\dagger_{-{k}}.
\]
and $\hat{P}_- = ( \hat{P}_+)^\dagger$.

This problem can be solved using what is called the quasi-spin formalism to obtain the 
exact results. Thereafter we will try again using the explicit Slater determinant formalism.
\end{block}
\end{frame}

\begin{frame}[plain,fragile]
\frametitle{Example case: pairing Hamiltonian}

\begin{block}{}

One can show (and this is part of the project) that
\[
\left [ \hat{P}_+, \hat{P}_- \right ] = \sum_{k> 0} \left( \hat{a}^\dagger_k \hat{a}_k 
+ \hat{a}^\dagger_{-{k}} \hat{a}_{-{k}} - 1 \right) = \hat{N} - \Omega.
\]
Now define 
\[
\hat{P}_z = \frac{1}{2} ( \hat{N} -\Omega).
\]
Finally you can show
\[
\left [ \hat{P}_z , \hat{P}_\pm \right ] = \pm \hat{P}_\pm.
\]
This means the operators $\hat{P}_\pm, \hat{P}_z$ form a so-called  $SU(2)$ algebra, and we can 
use all our insights about angular momentum, even though there is no actual 
angular momentum involved

So we rewrite the Hamiltonian to make this explicit:
\[
\hat{H} = -G \hat{P}_+ \hat{P}_- 
= -G \left( \hat{P}^2 - \hat{P}_z^2 + \hat{P}_z\right)
\]

\end{block}
\end{frame}

\begin{frame}[plain,fragile]
\frametitle{Example case: pairing Hamiltonian}

\begin{block}{}

Because of the SU(2) algebra, we know that the eigenvalues of 
$\hat{P}^2$ must be of the form $p(p+1)$, with $p$ either integer or half-integer, and the eigenvalues of $\hat{P}_z$ 
are $m_p$ with $p \geq | m_p|$, with $m_p$ also integer or half-integer. 


But because $\hat{P}_z = (1/2)(\hat{N}-\Omega)$, we know that for $N$ particles 
the value $m_p = (N-\Omega)/2$. Furthermore, the values of $m_p$ range from 
$-\Omega/2$ (for $N=0$) to $+\Omega/2$ (for $N=2\Omega$, with all states filled). 

We deduce the maximal $p = \Omega/2$ and for a given $n$ the 
values range of $p$ range from $|N-\Omega|/2$ to $\Omega/2$ in steps of 1 
(for an even number of particles) 


Following Racah we introduce the notation
$p = (\Omega - v)/2$
where $v = 0, 2, 4,..., \Omega - |N-\Omega|$ 
With this it is easy to deduce that the eigenvalues of the pairing Hamiltonian are
\[
-G(N-v)(2\Omega +2-N-v)/4
\]
This also works for $N$ odd, with $v= 1,3,5, \dots$.
\end{block}
\end{frame}

\begin{frame}[plain,fragile]
\frametitle{Example case: pairing Hamiltonian}

\begin{block}{}

Let's take a specific example: $\Omega = 3$ so there are 6 single-particle states, 
and $N = 3$, with $v= 1,3$. Therefore there are two distinct eigenvalues, 
\[
E = -2G, 0
\]
Now let's work this out explicitly. The single particle degrees of freedom are defined as


{\footnotesize
\begin{tabular}{ccc}
\hline
\multicolumn{1}{c}{ Index } & \multicolumn{1}{c}{ $k$ } & \multicolumn{1}{c}{ $m$ } \\
\hline
1     & 1   & -1/2 \\
2     & -1  & 1/2  \\
3     & 2   & -1/2 \\
4     & -2  & 1/2  \\
5     & 3   & -1/2 \\
6     & -3  & 1/2  \\
\hline
\end{tabular}
}

\noindent
 There are  $\left( \begin{array}{c}6 \\ 3 \end{array} \right) = 20$ three-particle states, but there 
are 9 states with $M = +1/2$, namely
$| 1,2,3 \rangle, |1,2,5\rangle, | 1,4,6 \rangle, | 2,3,4 \rangle, |2,3,6 \rangle, | 2,4,5 \rangle, | 2, 5, 6 \rangle, |3,4,6 \rangle, | 4,5,6 \rangle$.

\end{block}
\end{frame}

\begin{frame}[plain,fragile]
\frametitle{Example case: pairing Hamiltonian}

\begin{block}{}

In this basis, the operator 
\[
\hat{P}_+
= \hat{a}^\dagger_1 \hat{a}^\dagger_2 + \hat{a}^\dagger_3 \hat{a}^\dagger_4 +
\hat{a}^\dagger_5 \hat{a}^\dagger_6 
\]
From this we can determine that 
\[
\hat{P}_- | 1, 4, 6 \rangle = \hat{P}_- | 2, 3, 6 \rangle
= \hat{P}_- | 2, 4, 5 \rangle = 0
\]
so those states all have eigenvalue 0.
\end{block}
\end{frame}

\begin{frame}[plain,fragile]
\frametitle{Example case: pairing Hamiltonian}

\begin{block}{}
Now for further example, 
\[
\hat{P}_- | 1,2,3 \rangle = | 3 \rangle
\]
so
\[
\hat{P}_+ \hat{P}_- | 1,2,3\rangle = | 1,2,3\rangle+ | 3,4,3\rangle + | 5,6,3\rangle
\]
The second term vanishes because state 3 is occupied twice, and reordering the last 
term we
get
\[
\hat{P}_+ \hat{P}_- | 1,2,3\rangle = | 1,2,3\rangle+ |3, 5,6\rangle
\]
without picking up a phase.
\end{block}
\end{frame}

\begin{frame}[plain,fragile]
\frametitle{Example case: pairing Hamiltonian}

\begin{block}{}

Continuing in this fashion, with the previous ordering of the many-body states
(  $| 1,2,3 \rangle, |1,2,5\rangle, | 1,4,6 \rangle, | 2,3,4 \rangle, |2,3,6 \rangle, | 2,4,5 \rangle, | 2, 5, 6 \rangle, |3,4,6 \rangle, | 4,5,6 \rangle$) the 
Hamiltonian matrix of this system is 
\[
H = -G\left( 
\begin{array}{ccccccccc}
1 & 0 & 0 & 0 & 0 & 0 & 0 & 0 & 1  \\
0 & 1 & 0 & 0 & 0 & 0 & 0 & 1 & 0  \\
0 & 0 & 0 & 0 & 0 & 0 & 0 & 0 & 0  \\
0 & 0 & 0 & 1 & 0 & 0 & 1 & 0 & 0  \\
0 & 0 & 0 & 0 & 0 & 0 & 0 & 0 & 0  \\
0 & 0 & 0 & 0 & 0 & 0 & 0 & 0 & 0  \\
0 & 0 & 0 & 1 & 0 & 0 & 1 & 0 & 0  \\
0 & 0 & 0 & 0 & 0 & 0 & 0 & 0 & 0  \\
0 & 1 & 0 & 0 & 0 & 0 & 0 & 1 & 0  \\
1 & 0 & 0 & 0 & 0 & 0 & 0 & 0 & 1  
\end{array} \right )
\] 
This is useful for our project.  One can by hand confirm 
that there are 3 eigenvalues $-2G$ and 6 with value zero.

\end{block}
\end{frame}

\begin{frame}[plain,fragile]
\frametitle{Example case: pairing Hamiltonian}

\begin{block}{}

Another example
Using the $(1/2)^4$ single-particle space, resulting in eight single-particle states


{\footnotesize
\begin{tabular}{ccccc}
\hline
\multicolumn{1}{c}{ Index } & \multicolumn{1}{c}{ $n$ } & \multicolumn{1}{c}{ $l$ } & \multicolumn{1}{c}{ $s$ } & \multicolumn{1}{c}{ $m_s$ } \\
\hline
1     & 0   & 0   & 1/2 & -1/2  \\
2     & 0   & 0   & 1/2 & 1/2   \\
3     & 1   & 0   & 1/2 & -1/2  \\
4     & 1   & 0   & 1/2 & 1/2   \\
5     & 2   & 0   & 1/2 & -1/2  \\
6     & 2   & 0   & 1/2 & 1/2   \\
7     & 3   & 0   & 1/2 & -1/2  \\
8     & 3   & 0   & 1/2 & 1/2   \\
\hline
\end{tabular}
}

\noindent
and then taking only 4-particle, $M=0$ states that have no `broken pairs', there are six basis Slater 
determinants:

\begin{itemize}
\item $|           1,           2 ,          3         ,       4  \rangle , $

\item $|            1      ,     2        ,        5         ,       6 \rangle , $

\item $|            1         ,       2     ,           7         ,       8  \rangle , $

\item $|            3        ,        4      ,          5          ,      6  \rangle , $

\item $|            3        ,        4      ,          7         ,       8  \rangle , $

\item $|            5        ,        6     ,           7     ,           8  \rangle $
\end{itemize}

\noindent
\end{block}
\end{frame}

\begin{frame}[plain,fragile]
\frametitle{Example case: pairing Hamiltonian}

\begin{block}{}

Now we take the following Hamiltonian
\[
\hat{H} = \sum_n n \delta \hat{N}_n  - G \hat{P}^\dagger \hat{P}
\]
where 
\[
\hat{N}_n = \hat{a}^\dagger_{n, m=+1/2} \hat{a}_{n, m=+1/2} +
\hat{a}^\dagger_{n, m=-1/2} \hat{a}_{n, m=-1/2}
\]
and
\[
\hat{P}^\dagger = \sum_{n} \hat{a}^\dagger_{n, m=+1/2} \hat{a}^\dagger_{n, m=-1/2} 
\]
We can write down the $ 6 \times 6$  Hamiltonian in the basis from the prior slide:
\[
H = \left ( 
\begin{array}{cccccc}
2\delta -2G & -G & -G & -G & -G & 0 \\
 -G & 4\delta -2G & -G & -G & -0 & -G \\
-G & -G & 6\delta -2G & 0 & -G & -G \\
 -G & -G & 0 & 6\delta-2G & -G & -G \\
 -G & 0 & -G & -G & 8\delta-2G & -G \\
0 & -G & -G & -G & -G & 10\delta -2G 
\end{array} \right )
\]
(You should check by hand that this is correct.) 

For $\delta = 0$ we have the closed form solution of  the g.s. energy given by $-6G$.

\end{block}
\end{frame}

\end{document}
