%%
%% Automatically generated file from DocOnce source
%% (https://github.com/hplgit/doconce/)
%%
%%


%-------------------- begin preamble ----------------------

\documentclass[%
twoside,                 % oneside: electronic viewing, twoside: printing
final,                   % or draft (marks overfull hboxes, figures with paths)
10pt]{article}

\listfiles               % print all files needed to compile this document

\usepackage{relsize,makeidx,color,setspace,amsmath,amsfonts}
\usepackage[table]{xcolor}
\usepackage{bm,microtype}

\usepackage{fancyvrb} % packages needed for verbatim environments
\usepackage{minted}
\usemintedstyle{default}

\usepackage[T1]{fontenc}
%\usepackage[latin1]{inputenc}
\usepackage{ucs}
\usepackage[utf8x]{inputenc}

\usepackage{lmodern}         % Latin Modern fonts derived from Computer Modern

% Hyperlinks in PDF:
\definecolor{linkcolor}{rgb}{0,0,0.4}
\usepackage{hyperref}
\hypersetup{
    breaklinks=true,
    colorlinks=true,
    linkcolor=linkcolor,
    urlcolor=linkcolor,
    citecolor=black,
    filecolor=black,
    %filecolor=blue,
    pdfmenubar=true,
    pdftoolbar=true,
    bookmarksdepth=3   % Uncomment (and tweak) for PDF bookmarks with more levels than the TOC
    }
%\hyperbaseurl{}   % hyperlinks are relative to this root

\setcounter{tocdepth}{2}  % number chapter, section, subsection

\usepackage[framemethod=TikZ]{mdframed}

% --- begin definitions of admonition environments ---

% --- end of definitions of admonition environments ---

% prevent orhpans and widows
\clubpenalty = 10000
\widowpenalty = 10000

% --- end of standard preamble for documents ---


% insert custom LaTeX commands...

\raggedbottom
\makeindex

%-------------------- end preamble ----------------------

\begin{document}

% endif for #ifdef PREAMBLE


% ------------------- main content ----------------------

% Slides for PHY981


% ----------------- title -------------------------

\thispagestyle{empty}

\begin{center}
{\LARGE\bf
\begin{spacing}{1.25}
How do develop a numerical project
\end{spacing}
}
\end{center}

% ----------------- author(s) -------------------------

\begin{center}
{\bf Morten Hjorth-Jensen, National Superconducting Cyclotron Laboratory and Department of Physics and Astronomy, Michigan State University, East Lansing, MI 48824, USA {\&} Department of Physics, University of Oslo, Oslo, Norway${}^{}$} \\ [0mm]
\end{center}

    \begin{center}
% List of all institutions:
\end{center}
    
% ----------------- end author(s) -------------------------

\begin{center} % date
May 18-22 2015
\end{center}

\vspace{1cm}


% !split
\subsection*{Some basic ingredients for a successful numerical project}

% --- begin paragraph admon ---
\paragraph{}
In when building up a numerical project there are several elements you should think of
\begin{enumerate}
  \item How to structure a code in terms of functions

  \item How to make a module

  \item How to read input data flexibly from the command line

  \item How to create graphical/web user interfaces

  \item How to write unit tests (test functions or doctests)

  \item How to refactor code in terms of classes (instead of functions only)

  \item How to conduct and automate large-scale numerical experiments

  \item How to write scientific reports in various formats ({\LaTeX}, HTML)
\end{enumerate}

\noindent
% --- end paragraph admon ---




% !split
\subsection*{Additional benefits: A structure approach to solving problems}

% --- begin paragraph admon ---
\paragraph{}
The conventions and techniques outlined here will save you a lot of time when you incrementally extend software over time from simpler to more complicated problems. In particular, you will benefit from many good habits:
\begin{enumerate}
\item New code is added in a modular fashion to a library (modules)

\item Programs are run through convenient user interfaces

\item It takes one quick command to let all your code undergo heavy testing 

\item Tedious manual work with running programs is automated,

\item Your scientific investigations are reproducible, scientific reports with top quality typesetting are produced both for paper and electronic devices.
\end{enumerate}

\noindent
% --- end paragraph admon ---





% !split
\subsection*{Analysis of project, Configuration Interaction theory}

% --- begin paragraph admon ---
\paragraph{}
\begin{minted}[fontsize=\fontsize{9pt}{9pt},linenos=false,mathescape,baselinestretch=1.0,fontfamily=tt,xleftmargin=7mm]{python}
from numpy import *
from sympy import *
from matplotlib.pyplot import *


g_array = linspace(-1, 1, 1001)
e1_array = []
e2_array = []

for g in g_array:
	H1 = matrix([[2-g , -g/2.,  -g/2., -g/2., -g/2.,     0], 
		        [-g/2.,   4-g,  -g/2., -g/2.,    0., -g/2.],
		        [-g/2., -g/2.,    6-g,     0, -g/2., -g/2.],
				[-g/2., -g/2.,      0,   6-g, -g/2., -g/2.],
				[-g/2.,     0,  -g/2., -g/2.,   8-g, -g/2.],
				[0    , -g/2.,  -g/2., -g/2., -g/2.,  10-g]]) 

	H2 = matrix([[2-g , -g/2.,  -g/2., -g/2., -g/2.], 
		        [-g/2.,   4-g,  -g/2., -g/2.,    0.],
		        [-g/2., -g/2.,    6-g,     0, -g/2.],
				[-g/2., -g/2.,      0,   6-g, -g/2.],
				[-g/2.,     0,  -g/2., -g/2.,   8-g]]) 

		

	u1, v1 = linalg.eig(H1)
	u2, v2 = linalg.eig(H2)

	if g == 1./2:
		print argmin(u1)

		for i in range(5):
			print " %.3f " % v2[i,0],



	e1_array.append(min(u1))
	e2_array.append(min(u2))


plot(g_array, e1_array, linewidth=2.0)
#plot(g_array, e2_array, linewidth=2.0)
plot(g_array, (2-g_array), linewidth=2.0)
grid()
xlabel(r"Strength of interaction, $g$", fontsize=16)
ylabel(r'Ground state energy', fontsize=16)
#axis([-1,1,-0.4,0.05])
legend(['FCI -- Exact', 'Reference energy'])
savefig("proj1_ref2.pdf")
show()
	

\end{minted}
% --- end paragraph admon ---





% !split
\subsection*{Analysis of project, Many-body perturbation theory}

% --- begin paragraph admon ---
\paragraph{}
\begin{minted}[fontsize=\fontsize{9pt}{9pt},linenos=false,mathescape,baselinestretch=1.0,fontfamily=tt,xleftmargin=7mm]{python}
from sympy import *
from pylab import *

below_fermi = (0,1,2,3)
above_fermi = (4,5,6,7)

states = [(1,1),(1,-1),(2,1),(2,-1),(3,1),(3,-1),(4,1),(4,-1)]
N = 8
g = Symbol('g')



def h0(p,q):
	if p == q:
		p1, s1 = states[p]
		return (p1 - 1)	
	else:
		return 0

def f(p,q):
	if p == q:
		return 0

	s = h0(p,q)
	for i in below_fermi:
		s += assym(p,i,q,i)
	return s


def assym(p,q,r,s):
	p1, s1 = states[p]
	p2, s2 = states[q]
	p3, s3 = states[r]
	p4, s4 = states[s]

	if p1 != p2 or p3 != p4:
		return 0
	if s1 == s2 or s3 == s4:
		return 0
	if s1 == s3 and s2 == s4:
		return -g/2.
	if s1 == s4 and s2 == s3:
		return g/2.

def eps(holes, particles):
	E = 0
	for h in holes:
		p, s = states[h]
		E += (p-1)
	for p in particles:
		p, s = states[p]
		E -= (p-1)
	return E


# Diagram 3
# s = 0 
# for a in above_fermi:
# 	for b in above_fermi:
# 		for c in above_fermi:
# 			for i in below_fermi:
# 				for j in below_fermi:
# 					for k in below_fermi:
# 						s += assym(i,j,a,b)*assym(a,c,j,k)*assym(b,k,c,i)/eps((i,j),(a,b))/eps((k,j),(a,c))
# print s


# ga = linspace(-1,1,101)
# corr2 = []
# corr3 = []
# corrx = []


# Diagram 1
s1 = 0
for a in above_fermi:
	for b in above_fermi:
		for i in below_fermi:
			for j in below_fermi:
				s1 += 0.25*assym(a,b,i,j)*assym(i,j,a,b)/eps((i,j),(a,b))

# Diagram 4
s4 = 0
for a in above_fermi:
	for b in above_fermi:
		for c in above_fermi:
			for d in above_fermi:
				for i in below_fermi:
					for j in below_fermi:
						s4 += 0.125*assym(i,j,a,b)*assym(a,b,c,d)*assym(c,d,i,j)/eps((i,j),(a,b))/eps((i,j),(c,d))

# Diagram 5
s5 = 0
for a in above_fermi:
	for b in above_fermi:
		for i in below_fermi:
			for j in below_fermi:
				for k in below_fermi:
					for l in below_fermi:
						s5 += 0.125*assym(i,j,a,b)*assym(k,l,i,j)*assym(a,b,k,l)/eps((i,j),(a,b))/eps((k,l),(a,b))

# Diagram 8 (simplified)
s8 = 0 
for a in above_fermi:
	for b in above_fermi:
		for i in below_fermi:
			for j in below_fermi:
				for k in below_fermi:
					s8 -= 0.5*assym(i,j,a,b)*assym(a,b,i,k)*f(k,j)/eps((i,j),(a,b))/eps((i,k),(a,b))

# Diagram 9 (simplified)
s9 = 0 
for a in above_fermi:
	for b in above_fermi:
		for c in above_fermi:
			for i in below_fermi:
				for j in below_fermi:
					s9 += 0.5*assym(i,j,a,b)*assym(a,c,i,j)*f(b,c)/eps((i,j),(a,b))/eps((i,j),(a,c))


print s1
print s4
print s5
print s8
print s9

s_5 =  -0.0291521990740741*g**4
s14 =  -0.0308883101851853*g**4
s34 =  0.0163049768518519*g**4
s36 =  -0.0145760995370371*g**4
s38 =  -0.0201099537037037*g**4
s39 =  0.0176938657407407*g**4

ga = linspace(-1,1,10001)
e1 = []
corr2 = []
corr3 = []
corr4 = []
for g_val in ga:
	H1 = matrix([[2-g_val , -g_val/2.,  -g_val/2., -g_val/2., -g_val/2.,     0], 
		        [-g_val/2.,   4-g_val,  -g_val/2., -g_val/2.,    0., -g_val/2.],
		        [-g_val/2., -g_val/2.,    6-g_val,     0, -g_val/2., -g_val/2.],
				[-g_val/2., -g_val/2.,      0,   6-g_val, -g_val/2., -g_val/2.],
				[-g_val/2.,     0,  -g_val/2., -g_val/2.,   8-g_val, -g_val/2.],
				[0    , -g_val/2.,  -g_val/2., -g_val/2., -g_val/2.,  10-g_val]]) 

	u1, v1 = linalg.eig(H1)
	e1.append(min(u1))

	corr2.append((s1).subs(g,g_val))
	corr3.append((s1+s4+s5).subs(g,g_val))
	corr4.append((s1+s4+s5+2*s_5+2*s14+2*s34+2*s36+s38+2*s39).subs(g,g_val))

exact = e1 - (2-ga)

plot(ga, exact, linewidth=2.0)
plot(ga, corr2, linewidth=2.0)
plot(ga, corr3, linewidth=2.0)
plot(ga, corr4, linewidth=2.0)
xlabel(r'Interaction strength, $g$', fontsize=16)
ylabel(r'Correlation energy', fontsize=16)
axis([-1,1,-0.5,0.05])
grid()
legend(["Exact", "2. order MPBT", "3. order MPBT", "4. order MPBT"], 'lower left')
savefig("pert_2.pdf")
show()


error1 = zeros(len(exact))
error2 = zeros(len(exact))
error3 = zeros(len(exact))

for i in range(len(exact)):
	error1[i] = abs(float(exact[i]-corr2[i]))
	error2[i] = abs(float(exact[i]-corr3[i]))
	error3[i] = abs(float(exact[i]-corr4[i]))

error1 = array(error1)
error2 = array(error2)
error3 = array(error3)
print type(error1)

plot(ga, log10(error1))
plot(ga, log10(error2))
plot(ga, log10(error3))
xlabel(r"Strength of interaction, $g$", fontsize=16)
ylabel(r"Error, $\log_{\rm 10}({\rm abs}({\rm error})$", fontsize=16)
legend(["2. order MPBT", "3. order MPBT", "4. order MPBT"], 'lower left')
grid()
savefig("logerror.pdf")
show()
\end{minted}
% --- end paragraph admon ---







% ------------------- end of main content ---------------


\printindex

\end{document}

