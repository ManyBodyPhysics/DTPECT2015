%%
%% Automatically generated file from DocOnce source
%% (https://github.com/hplgit/doconce/)
%%


%-------------------- begin preamble ----------------------

\documentclass[%
twoside,                 % oneside: electronic viewing, twoside: printing
final,                   % or draft (marks overfull hboxes, figures with paths)
10pt]{article}

\listfiles               % print all files needed to compile this document

\usepackage{relsize,makeidx,color,setspace,amsmath,amsfonts}
\usepackage[table]{xcolor}
\usepackage{bm,microtype}

\usepackage{fancyvrb} % packages needed for verbatim environments

\usepackage{minted}
\usemintedstyle{trac}

\usepackage[T1]{fontenc}
%\usepackage[latin1]{inputenc}
\usepackage[utf8]{inputenc}

\usepackage{lmodern}         % Latin Modern fonts derived from Computer Modern

% Hyperlinks in PDF:
\definecolor{linkcolor}{rgb}{0,0,0.4}
\usepackage[%
    colorlinks=true,
    linkcolor=linkcolor,
    urlcolor=linkcolor,
    citecolor=black,
    filecolor=black,
    %filecolor=blue,
    pdfmenubar=true,
    pdftoolbar=true,
    bookmarksdepth=3   % Uncomment (and tweak) for PDF bookmarks with more levels than the TOC
            ]{hyperref}
%\hyperbaseurl{}   % hyperlinks are relative to this root

\setcounter{tocdepth}{2}  % number chapter, section, subsection

\usepackage[framemethod=TikZ]{mdframed}

% --- begin definitions of admonition environments ---

% --- end of definitions of admonition environments ---

% prevent orhpans and widows
\clubpenalty = 10000
\widowpenalty = 10000

% --- end of standard preamble for documents ---


% insert custom LaTeX commands...

\raggedbottom
\makeindex

%-------------------- end preamble ----------------------

\begin{document}



% ------------------- main content ----------------------

% Slides for FYS-KJM4480


% ----------------- title -------------------------

\thispagestyle{empty}

\begin{center}
{\LARGE\bf
\begin{spacing}{1.25}
Quantum Mechanics of Many-Particle Systems FYS-KJM4480/9480
\end{spacing}
}
\end{center}

% ----------------- author(s) -------------------------

\begin{center}
{\bf Morten Hjorth-Jensen${}^{1, 2}$} \\ [0mm]
\end{center}

    \begin{center}
% List of all institutions:
\centerline{{\small ${}^1$Department of Physics, University of Oslo}}
\centerline{{\small ${}^2$Department of Physics and Astronomy and National Superconducting Cyclotron Laboratory, Michigan State University}}
\end{center}
    
% ----------------- end author(s) -------------------------

\begin{center} % date
Jan 20, 2015
\end{center}

\vspace{1cm}


% !split
\subsection*{Overview of week 34}

% --- begin paragraph admon ---
\paragraph{}
\begin{itemize}
\item Monday:

\item Presentation of topics to be covered and introduction to Many-Body physics (Lecture notes, Shavitt and Bartlett chapter 1, Raimes chapter 1 and Gross, Runge and Heinonen (GRH) chapter 1).

\item Tuesday:

\item Discussion of wave functions for fermions and bosons.

\item Calculations of expectation values and start defining second quantization

\item No exercises this week.
\end{itemize}

\noindent
% --- end paragraph admon ---



% !split
\subsection*{Overview of week 35}

% --- begin paragraph admon ---
\paragraph{}
\begin{itemize}
\item Monday:

\item Second quantization and representation of operators

\item Tuesday:

\item Second quantization and representation of operators

\item Exercises 1 and 2
\end{itemize}

\noindent
% --- end paragraph admon ---




% !split
\subsection*{Lectures, exercise sessions and syllabus}

% --- begin paragraph admon ---
\paragraph{}
\begin{itemize}
\item Lectures: Monday (10.15-12.00, room 394) and Tuesday (10.15-12.00, room 394)

\item Detailed lecture notes, all exercises presented and projects can be found at the homepage of the course.

\item Exercises: No allocated time (but a given time can be determined)

\item Weekly plans and all other information are on the official webpage.

\item Syllabus: Lecture notes, exercises and projects. Shavitt and Bartlett as main text, chapter 1-7 and 9-10. Gross, Runge and Heinonen chapters 1-10 and 14-27or  Raimes (chapter 1-3, and 5-11) are also good alternatives.
\end{itemize}

\noindent
% --- end paragraph admon ---



% !split
\subsection*{Quantum Many-particle Methods covered}

% --- begin paragraph admon ---
\paragraph{}
\begin{itemize}
\item Large-scale diagonalization (Iterative methods, Lanczo's method, dimensionalities $10^{10}$ states)

\item Coupled cluster theory, favoured method in quantum chemistry, molecular and atomic physics. Applications to ab initio calculations in nuclear physics as well for large nuclei.

\item Perturbative many-body methods 

\item Density functional theories/Mean-field theory and Hartree-Fock theory (covered partly also in FYS-MENA4111)

\item Monte-Carlo methods (Only in FYS4411, Computational quantum mechanics)

\item Green's function theories (depending on interest)

\item and other. The physics of the system hints at which many-body methods to use.
\end{itemize}

\noindent
% --- end paragraph admon ---



% !split
\subsection*{Plan for the semester}

% --- begin paragraph admon ---
\paragraph{Projects, deadlines (proposed) and oral exam.}
\begin{itemize}
\item Weekly exercises count $10\%$ of final mark (no exercises  first week)

\item Project 1, counts $25\%$: hand out October 6, handin October 17 (12pm)

\item Project 2, counts $25\%$: hand out November 10, handin November 21 (12pm)

\item Final oral exam in December 15-19, counts $40\%$ of final mark
\end{itemize}

\noindent
% --- end paragraph admon ---



% !split
\subsection*{Selected Texts and Many-body theory}

% --- begin paragraph admon ---
\paragraph{}
\begin{itemize}
\item Blaizot and Ripka, \emph{Quantum Theory of Finite systems}, MIT press 1986

\item Negele and Orland, \emph{Quantum Many-Particle Systems}, Addison-Wesley, 1987.

\item Fetter and Walecka, \emph{Quantum Theory of Many-Particle Systems}, McGraw-Hill, 1971.

\item Helgaker, Jorgensen and Olsen, \emph{Molecular Electronic Structure Theory}, Wiley, 2001.

\item Mattuck, \emph{Guide to Feynman Diagrams in the Many-Body Problem}, Dover, 1971.

\item Dickhoff and Van Neck, \emph{Many-Body Theory Exposed}, World Scientific, 2006.
\end{itemize}

\noindent
% --- end paragraph admon ---




% !split 
\subsection*{Definitions}

% --- begin paragraph admon ---
\paragraph{}
An operator is defined as $\hat{O}$ throughout. Unless otherwise specified the number of particles is
always $N$ and $d$ is the dimension of the system.  In nuclear physics
we normally define the total number of particles to be $A=N+Z$, where
$N$ is total number of neutrons and $Z$ the total number of
protons. In case of other baryons such isobars $\Delta$ or various
hyperons such as $\Lambda$ or $\Sigma$, one needs to add their
definitions.  Hereafter, $N$ is reserved for the total number of
particles, unless otherwise specificied.
% --- end paragraph admon ---



% !split
\subsection*{Definitions}

% --- begin paragraph admon ---
\paragraph{}
The quantum numbers of a single-particle state in coordinate space are
defined by the variable 
\[
x=({\bf r},\sigma), 
\]
where 
\[
{\bf r}\in {\mathbb{R}}^{d},
\]
with $d=1,2,3$ represents the spatial coordinates and $\sigma$ is the eigenspin of the particle. For fermions with eigenspin $1/2$ this means that
\[
 x\in {\mathbb{R}}^{d}\oplus (\frac{1}{2}),
\]
and the integral
\[
\int dx = \sum_{\sigma}\int d^dr = \sum_{\sigma}\int d{\bf r},
\]
and
\[
\int d^Nx= \int dx_1\int dx_2\dots\int dx_N.
\]
% --- end paragraph admon ---




% !split
\subsection*{Definitions}

% --- begin paragraph admon ---
\paragraph{}
The quantum mechanical wave function of a given state with quantum numbers $\lambda$ (encompassing all quantum numbers needed to specify the system), ignoring time, is
\[
\Psi_{\lambda}=\Psi_{\lambda}(x_1,x_2,\dots,x_N),
\]
with $x_i=({\bf r}_i,\sigma_i)$ and the projection of $\sigma_i$ takes the values
$\{-1/2,+1/2\}$ for particles with spin $1/2$. 
We will hereafter always refer to $\Psi_{\lambda}$ as the exact wave function, and if the ground state is not degenerate we label it as 
\[
\Psi_0=\Psi_0(x_1,x_2,\dots,x_N).
\]
% --- end paragraph admon ---




% !split
\subsection*{Definitions}

% --- begin paragraph admon ---
\paragraph{}
Since the solution $\Psi_{\lambda}$ seldomly can be found in closed form, approximations are sought. In this text we define an approximative wave function or an ansatz to the exact wave function as 
\[
\Phi_{\lambda}=\Phi_{\lambda}(x_1,x_2,\dots,x_N),
\]
with
\[
\Phi_0=\Phi_0(x_1,x_2,\dots,x_N),
\]
being the ansatz to the ground state.
% --- end paragraph admon ---




% !split
\subsection*{Definitions}

% --- begin paragraph admon ---
\paragraph{}
The wave function $\Psi_{\lambda}$ is sought in the Hilbert space of either symmetric or anti-symmetric $N$-body functions, namely
\[
\Psi_{\lambda}\in {\cal H}_N:= {\cal H}_1\oplus{\cal H}_1\oplus\dots\oplus{\cal H}_1,
\]
where the single-particle Hilbert space $\hat{H}_1$ is the space of square integrable functions over
$\in {\mathbb{R}}^{d}\oplus (\sigma)$
resulting in
\[
{\cal H}_1:= L^2(\mathbb{R}^{d}\oplus (\sigma)).
\]
% --- end paragraph admon ---





% !split
\subsection*{Definitions}

% --- begin paragraph admon ---
\paragraph{}
Our Hamiltonian is invariant under the permutation (interchange) of two particles.
Since we deal with fermions however, the total wave function is antisymmetric.
Let $\hat{P}$ be an operator which interchanges two particles.
Due to the symmetries we have ascribed to our Hamiltonian, this operator commutes with the total Hamiltonian,
\[
[\hat{H},\hat{P}] = 0,
\]
meaning that $\Psi_{\lambda}(x_1, x_2, \dots , x_N)$ is an eigenfunction of 
$\hat{P}$ as well, that is
\[
\hat{P}_{ij}\Psi_{\lambda}(x_1, x_2, \dots,x_i,\dots,x_j,\dots,x_N)=
\beta\Psi_{\lambda}(x_1, x_2, \dots,x_j,\dots,x_i,\dots,x_N),
\]
where $\beta$ is the eigenvalue of $\hat{P}$. We have introduced the suffix $ij$ in order to indicate that we permute particles $i$ and $j$.
The Pauli principle tells us that the total wave function for a system of fermions
has to be antisymmetric, resulting in the eigenvalue $\beta = -1$.
% --- end paragraph admon ---




% !split
\subsection*{Definitions and notations}

% --- begin paragraph admon ---
\paragraph{}
The Schrodinger equation reads 
\begin{equation}
\hat{H}(x_1, x_2, \dots , x_N) \Psi_{\lambda}(x_1, x_2, \dots , x_N) = 
E_\lambda  \Psi_\lambda(x_1, x_2, \dots , x_N), \label{eq:basicSE1}
\end{equation}
where the vector $x_i$ represents the coordinates (spatial and spin) of particle $i$, $\lambda$ stands  for all the quantum
numbers needed to classify a given $N$-particle state and $\Psi_{\lambda}$ is the pertaining eigenfunction.  Throughout this course,
$\Psi$ refers to the exact eigenfunction, unless otherwise stated.
% --- end paragraph admon ---



% !split
\subsection*{Definitions and notations}

% --- begin paragraph admon ---
\paragraph{}
We write the Hamilton operator, or Hamiltonian,  in a generic way 
\[
	\hat{H} = \hat{T} + \hat{V} 
\]
where $\hat{T}$  represents the kinetic energy of the system
\[
	\hat{T} = \sum_{i=1}^N \frac{\mathbf{p}_i^2}{2m_i} = \sum_{i=1}^N \left( -\frac{\hbar^2}{2m_i} \mathbf{\nabla_i}^2 \right) =
		\sum_{i=1}^N t(x_i)
\]
while the operator $\hat{V}$ for the potential energy is given by
\begin{equation}
	\hat{V} = \sum_{i=1}^N \hat{u}_{\mathrm{ext}}(x_i) + \sum_{ji=1}^N v(x_i,x_j)+\sum_{ijk=1}^Nv(x_i,x_j,x_k)+\dots
\label{eq:firstv}
\end{equation}
Hereafter we use natural units, viz.~$\hbar=c=e=1$, with $e$ the elementary charge and $c$ the speed of light. This means that momenta and masses
have dimension energy.
% --- end paragraph admon ---




% !split
\subsection*{Definitions and notations}

% --- begin paragraph admon ---
\paragraph{}
If one does quantum chemistry, after having introduced the  Born-Oppenheimer approximation which effectively freezes out the nucleonic degrees of freedom, the Hamiltonian for $N=n_e$ electrons takes the following form 
\[
  \hat{H} = \sum_{i=1}^{n_e} t(x_i) - \sum_{i=1}^{n_e} k\frac{Z}{r_i} + \sum_{i < j}^{n_e} \frac{k}{r_{ij}},
\]
with $k=1.44$ eVnm
% --- end paragraph admon ---




% !split
\subsection*{Definitions and notations}

% --- begin paragraph admon ---
\paragraph{}
We can rewrite this as
\begin{equation}
    \hat{H} = \hat{H}_0 + \hat{H}_I 
    = \sum_{i=1}^{n_e}\hat{h}_0(x_i) + \sum_{i < j}^{n_e}\frac{1}{r_{ij}},
\label{H1H2}
\end{equation}
where  we have defined 
\[
r_{ij}=| {\bf r}_i-{\bf r}_j|,
\]
 and
\begin{equation}
  \hat{h}_0(x_i) =  \hat{t}(x_i) - \frac{Z}{x_i}.
\label{hi}
\end{equation}
The first term of Eq.~(\ref{H1H2}), $H_0$, is the sum of the $N$
\emph{one-body} Hamiltonians $\hat{h}_0$. Each individual
Hamiltonian $\hat{h}_0$ contains the kinetic energy operator of an
electron and its potential energy due to the attraction of the
nucleus. The second term, $H_I$, is the sum of the $n_e(n_e-1)/2$
two-body interactions between each pair of electrons. Note that the double sum carries a restriction $i < j$.
% --- end paragraph admon ---




% !split
\subsection*{Definitions and notations}

% --- begin paragraph admon ---
\paragraph{}
The potential energy term due to the attraction of the nucleus defines the onebody field $u_i=u_{\mathrm{ext}}(x_i)$ of Eq.~(\ref{eq:firstv}).
We have moved this term into the $\hat{H}_0$ part of the Hamiltonian, instead of keeping  it in $\hat{V}$ as in  Eq.~(\ref{eq:firstv}).
The reason is that we will hereafter treat $\hat{H}_0$ as our non-interacting  Hamiltonian. For a many-body wavefunction $\Phi_{\lambda}$ defined by an  
appropriate single-particle basis, we may solve exactly the non-interacting eigenvalue problem 
\[
\hat{H}_0\Phi_{\lambda}= w_{\lambda}\Phi_{\lambda},
\]
with $w_{\lambda}$ being the non-interacting energy. This energy is defined by the sum over single-particle energies to be defined below.
For atoms the single-particle energies could be the hydrogen-like single-particle energies corrected for the charge $Z$. For nuclei and quantum
dots, these energies could be given by the harmonic oscillator in three and two dimensions, respectively.
% --- end paragraph admon ---



% !split
\subsection*{Definitions and notations}

% --- begin paragraph admon ---
\paragraph{}
We will assume that the interacting part of the Hamiltonian
can be approximated by a two-body interaction.
This means that our Hamiltonian is written as 
\begin{equation}
    \hat{H} = \hat{H}_0 + \hat{H}_I 
    = \sum_{i=1}^N \hat{h}_0(x_i) + \sum_{i < j}^N V(r_{ij}),
\label{Hnuclei}
\end{equation}
with 
\begin{equation}
  H_0=\sum_{i=1}^N \hat{h}_0(x_i) =  \sum_{i=1}^N\left(\hat{t}(x_i) + \hat{u}_{\mathrm{ext}}(x_i)\right).
\label{hinuclei}
\end{equation}
The onebody part $u_{\mathrm{ext}}(x_i)$ is normally approximated by a harmonic oscillator potential or the Coulomb interaction an electron feels from the nucleus. However, other potentials are fully possible, such as 
one derived from the self-consistent solution of the Hartree-Fock equations.
% --- end paragraph admon ---




% !split
\subsection*{Definitions and notations}

% --- begin paragraph admon ---
\paragraph{}
Our Hamiltonian is invariant under the permutation (interchange) of two particles. % (exercise here, prove it)
Since we deal with fermions however, the total wave function is antisymmetric.
Let $\hat{P}$ be an operator which interchanges two particles.
Due to the symmetries we have ascribed to our Hamiltonian, this operator commutes with the total Hamiltonian,
\[
[\hat{H},\hat{P}] = 0,
 \]
meaning that $\Psi_{\lambda}(x_1, x_2, \dots , x_N)$ is an eigenfunction of 
$\hat{P}$ as well, that is
\[
\hat{P}_{ij}\Psi_{\lambda}(x_1, x_2, \dots,x_i,\dots,x_j,\dots,x_N)=
\beta\Psi_{\lambda}(x_1, x_2, \dots,x_i,\dots,x_j,\dots,x_N),
\]
where $\beta$ is the eigenvalue of $\hat{P}$. We have introduced the suffix $ij$ in order to indicate that we permute particles $i$ and $j$.
The Pauli principle tells us that the total wave function for a system of fermions
has to be antisymmetric, resulting in the eigenvalue $\beta = -1$.
% --- end paragraph admon ---



% !split
\subsection*{Definitions and notations}

% --- begin paragraph admon ---
\paragraph{}
In our case we assume that  we can approximate the exact eigenfunction with a Slater determinant
\begin{equation}
   \Phi(x_1, x_2,\dots ,x_N,\alpha,\beta,\dots, \sigma)=\frac{1}{\sqrt{N!}}
\left| \begin{array}{ccccc} \psi_{\alpha}(x_1)& \psi_{\alpha}(x_2)& \dots & \dots & \psi_{\alpha}(x_N)\\
                            \psi_{\beta}(x_1)&\psi_{\beta}(x_2)& \dots & \dots & \psi_{\beta}(x_N)\\  
                            \dots & \dots & \dots & \dots & \dots \\
                            \dots & \dots & \dots & \dots & \dots \\
                     \psi_{\sigma}(x_1)&\psi_{\sigma}(x_2)& \dots & \dots & \psi_{\sigma}(x_N)\end{array} \right|, \label{eq:HartreeFockDet}
\end{equation}
where  $x_i$  stand for the coordinates and spin values of a particle $i$ and $\alpha,\beta,\dots, \gamma$ 
are quantum numbers needed to describe remaining quantum numbers.
% --- end paragraph admon ---



% !split
\subsection*{Definitions and notations}

% --- begin paragraph admon ---
\paragraph{}
The single-particle function $\psi_{\alpha}(x_i)$  are eigenfunctions of the onebody
Hamiltonian $h_i$, that is
\[
\hat{h}_0(x_i)=\hat{t}(x_i) + \hat{u}_{\mathrm{ext}}(x_i),
\]
with eigenvalues 
\[
\hat{h}_0(x_i) \psi_{\alpha}(x_i)=\left(\hat{t}(x_i) + \hat{u}_{\mathrm{ext}}(x_i)\right)\psi_{\alpha}(x_i)=\varepsilon_{\alpha}\psi_{\alpha}(x_i).
\]
The energies $\varepsilon_{\alpha}$ are the so-called non-interacting single-particle energies, or unperturbed energies. 
The total energy is in this case the sum over all  single-particle energies, if no two-body or more complicated
many-body interactions are present.
% --- end paragraph admon ---



% !split
\subsection*{Definitions and notations}

% --- begin paragraph admon ---
\paragraph{}
Let us denote the ground state energy by $E_0$. According to the
variational principle we have
\[
  E_0 \le E[\Phi] = \int \Phi^*\hat{H}\Phi d\mathbf{\tau}
\]
where $\Phi$ is a trial function which we assume to be normalized
\[
  \int \Phi^*\Phi d\mathbf{\tau} = 1,
\]
where we have used the shorthand $d\mathbf{\tau}=d\mathbf{r}_1d\mathbf{r}_2\dots d\mathbf{r}_N$.
% --- end paragraph admon ---



% !split
\subsection*{Definitions and notations}

% --- begin paragraph admon ---
\paragraph{}
In the Hartree-Fock method the trial function is the Slater
determinant of Eq.~(\ref{eq:HartreeFockDet}) which can be rewritten as 
\[
  \Phi(x_1,x_2,\dots,x_N,\alpha,\beta,\dots,\nu) = \frac{1}{\sqrt{N!}}\sum_{P} (-)^P\hat{P}\psi_{\alpha}(x_1)
    \psi_{\beta}(x_2)\dots\psi_{\nu}(x_N)=\sqrt{N!}\hat{A}\Phi_H,
\]
where we have introduced the antisymmetrization operator $\hat{A}$ defined by the 
summation over all possible permutations of two particles.
% --- end paragraph admon ---



% !split
\subsection*{Definitions and notations}

% --- begin paragraph admon ---
\paragraph{}
It is defined as
\begin{equation}
  \hat{A} = \frac{1}{N!}\sum_{p} (-)^p\hat{P},
\label{antiSymmetryOperator}
\end{equation}
with $p$ standing for the number of permutations. We have introduced for later use the so-called
Hartree-function, defined by the simple product of all possible single-particle functions
\[
  \Phi_H(x_1,x_2,\dots,x_N,\alpha,\beta,\dots,\nu) =
  \psi_{\alpha}(x_1)
    \psi_{\beta}(x_2)\dots\psi_{\nu}(x_N).
\]
% --- end paragraph admon ---



% !split
\subsection*{Definitions and notations}

% --- begin paragraph admon ---
\paragraph{}
Both $\hat{H}_0$ and $\hat{H}_I$ are invariant under all possible permutations of any two particles
and hence commute with $\hat{A}$
\begin{equation}
  [H_0,\hat{A}] = [H_I,\hat{A}] = 0. \label{commutionAntiSym}
\end{equation}
Furthermore, $\hat{A}$ satisfies
\begin{equation}
  \hat{A}^2 = \hat{A},  \label{AntiSymSquared}
\end{equation}
since every permutation of the Slater
determinant reproduces it.
% --- end paragraph admon ---



% !split
\subsection*{Definitions and notations}

% --- begin paragraph admon ---
\paragraph{}
The expectation value of $\hat{H}_0$ 
\[
  \int \Phi^*\hat{H}_0\Phi d\mathbf{\tau} 
  = N! \int \Phi_H^*\hat{A}\hat{H}_0\hat{A}\Phi_H d\mathbf{\tau}
\]
is readily reduced to
\[
  \int \Phi^*\hat{H}_0\Phi d\mathbf{\tau} 
  = N! \int \Phi_H^*\hat{H}_0\hat{A}\Phi_H d\mathbf{\tau},
\]
where we have used Eqs.~(\ref{commutionAntiSym}) and
(\ref{AntiSymSquared}). The next step is to replace the antisymmetrization
operator by its definition and to
replace $\hat{H}_0$ with the sum of one-body operators
\[
  \int \Phi^*\hat{H}_0\Phi  d\mathbf{\tau}
  = \sum_{i=1}^N \sum_{p} (-)^p\int 
  \Phi_H^*\hat{h}_0\hat{P}\Phi_H d\mathbf{\tau}.
\]
% --- end paragraph admon ---



% !split
\subsection*{Definitions and notations}

% --- begin paragraph admon ---
\paragraph{}
The integral vanishes if two or more particles are permuted in only one
of the Hartree-functions $\Phi_H$ because the individual single-particle wave functions are
orthogonal. We obtain then
\[
  \int \Phi^*\hat{H}_0\Phi  d\mathbf{\tau}= \sum_{i=1}^N \int \Phi_H^*\hat{h}_0\Phi_H  d\mathbf{\tau}.
\]
Orthogonality of the single-particle functions allows us to further simplify the integral, and we
arrive at the following expression for the expectation values of the
sum of one-body Hamiltonians 
\begin{equation}
  \int \Phi^*\hat{H}_0\Phi  d\mathbf{\tau}
  = \sum_{\mu=1}^N \int \psi_{\mu}^*(\mathbf{r})\hat{h}_0\psi_{\mu}(\mathbf{r})
  d\mathbf{r}.
  \label{H1Expectation}
\end{equation}
% --- end paragraph admon ---



% !split
\subsection*{Definitions and notations}

% --- begin paragraph admon ---
\paragraph{}
We introduce the following shorthand for the above integral
\[
\langle \mu | \hat{h}_0 | \mu \rangle = \int \psi_{\mu}^*(\mathbf{r})\hat{h}_0\psi_{\mu}(\mathbf{r}),
\]
and rewrite Eq.~(\ref{H1Expectation}) as
\begin{equation}
  \int \Phi^*\hat{H}_0\Phi  d\mathbf{\tau}
  = \sum_{\mu=1}^N \langle \mu | \hat{h}_0 | \mu \rangle.
  \label{H1Expectation1}
\end{equation}
% --- end paragraph admon ---



% !split
\subsection*{Definitions and notations}

% --- begin paragraph admon ---
\paragraph{}
The expectation value of the two-body part of the Hamiltonian is obtained in a
similar manner. We have
\[
  \int \Phi^*\hat{H}_I\Phi d\mathbf{\tau} 
  = N! \int \Phi_H^*\hat{A}\hat{H}_I\hat{A}\Phi_H d\mathbf{\tau},
\]
which reduces to
\[
 \int \Phi^*\hat{H}_I\Phi d\mathbf{\tau} 
  = \sum_{i\le j=1}^N \sum_{p} (-)^p\int 
  \Phi_H^*V(r_{ij})\hat{P}\Phi_H d\mathbf{\tau},
\]
by following the same arguments as for the one-body
Hamiltonian.
% --- end paragraph admon ---



% !split
\subsection*{Definitions and notations}

% --- begin paragraph admon ---
\paragraph{}
Because of the dependence on the inter-particle distance $r_{ij}$,  permutations of
any two particles no longer vanish, and we get
\[
  \int \Phi^*\hat{H}_I\Phi d\mathbf{\tau} 
  = \sum_{i < j=1}^N \int  
  \Phi_H^*V(r_{ij})(1-P_{ij})\Phi_H d\mathbf{\tau}.
\]
where $P_{ij}$ is the permutation operator that interchanges
particle $i$ and particle $j$. Again we use the assumption that the single-particle wave functions
are orthogonal.
% --- end paragraph admon ---




% !split
\subsection*{Definitions and notations}

% --- begin paragraph admon ---
\paragraph{}
We obtain
\begin{equation}
\begin{split}
  \int \Phi^*\hat{H}_I\Phi d\mathbf{\tau} 
  = \frac{1}{2}\sum_{\mu=1}^N\sum_{\nu=1}^N
    &\left[ \int \psi_{\mu}^*(x_i)\psi_{\nu}^*(x_j)V(r_{ij})\psi_{\mu}(x_i)\psi_{\nu}(x_j)
    dx_ix_j \right.\\
  &\left.
  - \int \psi_{\mu}^*(x_i)\psi_{\nu}^*(x_j)
  V(r_{ij})\psi_{\nu}(x_i)\psi_{\mu}(x_j)
  dx_ix_j
  \right]. \label{H2Expectation}
\end{split}
\end{equation}
The first term is the so-called direct term. It is frequently also called the  Hartree term, 
while the second is due to the Pauli principle and is called
the exchange term or just the Fock term.
The factor  $1/2$ is introduced because we now run over
all pairs twice.
% --- end paragraph admon ---



% !split
\subsection*{Definitions and notations}

% --- begin paragraph admon ---
\paragraph{}
The last equation allows us to  introduce some further definitions.  
The single-particle wave functions $\psi_{\mu}(x)$, defined by the quantum numbers $\mu$ and $x$
are defined as the overlap 
\[
   \psi_{\alpha}(x)  = \langle x | \alpha \rangle .
\]
% --- end paragraph admon ---



% !split
\subsection*{Definitions and notations}

% --- begin paragraph admon ---
\paragraph{}
We introduce the following shorthands for the above two integrals
\[
\langle \mu\nu|\hat{v}|\mu\nu\rangle =  \int \psi_{\mu}^*(x_i)\psi_{\nu}^*(x_j)V(r_{ij})\psi_{\mu}(x_i)\psi_{\nu}(x_j)
    dx_ix_j,
\]
and
\[
\langle \mu\nu|\hat{v}|\nu\mu\rangle = \int \psi_{\mu}^*(x_i)\psi_{\nu}^*(x_j)
  V(r_{ij})\psi_{\nu}(x_i)\psi_{\mu}(x_j)
  dx_ix_j.  
\]
% --- end paragraph admon ---



% !split
\subsection*{Definitions and notations}

% --- begin paragraph admon ---
\paragraph{}
The direct and exchange matrix elements can be  brought together if we define the antisymmetrized matrix element
\[
\langle \mu\nu|\hat{v}|\mu\nu\rangle_{\mathrm{AS}}= \langle \mu\nu|\hat{v}|\mu\nu\rangle-\langle \mu\nu|\hat{v}|\nu\mu\rangle,
\]
or for a general matrix element  
\[
\langle \mu\nu|\hat{v}|\sigma\tau\rangle_{\mathrm{AS}}= \langle \mu\nu|\hat{v}|\sigma\tau\rangle-\langle \mu\nu|\hat{v}|\tau\sigma\rangle.
\]
It has the symmetry property
\[
\langle \mu\nu|\hat{v}|\sigma\tau\rangle_{\mathrm{AS}}= -\langle \mu\nu|\hat{v}|\tau\sigma\rangle_{\mathrm{AS}}=-\langle \nu\mu|\hat{v}|\sigma\tau\rangle_{\mathrm{AS}}.
\]
% --- end paragraph admon ---




% !split
\subsection*{Definitions and notations}

% --- begin paragraph admon ---
\paragraph{}
The antisymmetric matrix element is also hermitian, implying 
\[
\langle \mu\nu|\hat{v}|\sigma\tau\rangle_{\mathrm{AS}}= \langle \sigma\tau|\hat{v}|\mu\nu\rangle_{\mathrm{AS}}.
\]
With these notations we rewrite Eq.~(\ref{H2Expectation}) as 
\begin{equation}
  \int \Phi^*\hat{H}_I\Phi d\mathbf{\tau} 
  = \frac{1}{2}\sum_{\mu=1}^N\sum_{\nu=1}^N \langle \mu\nu|\hat{v}|\mu\nu\rangle_{\mathrm{AS}}.
\label{H2Expectation2}
\end{equation}
% --- end paragraph admon ---



% !split
\subsection*{Definitions and notations}

% --- begin paragraph admon ---
\paragraph{}
Combining Eqs.~(\ref{H1Expectation1}) and
(\ref{H2Expectation2}) we obtain the energy functional 
\begin{equation}
  E[\Phi] 
  = \sum_{\mu=1}^N \langle \mu | \hat{h}_0 | \mu \rangle +
  \frac{1}{2}\sum_{{\mu}=1}^N\sum_{{\nu}=1}^N \langle \mu\nu|\hat{v}|\mu\nu\rangle_{\mathrm{AS}}.
\label{FunctionalEPhi}
\end{equation}
which we will use as our starting point for the Hartree-Fock calculations later in this course.
% --- end paragraph admon ---



% !split
\subsection*{Exercises week 35}

% --- begin paragraph admon ---
\paragraph{Exercise 1.}
Consider the Slater determinant
\[
\Phi_{\lambda}^{AS}(x_{1}x_{2}\dots x_{N};\alpha_{1}\alpha_{2}\dots\alpha_{N})
=\frac{1}{\sqrt{N!}}\sum_{p}(-)^{p}P\prod_{i=1}^{N}\psi_{\alpha_{i}}(x_{i}).
\]
where $P$ is an operator which permutes the coordinates of two particles. We have assumed here that the 
number of particles is the same as the number of available single-particle states, represented by the
greek letters $\alpha_{1}\alpha_{2}\dots\alpha_{N}$.
\begin{itemize}
\item Write  out $\Phi^{AS}$ for $N=3$.  

\item Show that
\end{itemize}

\noindent
\[
\int dx_{1}dx_{2}\dots dx_{N}\left\vert
\Phi_{\lambda}^{AS}(x_{1}x_{2}\dots x_{N};\alpha_{1}\alpha_{2}\dots\alpha_{N})
\right\vert^{2} = 1.
\]
\begin{itemize}
\item Define a general onebody operator $\hat{F} = \sum_{i}^N\hat{f}(x_{i})$ and a general 
\end{itemize}

\noindent
twobody operator $\hat{G}=\sum_{i>j}^N\hat{g}(x_{i},x_{j})$
with $g$ being invariant under the interchange of the coordinates of particles $i$ and $j$.
Calculate the matrix elements for a two-particle Slater determinant
\[
\langle\Phi_{\alpha_{1}\alpha_{2}}^{AS}|\hat{F}|\Phi_{\alpha_{1}\alpha_{2}}^{AS} \rangle,
\]
and
\[
\langle \Phi_{\alpha_{1}\alpha_{2}}^{AS}|\hat{G}|\Phi_{\alpha_{1}\alpha_{2}}^{AS}\rangle.
\]
Explain the short-hand notation for the Slater determinant.
Which properties do you expect these operators to have in addition to an eventual permutation
symmetry?
\begin{itemize}
\item Compute the corresponding matrix elements for $N$ particles which can occupy $N$ single particle states.
\end{itemize}

\noindent
% --- end paragraph admon ---




% !split
\subsection*{Exercises week 35}

% --- begin paragraph admon ---
\paragraph{Exercise 2.}
We will now consider a simple three-level problem, depicted in the figure below. 
The single-particle states are labelled by the quantum number $p$ and can accomodate up to two single particles, 
viz., every single-particle state 
is doubly degenerate (you could think of this as one state having spin up and the other spin down). 
We let the spacing between the doubly degenerate single-particle states be constant, with value $d$.  The first state
has energy $d$. There are only three available single-particle states, $p=1$, $p=2$ and $p=3$, as illustrated
in the figure. 
\begin{itemize}
\item How many two-particle Slater determinants can we construct in this space? 

\item We limit ourselves to a system with only the two lowest single-particle orbits and two particles, $p=1$ and $p=2$.
\end{itemize}

\noindent
We assume that we can write the Hamiltonian as
\[
       \hat{H}=\hat{H}_0+\hat{H}_I,
\]
and that the onebody part of the Hamiltonian with single-particle operator $\hat{h}_0$ has the property
\[
\hat{h}_0\psi_{p\sigma} = p\times d \psi_{p\sigma},
\]
where we have added a spin quantum number $\sigma$. 
We assume also that the only two-particle states that can exist are those where two particles are in the 
same state $p$, as shown by the two possibilities to the left in the figure.
The two-particle matrix elements of $\hat{H}_I$ have all a constant value, $-g$.
Show then that the Hamiltonian matrix can be written as 
\[
\left(\begin{array}{cc}2d-g &-g \\
-g &4d-g \end{array}\right),
\]
and find the eigenvalues and eigenvectors.  What is mixing of the state with two particles in $p=2$ 
to the wave function with two-particles in $p=1$? Discuss your results in terms of a linear combination
of Slater determinants. 
\begin{itemize}
\item Add the possibility that the two particles can be in the state with $p=3$ as well and find the Hamiltonian
\end{itemize}

\noindent
matrix, the eigenvalues and the eigenvectors. We still insist that we only have two-particle states composed of two particles being in the same
level $p$. You can diagonalize numerically your $3\times 3$ matrix.

This simple model catches several birds with a stone. It demonstrates how we can build linear combinations
of Slater determinants and interpret these as different admixtures to a given state. It represents also the way we are going to interpret these contributions.  The two-particle states above $p=1$ will be interpreted as 
excitations from the ground state configuration, $p=1$ here.  The reliability of this ansatz for the ground state, 
with two particles in $p=1$,
depends on the strength of the interaction $g$ and the single-particle spacing $d$.
Finally, this model is a simple schematic ansatz for studies of pairing correlations and thereby superfluidity/superconductivity  
in fermionic systems.
% --- end paragraph admon ---



% !split
\subsection*{Python code for solving exercise 2}

% --- begin paragraph admon ---
\paragraph{}
\begin{minted}[fontsize=\fontsize{9pt}{9pt},linenos=false,mathescape,baselinestretch=1.0,fontfamily=tt,xleftmargin=7mm]{python}
from numpy import *
from numpy.linalg import eig
from pylab import *
from matplotlib import rc, rcParams
import matplotlib.units as units
import matplotlib.ticker as ticker
import sys
import os
# set tickers etc
rc('text',usetex=True)
rc('font',**{'family':'serif','serif':['Energies']})
title(r'{\bf Energies}', fontsize=20)     


def exercise2(g):
    z = ones((3,3), dtype = float)*-.5*g
    z[0,0] = 2-g
    z[1,1] = 4-g
    z[2,2] = 6-g
    n1 = range(3)
    n2 = range(3)
    n2.reverse()
    z[n2, n1] = 0
    return z
print exercise2(1)

N = 1000
Z = zeros((N,3), dtype = float)
G = linspace(1,10,N)
for i in range(N):
    Z[i] = eig(exercise2(G[i]))[0]

plot(G,Z[:,0])

#Z2 = zeros((N,5), dtype = float)
#for i in range(N):
#    Z2[i] = eig(exercise2(G[i])[0:2,0:2])[0]
#plot(G,Z)

savefig('exercise2.pdf', format='pdf')
# Draw the plot to screen
show()
\end{minted}
% --- end paragraph admon ---

    



% ------------------- end of main content ---------------


\printindex

\end{document}

