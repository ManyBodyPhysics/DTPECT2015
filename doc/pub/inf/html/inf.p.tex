%%
%% Automatically generated file from DocOnce source
%% (https://github.com/hplgit/doconce/)
%%
%%
% #ifdef PTEX2TEX_EXPLANATION
%%
%% The file follows the ptex2tex extended LaTeX format, see
%% ptex2tex: http://code.google.com/p/ptex2tex/
%%
%% Run
%%      ptex2tex myfile
%% or
%%      doconce ptex2tex myfile
%%
%% to turn myfile.p.tex into an ordinary LaTeX file myfile.tex.
%% (The ptex2tex program: http://code.google.com/p/ptex2tex)
%% Many preprocess options can be added to ptex2tex or doconce ptex2tex
%%
%%      ptex2tex -DMINTED myfile
%%      doconce ptex2tex myfile envir=minted
%%
%% ptex2tex will typeset code environments according to a global or local
%% .ptex2tex.cfg configure file. doconce ptex2tex will typeset code
%% according to options on the command line (just type doconce ptex2tex to
%% see examples). If doconce ptex2tex has envir=minted, it enables the
%% minted style without needing -DMINTED.
% #endif

% #define PREAMBLE

% #ifdef PREAMBLE
%-------------------- begin preamble ----------------------

\documentclass[%
twoside,                 % oneside: electronic viewing, twoside: printing
final,                   % or draft (marks overfull hboxes, figures with paths)
10pt]{article}

\listfiles               % print all files needed to compile this document

\usepackage{relsize,makeidx,color,setspace,amsmath,amsfonts}
\usepackage[table]{xcolor}
\usepackage{bm,microtype}

\usepackage{ptex2tex}
% #ifdef MINTED
\usepackage{minted}
\usemintedstyle{default}
% #endif

\usepackage[T1]{fontenc}
%\usepackage[latin1]{inputenc}
\usepackage{ucs}
\usepackage[utf8x]{inputenc}

\usepackage{lmodern}         % Latin Modern fonts derived from Computer Modern

% Hyperlinks in PDF:
\definecolor{linkcolor}{rgb}{0,0,0.4}
\usepackage{hyperref}
\hypersetup{
    breaklinks=true,
    colorlinks=true,
    linkcolor=linkcolor,
    urlcolor=linkcolor,
    citecolor=black,
    filecolor=black,
    %filecolor=blue,
    pdfmenubar=true,
    pdftoolbar=true,
    bookmarksdepth=3   % Uncomment (and tweak) for PDF bookmarks with more levels than the TOC
    }
%\hyperbaseurl{}   % hyperlinks are relative to this root

\setcounter{tocdepth}{2}  % number chapter, section, subsection

% prevent orhpans and widows
\clubpenalty = 10000
\widowpenalty = 10000

\newenvironment{doconceexercise}{}{}
\newcounter{doconceexercisecounter}


% ------ header in subexercises ------
%\newcommand{\subex}[1]{\paragraph{#1}}
%\newcommand{\subex}[1]{\par\vspace{1.7mm}\noindent{\bf #1}\ \ }
\makeatletter
% 1.5ex is the spacing above the header, 0.5em the spacing after subex title
\newcommand\subex{\@startsection{paragraph}{4}{\z@}%
                  {1.5ex\@plus1ex \@minus.2ex}%
                  {-0.5em}%
                  {\normalfont\normalsize\bfseries}}
\makeatother


% --- end of standard preamble for documents ---


% insert custom LaTeX commands...

\raggedbottom
\makeindex

%-------------------- end preamble ----------------------

\begin{document}

% endif for #ifdef PREAMBLE
% #endif


% ------------------- main content ----------------------



% ----------------- title -------------------------

\thispagestyle{empty}

\begin{center}
{\LARGE\bf
\begin{spacing}{1.25}
Infinite matter, from the electron gas to nuclear matter
\end{spacing}
}
\end{center}

% ----------------- author(s) -------------------------

\begin{center}
{\bf Morten Hjorth-Jensen, National Superconducting Cyclotron Laboratory and Department of Physics and Astronomy, Michigan State University, East Lansing, MI 48824, USA {\&} Department of Physics, University of Oslo, Oslo, Norway${}^{}$} \\ [0mm]
\end{center}

    \begin{center}
% List of all institutions:
\end{center}
    
% ----------------- end author(s) -------------------------

\begin{center} % date
May 18-22 2015
\end{center}

\vspace{1cm}


\subsection{Introduction to studies of infinite matter}


Studies of infinite nuclear matter play an important role  in nuclear physics. The aim of this part of the lectures is to provide the necessary ingredients for perfoming studies of neutron star matter (or matter in $\beta$-equilibrium) and symmetric nuclear matter. We start however with the electron gas in two and three dimensions for both historical and pedagogical reasons. Since there are several benchmark calculations for the electron gas, this small detour will allow us to establish the necessary formalism. Thereafter we will study infinite nuclear matter 
\begin{itemize}
\item at the Hartree-Fock with realistic nuclear forces and

\item using many-body methods like coupled-cluster theory or in-medium SRG as discussed in our previous sections.
\end{itemize}

\noindent
\subsection{The infinite electron gas}

The electron gas is perhaps the only realistic model of a 
system of many interacting particles that allows for a solution
of the Hartree-Fock equations on a closed form. Furthermore, to first order in the interaction, one can also
compute on a closed form the total energy and several other properties of a many-particle systems. 
The model gives a very good approximation to the properties of valence electrons in metals.
The assumptions are

\begin{itemize}
 \item System of electrons that is not influenced by external forces except by an attraction provided by a uniform background of ions. These ions give rise to a uniform background charge. The ions are stationary.

 \item The system as a whole is neutral.

 \item We assume we have $N_e$ electrons in a cubic box of length $L$ and volume $\Omega=L^3$. This volume contains also a uniform distribution of positive charge with density $N_ee/\Omega$. 
\end{itemize}

\noindent
The homogeneous electron gas is one of the few examples of a system of many
interacting particles that allows for a solution of the mean-field
Hartree-Fock equations on a closed form.  To first order in the
electron-electron interaction, this applies to ground state properties
like the energy and its pertinent equation of state as well.  The
homogeneus electron gas is a system of electrons that is not
influenced by external forces except by an attraction provided by a
uniform background of ions. These ions give rise to a uniform
background charge.  The ions are stationary and the system as a whole
is neutral.
Irrespective of this simplicity, this system, in both two and
three-dimensions, has eluded a proper description of correlations in
terms of various first principle methods, except perhaps for quantum
Monte Carlo methods. In particular, the diffusion Monte Carlo
calculations of \href{{http://journals.aps.org/prl/abstract/10.1103/PhysRevLett.45.566}}{Ceperley} 
and \href{{http://journals.aps.org/prb/abstract/10.1103/PhysRevB.39.5005}}{Ceperley and Tanatar} 
are presently still considered as the
best possible benchmarks for the two- and three-dimensional electron
gas. 



The electron gas, in 
two or three dimensions is thus interesting as a test-bed for 
electron-electron correlations. The three-dimensional 
electron gas is particularly important as a cornerstone 
of the local-density approximation in density-functional 
theory. In the physical world, systems 
similar to the three-dimensional electron gas can be 
found in, for example, alkali metals and doped 
semiconductors. Two-dimensional electron fluids are 
observed on metal and liquid-helium surfaces, as well as 
at metal-oxide-semiconductor interfaces. However, the Coulomb 
interaction has an infinite range, and therefore 
long-range correlations play an essential role in the
electron gas. 




At low densities, the electrons become 
localized and form a lattice. This so-called Wigner 
crystallization is a direct consequence 
of the long-ranged repulsive interaction. At higher
densities, the electron gas is better described as a
liquid.
When using, for example, Monte Carlo methods the electron gas must be approximated 
by a finite system. The long-range Coulomb interaction 
in the electron gas causes additional finite-size effects  that are not
present in other infinite systems like nuclear matter or neutron star matter.
This poses additional challenges to many-body methods when applied 
to the electron gas.





\subsection{The infinite electron gas as a homogenous system}

This is a homogeneous system and the one-particle wave functions are given by plane wave functions normalized to a volume $\Omega$ 
for a box with length $L$ (the limit $L\rightarrow \infty$ is to be taken after we have computed various expectation values)
\[
\psi_{\mathbf{k}\sigma}(\mathbf{r})= \frac{1}{\sqrt{\Omega}}\exp{(i\mathbf{kr})}\xi_{\sigma}
\]
where $\mathbf{k}$ is the wave number and  $\xi_{\sigma}$ is a spin function for either spin up or down
\[ 
\xi_{\sigma=+1/2}=\left(\begin{array}{c} 1 \\ 0 \end{array}\right) \hspace{0.5cm}
\xi_{\sigma=-1/2}=\left(\begin{array}{c} 0 \\ 1 \end{array}\right).
\]




\subsection{Periodic boundary conditions}


We assume that we have periodic boundary conditions which limit the allowed wave numbers to
\[
k_i=\frac{2\pi n_i}{L}\hspace{0.5cm} i=x,y,z \hspace{0.5cm} n_i=0,\pm 1,\pm 2, \dots
\]
We assume first that the electrons interact via a central, symmetric and translationally invariant
interaction  $V(r_{12})$ with
$r_{12}=|\mathbf{r}_1-\mathbf{r}_2|$.  The interaction is spin independent.

The total Hamiltonian consists then of kinetic and potential energy
\[
\hat{H} = \hat{T}+\hat{V}.
\]
The operator for the kinetic energy can be written as
\[
\hat{T}=\sum_{\mathbf{k}\sigma}\frac{\hbar^2k^2}{2m}a_{\mathbf{k}\sigma}^{\dagger}a_{\mathbf{k}\sigma}.
\]



\subsection{Defining the Hamiltonian operator}

The Hamiltonian operator is given by
\[
\hat{H}=\hat{H}_{el}+\hat{H}_{b}+\hat{H}_{el-b},
\]
with the electronic part
\[
\hat{H}_{el}=\sum_{i=1}^N\frac{p_i^2}{2m}+\frac{e^2}{2}\sum_{i\ne j}\frac{e^{-\mu |\mathbf{r}_i-\mathbf{r}_j|}}{|\mathbf{r}_i-\mathbf{r}_j|},
\]
where we have introduced an explicit convergence factor
(the limit $\mu\rightarrow 0$ is performed after having calculated the various integrals).
Correspondingly, we have
\[
\hat{H}_{b}=\frac{e^2}{2}\int\int d\mathbf{r}d\mathbf{r}'\frac{n(\mathbf{r})n(\mathbf{r}')e^{-\mu |\mathbf{r}-\mathbf{r}'|}}{|\mathbf{r}-\mathbf{r}'|},
\]
which is the energy contribution from the positive background charge with density
$n(\mathbf{r})=N/\Omega$. Finally,
\[
\hat{H}_{el-b}=-\frac{e^2}{2}\sum_{i=1}^N\int d\mathbf{r}\frac{n(\mathbf{r})e^{-\mu |\mathbf{r}-\mathbf{x}_i|}}{|\mathbf{r}-\mathbf{x}_i|},
\]
is the interaction between the electrons and the positive background.



\subsection{Single-particle Hartree-Fock energy}

In the first exercise below we show that the Hartree-Fock energy can be written as 
\[
\varepsilon_{k}^{HF}=\frac{\hbar^{2}k^{2}}{2m_e}-\frac{e^{2}}
{\Omega^{2}}\sum_{k'\leq
k_{F}}\int d\mathbf{r}e^{i(\mathbf{k}'-\mathbf{k})\mathbf{r}}\int
d\mathbf{r'}\frac{e^{i(\mathbf{k}-\mathbf{k}')\mathbf{r}'}}
{\vert\mathbf{r}-\mathbf{r}'\vert}
\]
resulting in
\[
\varepsilon_{k}^{HF}=\frac{\hbar^{2}k^{2}}{2m_e}-\frac{e^{2}
k_{F}}{2\pi}
\left[
2+\frac{k_{F}^{2}-k^{2}}{kk_{F}}ln\left\vert\frac{k+k_{F}}
{k-k_{F}}\right\vert
\right]
\]



The previous result can be rewritten in terms of the density
\[
n= \frac{k_F^3}{3\pi^2}=\frac{3}{4\pi r_s^3},
\]
where $n=N_e/\Omega$, $N_e$ being the number of electrons, and $r_s$ is the radius of a sphere which represents the volum per conducting electron.  
It can be convenient to use the Bohr radius $a_0=\hbar^2/e^2m_e$.
For most metals we have a relation $r_s/a_0\sim 2-6$.  The quantity $r_s$ is dimensionless.


In the second exercise below  we find that
the total energy
$E_0/N_e=\langle\Phi_{0}|\hat{H}|\Phi_{0}\rangle/N_e$ for
for this system to first order in the interaction is given as 
\[
E_0/N_e=\frac{e^2}{2a_0}\left[\frac{2.21}{r_s^2}-\frac{0.916}{r_s}\right].
\]











% --- begin exercise ---
\begin{doconceexercise}
\refstepcounter{doconceexercisecounter}

\subsection*{Exercise \thedoconceexercisecounter: Hartree-Fock single-particle solution for the electron gas}


The electron gas model allows closed form solutions for quantities like the 
single-particle Hartree-Fock energy.  The latter quantity is given by the following expression
\[
\varepsilon_{k}^{HF}=\frac{\hbar^{2}k^{2}}{2m}-\frac{e^{2}}
{V^{2}}\sum_{k'\leq
k_{F}}\int d\mathbf{r}e^{i(\mathbf{k'}-\mathbf{k})\mathbf{r}}\int
d\mathbf{r}'\frac{e^{i(\mathbf{k}-\mathbf{k'})\mathbf{r}'}}
{\vert\mathbf{r}-\mathbf{r'}\vert}
\]


\subex{a)}
Show first that
\[
\varepsilon_{k}^{HF}=\frac{\hbar^{2}k^{2}}{2m}-\frac{e^{2}
k_{F}}{2\pi}
\left[
2+\frac{k_{F}^{2}-k^{2}}{kk_{F}}ln\left\vert\frac{k+k_{F}}
{k-k_{F}}\right\vert
\right]
\]

% --- begin hint in exercise ---

\paragraph{Hint.}
Hint: Introduce the convergence factor 
$e^{-\mu\vert\mathbf{r}-\mathbf{r}'\vert}$
in the potential and use  $\sum_{\mathbf{k}}\rightarrow
\frac{V}{(2\pi)^{3}}\int d\mathbf{k}$

% --- end hint in exercise ---


% --- begin solution of exercise ---
\paragraph{Solution.}
We want to show that, given the Hartree-Fock equation for the electron gas
\[
\varepsilon_{k}^{HF}=\frac{\hbar^{2}k^{2}}{2m}-\frac{e^{2}}
{V^{2}}\sum_{p\leq
k_{F}}\int d\mathbf{r}\exp{(i(\mathbf{p}-\mathbf{k})\mathbf{r})}\int
d\mathbf{r}'\frac{\exp{(i(\mathbf{k}-\mathbf{p})\mathbf{r}'})}
{\vert\mathbf{r}-\mathbf{r'}\vert}
\]
the single-particle energy can be written as
\[
\varepsilon_{k}^{HF}=\frac{\hbar^{2}k^{2}}{2m}-\frac{e^{2}
k_{F}}{2\pi}
\left[
2+\frac{k_{F}^{2}-k^{2}}{kk_{F}}ln\left\vert\frac{k+k_{F}}
{k-k_{F}}\right\vert
\right].
\]

We introduce the convergence factor 
$e^{-\mu\vert\mathbf{r}-\mathbf{r}'\vert}$
in the potential and use  $\sum_{\mathbf{k}}\rightarrow
\frac{V}{(2\pi)^{3}}\int d\mathbf{k}$. We can then rewrite the integral as 
\begin{align}
\frac{e^{2}}
{V^{2}}\sum_{k'\leq
k_{F}}\int d\mathbf{r}\exp{(i(\mathbf{k'}-\mathbf{k})\mathbf{r})}\int
d\mathbf{r}'\frac{\exp{(i(\mathbf{k}-\mathbf{p})\mathbf{r}'})}
{\vert\mathbf{r}-\mathbf{r'}\vert}= & \\
\frac{e^{2}}{V (2\pi)^3}  \int d\mathbf{r}\int
\frac{d\mathbf{r}'}{\vert\mathbf{r}-\mathbf{r'}\vert}\exp{(-i\mathbf{k}(\mathbf{r}-\mathbf{r}'))}\int d\mathbf{p}\exp{(i\mathbf{p}(\mathbf{r}-\mathbf{r}'))},
\end{align}
and introducing the abovementioned convergence factor we have
\begin{align}
\lim_{\mu \to 0}\frac{e^{2}}{V (2\pi)^3}  \int d\mathbf{r}\int d\mathbf{r}'\frac{\exp{(-\mu\vert\mathbf{r}-\mathbf{r}'\vert})}{\vert\mathbf{r}-\mathbf{r'}\vert}\int d\mathbf{p}\exp{(i(\mathbf{p}-\mathbf{k})(\mathbf{r}-\mathbf{r}'))}.
\end{align}


With a change variables to $\mathbf{x} = \mathbf{r}-\mathbf{r}'$ and $\mathbf{y}=\mathbf{r}'$ we rewrite the last integral as
\[
\lim_{\mu \to 0}\frac{e^{2}}{V (2\pi)^3}  \int d\mathbf{p}\int d\mathbf{y}\int d\mathbf{x}\exp{(i(\mathbf{p}-\mathbf{k})\mathbf{x})}\frac{\exp{(-\mu\vert\mathbf{x}\vert})}{\vert\mathbf{x}\vert}.
\]
The integration over $\mathbf{x}$ can be performed using spherical coordinates, resulting in (with $x=\vert \mathbf{x}\vert$)
\[
\int d\mathbf{x}\exp{(i(\mathbf{p}-\mathbf{k})\mathbf{x})}\frac{\exp{(-\mu\vert\mathbf{x}\vert})}{\vert\mathbf{x}\vert}=\int x^2 dx d\phi d\cos{(\theta)}\exp{(i(\mathbf{p}-\mathbf{k})x\cos{(\theta))}}\frac{\exp{(-\mu x)}}{x}.
\]


We obtain
\begin{align}
4\pi \int dx \frac{ \sin{(\vert \mathbf{p}-\mathbf{k}\vert)x} }{\vert \mathbf{p}-\mathbf{k}\vert}{\exp{(-\mu x)}}= \frac{4\pi}{\mu^2+\vert \mathbf{p}-\mathbf{k}\vert^2}.
\end{align}
This results gives us 
\begin{align}
\lim_{\mu \to 0}\frac{e^{2}}{V (2\pi)^3}  \int d\mathbf{p}\int d\mathbf{y}\frac{4\pi}{\mu^2+\vert \mathbf{p}-\mathbf{k}\vert^2}=\lim_{\mu \to 0}\frac{e^{2}}{ 2\pi^2}  \int d\mathbf{p}\frac{1}{\mu^2+\vert \mathbf{p}-\mathbf{k}\vert^2},
\end{align}
where we have used that the integrand on the left-hand side does not depend on $\mathbf{y}$ and that $\int d\mathbf{y}=V$.

Introducing spherical coordinates we can rewrite the integral as 
\begin{align}
\lim_{\mu \to 0}\frac{e^{2}}{ 2\pi^2}  \int d\mathbf{p}\frac{1}{\mu^2+\vert \mathbf{p}-\mathbf{k}\vert^2}=\frac{e^{2}}{ 2\pi^2}  \int d\mathbf{p}\frac{1}{\vert \mathbf{p}-\mathbf{k}\vert^2}=& \\
\frac{e^{2}}{\pi}  \int_0^{k_F} p^2dp\int_0^{\pi} d\theta\cos{(\theta)}\frac{1}{p^2+k^2-2pk\cos{(\theta)}},
\end{align}
and with the change of variables $\cos{(\theta)}=u$ we have 
\[
\frac{e^{2}}{\pi}  \int_0^{k_F} p^2dp\int_{0}^{\pi} d\theta\cos{(\theta)}\frac{1}{p^2+k^2-2pk\cos{(\theta)}}=\frac{e^{2}}{\pi}  \int_0^{k_F} p^2dp\int_{-1}^{1} du\frac{1}{p^2+k^2-2pku},
\]
which gives
\[
\frac{e^{2}}{k\pi}  \int_0^{k_F} pdp\left\{ln(\vert p+k\vert)-ln(\vert p-k\vert)\right\}.
\]

Introducing new variables $x=p+k$ and $y=p-k$, we obtain after some straightforward reordering of the integral
\[
\frac{e^{2}}{k\pi}\left[
kk_F+\frac{k_{F}^{2}-k^{2}}{kk_{F}}ln\left\vert\frac{k+k_{F}}
{k-k_{F}}\right\vert
\right],
\]
which gives the abovementioned expression for the single-particle energy.

% --- end solution of exercise ---

\subex{b)}
Rewrite the above result as a function of the density
\[
n= \frac{k_F^3}{3\pi^2}=\frac{3}{4\pi r_s^3},
\]
where $n=N/V$, $N$ being the number of particles, and $r_s$ is the radius of a sphere which represents the volum per conducting electron.


% --- begin solution of exercise ---
\paragraph{Solution.}
Introducing the dimensionless quantity $x=k/k_F$ and the function
\[
F(x) = \frac{1}{2}+\frac{1-x^2}{4x}\ln{\left\vert \frac{1+x}{1-x}\right\vert},
\]
we can rewrite the single-particle Hartree-Fock energy as 
\[
\varepsilon_{k}^{HF}=\frac{\hbar^{2}k^{2}}{2m}-\frac{2e^{2}
k_{F}}{\pi}F(k/k_F),
\]
and dividing by the non-interacting contribution at the Fermi level, 
\[
\varepsilon_{0}^{F}=\frac{\hbar^{2}k_F^{2}}{2m},
\]
we have
\[
\frac{\varepsilon_{k}^{HF} }{\varepsilon_{0}^{F}}=x^2-\frac{e^2m}{\hbar^2 k_F\pi}F(x)=x^2-\frac{4}{\pi k_Fa_0}F(x),
\]
where $a_0=0.0529$ nm is the Bohr radius, setting thereby a natural length scale. 


By introducing the radius $r_s$ of a sphere whose volume is the volume occupied by each electron, we can rewrite the previous equation in terms of $r_s$ using that the electron density $n=N/V$
\[
n=\frac{k_F^3}{3\pi^2} = \frac{3}{4\pi r_s^3},
\]
we have (with $k_F=1.92/r_s$,
\[
\frac{\varepsilon_{k}^{HF} }{\varepsilon_{0}^{F}}=x^2-\frac{e^2m}{\hbar^2 k_F\pi}F(x)=x^2-\frac{r_s}{a_0}0.663F(x),
\]
with $r_s \sim 2-6$ for most metals.

% --- end solution of exercise ---

It can be convenient to use the Bohr radius $a_0=\hbar^2/e^2m$.
For most metals we have a relation $r_s/a_0\sim 2-6$.

\subex{c)}
Make a plot of the free electron energy and the Hartree-Fock energy and discuss the behavior around the Fermi surface. Extract also   the Hartree-Fock band width $\Delta\varepsilon^{HF}$ defined as
\[ 
\Delta\varepsilon^{HF}=\varepsilon_{k_{F}}^{HF}-
\varepsilon_{0}^{HF}.
\]
Compare this results with the corresponding one for a free electron and comment your results. How large is the contribution due to the exchange term in the Hartree-Fock equation?


% --- begin solution of exercise ---
\paragraph{Solution.}
We can now define the so-called band gap, that is the scatter between the maximal and the minimal value of the electrons in the conductance band of a metal (up to the Fermi level). 
For $x=1$ and $r_s/a_0=4$ we have 
\[
\frac{\varepsilon_{k=k_F}^{HF} }{\varepsilon_{0}^{F}} = -0.326,
\]
and for $x=0$ we have
\[
\frac{\varepsilon_{k=0}^{HF} }{\varepsilon_{0}^{F}} = -2.652,
\]
which results in a gap at the Fermi level of 
\[
\Delta \varepsilon^{HF} = \frac{\varepsilon_{k=k_F}^{HF} }{\varepsilon_{0}^{F}}-\frac{\varepsilon_{k=0}^{HF} }{\varepsilon_{0}^{F}} = 2.326.
\]
This quantity measures the deviation from the $k=0$ single-particle energy and the energy at the Fermi level.
The general result is 
\[
\Delta \varepsilon^{HF} = 1+\frac{r_s}{a_0}0.663.
\]

The following python code produces a plot of the electron energy for a free electron (only kinetic energy) and 
for the Hartree-Fock solution. We have chosen here a ratio $r_s/a_0=4$ and the equations are plotted as funtions
of $k/f_F$. 
\bpycod
import numpy as np
from math import log
from  matplotlib import pyplot as plt
from matplotlib import rc, rcParams
import matplotlib.units as units
import matplotlib.ticker as ticker
rc('text',usetex=True)
rc('font',**{'family':'serif','serif':['Hartree-Fock energy']})
font = {'family' : 'serif',
        'color'  : 'darkred',
        'weight' : 'normal',
        'size'   : 16,
        }

N = 100
x = np.linspace(0.0, 2.0,N)
F = 0.5+np.log(abs((1.0+x)/(1.0-x)))*(1.0-x*x)*0.25/x
y = x*x -4.0*0.663*F

plt.plot(x, y, 'b-')
plt.plot(x, x*x, 'r-')
plt.title(r'{\bf Hartree-Fock single-particle energy for electron gas}', fontsize=20)     
plt.text(3, -40, r'Parameters: $r_s/a_0=4$', fontdict=font)
plt.xlabel(r'$k/k_F$',fontsize=20)
plt.ylabel(r'$\varepsilon_k^{HF}/\varepsilon_0^F$',fontsize=20)
# Tweak spacing to prevent clipping of ylabel
plt.subplots_adjust(left=0.15)
plt.savefig('hartreefockspelgas.pdf', format='pdf')
plt.show()
\epycod
From the plot we notice that the exchange term increases considerably the band gap
compared with the non-interacting gas of electrons.

% --- end solution of exercise ---
We will now define a quantity called the effective mass.
For $\vert\mathbf{k}\vert$ near $k_{F}$, we can Taylor expand the Hartree-Fock energy as  
\[
\varepsilon_{k}^{HF}=\varepsilon_{k_{F}}^{HF}+
\left(\frac{\partial\varepsilon_{k}^{HF}}{\partial k}\right)_{k_{F}}(k-k_{F})+\dots
\]
If we compare the latter with the corresponding expressiyon for the non-interacting system
\[
\varepsilon_{k}^{(0)}=\frac{\hbar^{2}k^{2}_{F}}{2m}+
\frac{\hbar^{2}k_{F}}{m}\left(k-k_{F}\right)+\dots ,
\]
we can define the so-called effective Hartree-Fock mass as
\[
m_{HF}^{*}\equiv\hbar^{2}k_{F}\left(
\frac{\partial\varepsilon_{k}^{HF}}
{\partial k}\right)_{k_{F}}^{-1}
\]
\subex{d)}
Compute $m_{HF}^{*}$ and comment your results.

\subex{e)}
Show that the level density (the number of single-electron states per unit energy) can be written as
\[
n(\varepsilon)=\frac{Vk^{2}}{2\pi^{2}}\left(
\frac{\partial\varepsilon}{\partial k}\right)^{-1}
\]
Calculate $n(\varepsilon_{F}^{HF})$ and comment the results.




\end{doconceexercise}
% --- end exercise ---




% --- begin exercise ---
\begin{doconceexercise}
\refstepcounter{doconceexercisecounter}

\subsection*{Exercise \thedoconceexercisecounter: Hartree-Fock ground state energy for the  electron gas in three dimensions}


We consider a system of electrons in infinite matter, the so-called electron gas. This is a homogeneous system and the one-particle states are given by plane wave function normalized to a volume $\Omega$ 
for a box with length $L$ (the limit $L\rightarrow \infty$ is to be taken after we have computed various expectation values)
\[
\psi_{\mathbf{k}\sigma}(\mathbf{r})= \frac{1}{\sqrt{\Omega}}\exp{(i\mathbf{kr})}\xi_{\sigma}
\]
where $\mathbf{k}$ is the wave number and  $\xi_{\sigma}$ is a spin function for either spin up or down
\[ 
\xi_{\sigma=+1/2}=\left(\begin{array}{c} 1 \\ 0 \end{array}\right) \hspace{0.5cm}
\xi_{\sigma=-1/2}=\left(\begin{array}{c} 0 \\ 1 \end{array}\right).
\]
We assume that we have periodic boundary conditions which limit the allowed wave numbers to
\[
k_i=\frac{2\pi n_i}{L}\hspace{0.5cm} i=x,y,z \hspace{0.5cm} n_i=0,\pm 1,\pm 2, \dots
\]
We assume first that the particles interact via a central, symmetric and translationally invariant
interaction  $V(r_{12})$ with
$r_{12}=|\mathbf{r}_1-\mathbf{r}_2|$.  The interaction is spin independent.

The total Hamiltonian consists then of kinetic and potential energy
\[
\hat{H} = \hat{T}+\hat{V}.
\]


\subex{a)}
Show that the operator for the kinetic energy can be written as
\[
\hat{T}=\sum_{\mathbf{k}\sigma}\frac{\hbar^2k^2}{2m}a_{\mathbf{k}\sigma}^{\dagger}a_{\mathbf{k}\sigma}.
\]
Find also the number operator $\hat{N}$ and find a corresponding expression for the interaction
$\hat{V}$ expressed with creation and annihilation operators.   The expression for the interaction
has to be written in  $k$ space, even though $V$ depends only on the relative distance. It means that you ned to set up the Fourier transform $\langle \mathbf{k}_i\mathbf{k}_j| V | \mathbf{k}_m\mathbf{k}_n\rangle$.

The Hamiltonian operator is given by
\[
\hat{H}=\hat{H}_{el}+\hat{H}_{b}+\hat{H}_{el-b},
\]
with the electronic part
\[
\hat{H}_{el}=\sum_{i=1}^N\frac{p_i^2}{2m}+\frac{e^2}{2}\sum_{i\ne j}\frac{e^{-\mu |\mathbf{r}_i-\mathbf{r}_j|}}{|\mathbf{r}_i-\mathbf{r}_j|},
\]
where we have introduced an explicit convergence factor
(the limit $\mu\rightarrow 0$ is performed after having calculated the various integrals).
Correspondingly, we have
\[
\hat{H}_{b}=\frac{e^2}{2}\int\int d\mathbf{r}d\mathbf{r}'\frac{n(\mathbf{r})n(\mathbf{r}')e^{-\mu |\mathbf{r}-\mathbf{r}'|}}{|\mathbf{r}-\mathbf{r}'|},
\]
which is the energy contribution from the positive background charge with density
$n(\mathbf{r})=N/\Omega$. Finally,
\[
\hat{H}_{el-b}=-\frac{e^2}{2}\sum_{i=1}^N\int d\mathbf{r}\frac{n(\mathbf{r})e^{-\mu |\mathbf{r}-\mathbf{x}_i|}}{|\mathbf{r}-\mathbf{x}_i|},
\]
is the interaction between the electrons and the positive background.
\subex{b)}
Show that
\[
\hat{H}_{b}=\frac{e^2}{2}\frac{N^2}{\Omega}\frac{4\pi}{\mu^2},
\]
and
\[
\hat{H}_{el-b}=-e^2\frac{N^2}{\Omega}\frac{4\pi}{\mu^2}.
\]

\subex{c)}
Show thereafter that the final Hamiltonian can be written as 
\[
H=H_{0}+H_{I},
\]
with
\[
H_{0}={\displaystyle\sum_{\mathbf{k}\sigma}}
\frac{\hbar^{2}k^{2}}{2m}a_{\mathbf{k}\sigma}^{\dagger}
a_{\mathbf{k}\sigma},
\]
and
\[
H_{I}=\frac{e^{2}}{2\Omega}{\displaystyle\sum_{\sigma_{1}\sigma_{2}}}{\displaystyle\sum_{\mathbf{q}\neq 0,\mathbf{k},\mathbf{p}}}\frac{4\pi}{q^{2}}
a_{\mathbf{k}+\mathbf{q},\sigma_{1}}^{\dagger}
a_{\mathbf{p}-\mathbf{q},\sigma_{2}}^{\dagger}
a_{\mathbf{p}\sigma_{2}}a_{\mathbf{k}\sigma_{1}}.
\] 

\subex{d)}
Calculate $E_0/N=\langle \Phi_{0}\vert H\vert \Phi_{0}\rangle/N$ for for this system to first order in the interaction. Show that, by using
\[
\rho= \frac{k_F^3}{3\pi^2}=\frac{3}{4\pi r_0^3},
\]
with $\rho=N/\Omega$, $r_0$
being the radius of a sphere representing the volume an electron occupies 
and the Bohr radius $a_0=\hbar^2/e^2m$, 
that the energy per electron can be written as 
\[
E_0/N=\frac{e^2}{2a_0}\left[\frac{2.21}{r_s^2}-\frac{0.916}{r_s}\right].
\]
Here we have defined
$r_s=r_0/a_0$ to be a dimensionless quantity.

\subex{e)}
Plot your results. Why is this system stable?
Calculate thermodynamical quantities like the pressure, given by
\[
P=-\left(\frac{\partial E}{\partial \Omega}\right)_N,
\]
and the bulk modulus
\[
B=-\Omega\left(\frac{\partial P}{\partial \Omega}\right)_N,
\]
and comment your results.

We have to show first  that
\[
\hat{H}_{b}=\frac{e^2}{2}\frac{N_e^2}{\Omega}\frac{4\pi}{\mu^2},
\]
and
\[
\hat{H}_{el-b}=-e^2\frac{N_e^2}{\Omega}\frac{4\pi}{\mu^2}.
\]


And then that the final Hamiltonian can be written as 
\[
H=H_{0}+H_{I},
\]
with
\[
H_{0}={\displaystyle\sum_{\mathbf{k}\sigma}}
\frac{\hbar^{2}k^{2}}{2m_e}a_{\mathbf{k}\sigma}^{\dagger}
a_{\mathbf{k}\sigma},
\]
and
\[
H_{I}=\frac{e^{2}}{2\Omega}{\displaystyle\sum_{\sigma_{1}
\sigma_{2}}}{\displaystyle
\sum_{\mathbf{q}\neq 0,\mathbf{k},\mathbf{p}}}\frac{4\pi}{q^{2}}
a_{\mathbf{k}+\mathbf{q},\sigma_{1}}^{\dagger}
a_{\mathbf{p}-\mathbf{q},\sigma_{2}}^{\dagger}
a_{\mathbf{p}\sigma_{2}}a_{\mathbf{k}\sigma_{1}}.
\] 

Let us now calculate the following part of the Hamiltonian
\[ \hat H_b = \frac{e^2}{2} \iint \frac{n(\mathbf{r}) n(\mathbf{r}')e^{-\mu|\mathbf{r} - \mathbf{r}'|}}{|\mathbf{r} - \mathbf{r}'|} d\mathbf{r} d\mathbf{r}' , 
\]
where $n(\mathbf{r}) = N_e/\Omega$, the density of the positive background charge. We define $\mathbf{r}_{12} = \mathbf{r} - \mathbf{r}'$, resulting in $d\mathbf{r}_{12} = d\mathbf{r}$, and allowing us to rewrite the integral as
\[ 
\hat H_b = \frac{e^2 N_e^2}{2\Omega^2} \iint \frac{e^{-\mu |\mathbf{r}_{12}|}}{|\mathbf{r}_{12}|} d\mathbf{r}_{12} d\mathbf{r}' = \frac{e^2 N_e^2}{2\Omega} \int \frac{e^{-\mu |\mathbf{r}_{12}|}}{|\mathbf{r}_{12}|} d\mathbf{r}_{12} . 
\]
Here we have used that $\int \! \mathbf{r} = \Omega$. We change to spherical coordinates and the lack of angle 
dependencies yields a factor $4\pi$, resulting in
\[ 
\hat H_b = \frac{4\pi e^2 N_e^2}{2\Omega} \int_0^\infty re^{-\mu r} \, \mathrm{d} r . 
\]
Solving by partial integration
\[ \int_0^\infty re^{-\mu r} \, \mathrm{d} r = \left[ -\frac{r}{\mu} e^{-\mu r} \right]_0^\infty + \frac{1}{\mu} \int_0^\infty e^{-\mu r} \, \mathrm{d} r
= \frac{1}{\mu} \left[ - \frac{1}{\mu} e^{-\mu r} \right]_0^\infty = \frac{1}{\mu^2}, 
\]
gives
\[
\hat{H}_b = \frac{e^2}{2} \frac{N_e^2}{\Omega} \frac{4\pi}{\mu^2} .
\]
The next term is 
\[ 
\hat H_{el-b} = -e^2 \sum_{i = 1}^N \int \frac{n(\mathbf{r}) e^{-\mu |\mathbf{r} - \mathbf{x}_i|}}{|\mathbf{r} - \mathbf{x}_i|} \mathbf{r} . 
\]
Inserting  $n(\mathbf{r})$ and changing variables in the same way as in the previous integral $\mathbf{y} = \mathbf{r} - \mathbf{x}_i$, we get $\mathrm{d}^3 \mathbf{y} = \mathrm{d}^3 \mathbf{r}$. This gives
\begin{align}
\hat H_{el-b} = -\frac{e^2 N_e}{\Omega} \sum_{i = 1^N} \int \frac{e^{-\mu |\mathbf{y}|}}{|\mathbf{y}|} \, \mathrm{d}^3 \mathbf{y}
=  -\frac{4\pi e^2 N_e}{\Omega} \sum_{i = 1}^N \int_0^\infty y e^{-\mu y} \mathrm{d} y. 
\end{align}
We have already seen this  type of integral. The answer is 
\[ 
\hat H_{el-b} = -\frac{4\pi e^2 N_e}{\Omega} \sum_{i = 1}^N \frac{1}{\mu^2}, 
\]
which gives
\[
\hat H_{el-b} = -e^2 \frac{N_e^2}{\Omega} \frac{4\pi}{\mu^2} .
\]

Finally, we need to evaluate $\hat H_{el}$. This term reads
\[ 
\hat H_{el} = \sum_{i=1}^{N_e} \frac{\hat{\vec p}_i^2}{2m_e} + \frac{e^2}{2} \sum_{i \neq j} \frac{e^{-\mu |\mathbf{r}_i - \mathbf{r}_j|}}{\mathbf{r}_i - \mathbf{r}_j} . 
\]
The last term represents the repulsion between two electrons. It is a central symmetric interaction
and is translationally invariant. The potential is given by the expression
\[ 
v(|\mathbf{r}|) = e^2 \frac{e^{\mu|\mathbf{r}|}}{|\mathbf{r}|}, 
\]
which we derived in connection with the single-particle Hartree-Fock derivation.

More material will be added here!





\end{doconceexercise}
% --- end exercise ---


\subsection{Preparing the ground for numerical calculations; kinetic energy and Ewald term}

The kinetic energy operator is
\begin{align}
  \hat{H}_{\text{kin}} = -\frac{\hbar^{2}}{2m}\sum_{i=1}^{N}\nabla_{i}^{2},
\end{align}
where the sum is taken over all particles in the finite
box. The Ewald electron-electron interaction operator 
can be written as 
\begin{align}
  \hat{H}_{ee} = \sum_{i < j}^{N} v_{E}\left( \mathbf{r}_{i}-\mathbf{r}_{j}\right)
  + \frac{1}{2}Nv_{0},
\end{align}
where $v_{E}(\mathbf{r})$ is the effective two-body 
interaction and $v_{0}$ is the self-interaction, defined 
as $v_{0} = \lim_{\mathbf{r} \rightarrow 0} \left\{ v_{E}(\mathbf{r}) - 1/r\right\} $. 

The negative 
electron charges are neutralized by a positive, homogeneous 
background charge. Fraser \emph{et al.} explain how the
electron-background and background-background terms, 
$\hat{H}_{eb}$ and $\hat{H}_{bb}$, vanish
when using Ewald's interaction for the three-dimensional
electron gas. Using the same arguments, one can show that
these terms are also zero in the corresponding 
two-dimensional system. 




\subsection{Ewald correction term}

In the three-dimensional electron gas, the Ewald 
interaction is 
\begin{align}
  v_{E}(\mathbf{r}) &= \sum_{\mathbf{k} \neq \mathbf{0}}
  \frac{4\pi }{L^{3}k^{2}}e^{i\mathbf{k}\cdot \mathbf{r}}
  e^{-\eta^{2}k^{2}/4} \nonumber \\
  &+ \sum_{\mathbf{R}}\frac{1}{\left| \mathbf{r}
    -\mathbf{R}\right| } \mathrm{erfc} \left( \frac{\left| 
    \mathbf{r}-\mathbf{R}\right|}{\eta }\right)
  - \frac{\pi \eta^{2}}{L^{3}},
\end{align}
where $L$ is the box side length, $\mathrm{erfc}(x)$ is the 
complementary error function, and $\eta $ is a free
parameter that can take any value in the interval 
$(0, \infty )$.



\subsection{Interaction in momentum space}

The translational vector 
\begin{align}
  \mathbf{R} = L\left(n_{x}\mathbf{u}_{x} + n_{y}
  \mathbf{u}_{y} + n_{z}\mathbf{u}_{z}\right) ,
\end{align}
where $\mathbf{u}_{i}$ is the unit vector for dimension $i$,
is defined for all integers $n_{x}$, $n_{y}$, and 
$n_{z}$. These vectors are used to obtain all image
cells in the entire real space. 
The parameter $\eta $ decides how 
the Coulomb interaction is divided into a short-ranged
and long-ranged part, and does not alter the total
function. However, the number of operations needed
to calculate the Ewald interaction with a desired 
accuracy depends on $\eta $, and $\eta $ is therefore 
often chosen to optimize the convergence as a function
of the simulation-cell size. In
our calculations, we choose $\eta $ to be an infinitesimally
small positive number, similarly as was done by \href{{https://journals.aps.org/prb/abstract/10.1103/PhysRevB.86.035111}}{Shepherd *et al.*} and \href{{https://journals.aps.org/prb/abstract/10.1103/PhysRevB.88.115138}}{Roggero *et al.*}.

This gives an interaction that is evaluated only in
Fourier space. 

When studying the two-dimensional electron gas, we
use an Ewald interaction that is quasi two-dimensional.
The interaction is derived in three dimensions, with 
Fourier discretization in only two dimensions. The Ewald effective
interaction has the form 
\begin{align}
  v_{E}(\mathbf{r}) &= \sum_{\mathbf{k} \neq \mathbf{0}} 
  \frac{\pi }{L^{2}k}\left\{ e^{-kz} \mathrm{erfc} \left(
  \frac{\eta k}{2} - \frac{z}{\eta }\right)+ \right. \nonumber \\
  & \left. e^{kz}\mathrm{erfc} \left( \frac{\eta k}{2} + \frac{z}{\eta }
  \right) \right\} e^{i\mathbf{k}\cdot \mathbf{r}_{xy}} 
  \nonumber \\
  & + \sum_{\mathbf{R}}\frac{1}{\left| \mathbf{r}-\mathbf{R}
    \right| } \mathrm{erfc} \left( \frac{\left| \mathbf{r}-\mathbf{R}
    \right|}{\eta }\right) \nonumber \\
  & - \frac{2\pi}{L^{2}}\left\{ z\mathrm{erf} \left( \frac{z}{\eta }
  \right) + \frac{\eta }{\sqrt{\pi }}e^{-z^{2}/\eta^{2}}\right\},
\end{align}
where the Fourier vectors $\mathbf{k}$ and the position vector
$\mathbf{r}_{xy}$ are defined in the $(x,y)$ plane. When
applying the interaction $v_{E}(\mathbf{r})$ to two-dimensional
systems, we set $z$ to zero. 


Similarly as in the 
three-dimensional case, also here we 
choose $\eta $ to approach zero from above. The resulting 
Fourier-transformed interaction is
\begin{align}
  v_{E}^{\eta = 0, z = 0}(\mathbf{r}) = \sum_{\mathbf{k} \neq \mathbf{0}} 
  \frac{2\pi }{L^{2}k}e^{i\mathbf{k}\cdot \mathbf{r}_{xy}}. 
\end{align}
The self-interaction $v_{0}$ is a constant that can be 
included in the reference energy.




\subsection{Antisymmetrized matrix elements in three dimensions}

In the three-dimensional electron gas, the antisymmetrized
matrix elements are
\begin{align} \label{eq:vmat_3dheg}
  & \langle \mathbf{k}_{p}m_{s_{p}}\mathbf{k}_{q}m_{s_{q}}
  |\tilde{v}|\mathbf{k}_{r}m_{s_{r}}\mathbf{k}_{s}m_{s_{s}}\rangle_{AS} 
  \nonumber \\
  & = \frac{4\pi }{L^{3}}\delta_{\mathbf{k}_{p}+\mathbf{k}_{q},
    \mathbf{k}_{r}+\mathbf{k}_{s}}\left\{ 
  \delta_{m_{s_{p}}m_{s_{r}}}\delta_{m_{s_{q}}m_{s_{s}}}
  \left( 1 - \delta_{\mathbf{k}_{p}\mathbf{k}_{r}}\right) 
  \frac{1}{|\mathbf{k}_{r}-\mathbf{k}_{p}|^{2}}
  \right. \nonumber \\
  & \left. - \delta_{m_{s_{p}}m_{s_{s}}}\delta_{m_{s_{q}}m_{s_{r}}}
  \left( 1 - \delta_{\mathbf{k}_{p}\mathbf{k}_{s}} \right)
  \frac{1}{|\mathbf{k}_{s}-\mathbf{k}_{p}|^{2}} 
  \right\} ,
\end{align}
where the Kronecker delta functions 
$\delta_{\mathbf{k}_{p}\mathbf{k}_{r}}$ and
$\delta_{\mathbf{k}_{p}\mathbf{k}_{s}}$ ensure that the 
contribution with zero momentum transfer vanishes.


Similarly, the matrix elements for the two-dimensional
electron gas are
\begin{align} \label{eq:vmat_2dheg}
  & \langle \mathbf{k}_{p}m_{s_{p}}\mathbf{k}_{q}m_{s_{q}}
  |v|\mathbf{k}_{r}m_{s_{r}}\mathbf{k}_{s}m_{s_{s}}\rangle_{AS} 
  \nonumber \\
  & = \frac{2\pi }{L^{2}}
  \delta_{\mathbf{k}_{p}+\mathbf{k}_{q},\mathbf{k}_{r}+\mathbf{k}_{s}}
  \left\{ \delta_{m_{s_{p}}m_{s_{r}}}\delta_{m_{s_{q}}m_{s_{s}}} 
  \left( 1 - \delta_{\mathbf{k}_{p}\mathbf{k}_{r}}\right)
  \frac{1}{
    |\mathbf{k}_{r}-\mathbf{k}_{p}|} \right.
  \nonumber \\
  & - \left. \delta_{m_{s_{p}}m_{s_{s}}}\delta_{m_{s_{q}}m_{s_{r}}}
  \left( 1 - \delta_{\mathbf{k}_{p}\mathbf{k}_{s}}\right)
  \frac{1}{ 
    |\mathbf{k}_{s}-\mathbf{k}_{p}|}
  \right\} ,
\end{align}
where the single-particle momentum vectors $\mathbf{k}_{p,q,r,s}$
are now defined in two dimensions.



\subsection{Fock operator}

In the calculations, the 
self-interaction constant is included in the reference 
energy. We therefore get the 
Fock-operator matrix elements 
\begin{align}
  \langle \mathbf{k}_{p}|f|\mathbf{k}_{q} \rangle
  = \frac{\hbar^{2}k_{p}^{2}}{2m}\delta_{\mathbf{k}_{p},
  \mathbf{k}_{q}} + \sum_{\mathbf{k}_{i}}\langle 
  \mathbf{k}_{p}\mathbf{k}_{i}|v|\mathbf{k}_{q}
  \mathbf{k}_{i}\rangle_{AS}.
  \label{eq:fock_heg}
\end{align}
In work of Shepherd \emph{et al.} the matrix elements were 
defined with the self-interaction constant included in the
two-body interaction. This gives Fock-operator matrix 
elements with a gap constant.



\subsection{Periodic boundary conditions and single-particle states}

When using periodic boundary conditions, the 
discrete-momentum single-particle basis functions 
\[
\phi_{\mathbf{k}}(\mathbf{r}) =
e^{i\mathbf{k}\cdot \mathbf{r}}/L^{d/2}
\]
are associated with 
the single-particle energy   
\begin{align}
  \varepsilon_{n_{x}, n_{y}} = \frac{\hbar^{2}}{2m} \left( \frac{2\pi }{L}\right)^{2}\left( n_{x}^{2} + n_{y}^{2}\right)
\end{align}
for two-dimensional sytems and 
\begin{align}
  \varepsilon_{n_{x}, n_{y}, n_{z}} = \frac{\hbar^{2}}{2m}
  \left( \frac{2\pi }{L}\right)^{2}
  \left( n_{x}^{2} + n_{y}^{2} + n_{z}^{2}\right)
\end{align} 
for three-dimensional systems.


We choose  the single-particle basis such that both the occupied and 
unoccupied single-particle spaces have a closed-shell 
structure. This means that all single-particle states 
corresponding to energies below a chosen cutoff are
included in the basis. We study only the unpolarized spin
phase, in which all orbitals are occupied with one spin-up 
and one spin-down electron. 


The table illustrates  how single-particle energies
    fill energy shells in a two-dimensional electron box.
  Here $n_{x}$ and $n_{y}$ are the momentum quantum numbers,
  $n_{x}^{2} + n_{y}^{2}$ determines the single-particle 
  energy level, $N_{\uparrow \downarrow }$ represents the 
  cumulated number of spin-orbitals in an unpolarized spin
  phase, and $N_{\uparrow \uparrow }$ stands for the
  cumulated number of spin-orbitals in a spin-polarized
  system.




\subsection{Magic number for the two-dimensional electron gas}



\begin{quote}
\begin{tabular}{ccccc}
\hline
\multicolumn{1}{c}{ $n_{x}^{2}+n_{y}^{2}$ } & \multicolumn{1}{c}{ $n_{x}$ } & \multicolumn{1}{c}{ $n_{y}$ } & \multicolumn{1}{c}{ $N_{\uparrow \downarrow }$ } & \multicolumn{1}{c}{ $N_{\uparrow \uparrow }$ } \\
\hline
0                     & 0       & 0       & 2                          & 1                        \\
\hline
1                     & -1      & 0       &                            &                          \\
                      & 1       & 0       &                            &                          \\
                      & 0       & -1      &                            &                          \\
                      & 0       & 1       & 10                         & 5                        \\
\hline
2                     & -1      & -1      &                            &                          \\
                      & -1      & 1       &                            &                          \\
                      & 1       & -1      &                            &                          \\
                      & 1       & 1       & 18                         & 9                        \\
\hline
4                     & -2      & 0       &                            &                          \\
                      & 2       & 0       &                            &                          \\
                      & 0       & -2      &                            &                          \\
                      & 0       & 2       & 26                         & 13                       \\
\hline
5                     & -2      & -1      &                            &                          \\
                      & 2       & -1      &                            &                          \\
                      & -2      & 1       &                            &                          \\
                      & 2       & 1       &                            &                          \\
                      & -1      & -2      &                            &                          \\
                      & -1      & 2       &                            &                          \\
                      & 1       & -2      &                            &                          \\
                      & 1       & 2       & 42                         & 21                       \\
\hline
\end{tabular}
\end{quote}

\noindent
\subsection{Hartree-Fock energies}

Finally, a useful benchmark for our calculations is the expression for
the reference energy $E_0$ per particle.
Defining the $T=0$ density $\rho_0$, we can in turn determine in three
dimensions the radius $r_0$ of a sphere representing the volume an
electron occupies (the classical electron radius) as
\[
r_0= \left(\frac{3}{4\pi \rho}\right)^{1/3}.
\]
In two dimensions the corresponding quantity is
\[
r_0= \left(\frac{1}{\pi \rho}\right)^{1/2}.
\]




One can then express the reference energy per electron in terms of the
dimensionless quantity $r_s=r_0/a_0$, where we have introduced the
Bohr radius $a_0=\hbar^2/e^2m$. The energy per electron computed with
the reference Slater determinant can then be written as
(using hereafter only atomic units, meaning that $\hbar = m = e = 1$)
\[
E_0/N=\frac{1}{2}\left[\frac{2.21}{r_s^2}-\frac{0.916}{r_s}\right],
\]
for the three-dimensional electron gas.  For the two-dimensional gas
the corresponding expression is (show this)
\[
E_0/N=\frac{1}{r_s^2}-\frac{8\sqrt{2}}{3\pi r_s}.
\]


For an infinite homogeneous system, there are some particular
simplications due to the conservation of the total momentum of the
particles.  By symmetry considerations, the total momentum of the
system has to be zero. Both the kinetic energy operator and the
total Hamiltonian $\hat{H}$ are assumed to be diagonal in the total
momentum $\mathbf{K}$. Hence, both the reference state $\Phi_{0}$ and
the correlated ground state $\Psi$ must be eigenfunctions of the
operator $\mathbf{\hat{K}}$ with the corresponding eigenvalue
$\mathbf{K} = \mathbf{0}$.  This leads to important
simplications to our different many-body methods. In coupled cluster
theory for example, all
terms that involve single particle-hole excitations vanish. 






% ------------------- end of main content ---------------


% #ifdef PREAMBLE
\printindex

\end{document}
% #endif

